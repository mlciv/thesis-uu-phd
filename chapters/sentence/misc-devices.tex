\section{Models for MISC Prediction}
\label{sec:snt:devices}

Modeling the categorization and forcasting tasks defined in~\S\ref{sec:snt:task} requires
addressing four questions:
\begin{inparaenum}[(1)]
\item How do we encode a dialogue and its utterances?
\item Can we discover discriminative words in each utterance?
\item Can we discover which of the previous utterances are relevant?
\item How do we handle label imbalance in our data?
\end{inparaenum}
%
Many recent advances in neural networks can be seen as plug-and-play
components. To facilitate the comparative study of models, we will
describe components that address the above questions.  % Then, we
% will compare the models and their various ablations as a combination
% of these modules in \S\ref{sec:experiments} and \ref{sec:analysis}
% respectively.
%
In the rest of the chapter, we will use~\textbf{boldfaced} terms to
denote vectors and matrices and \module{small caps} to denote
component names.

% In this section, for better comparative study on modeling
% psychotherapy dialogue our tasks, we first introduce thress
% plug-and-play components to study each of the following questions in
% our tasks. Then for each task, we will describe the best models and
% their ablations by representing it with a combinataion of these
% components in \S\ref{sec:models} and \ref{sec:analysis}.
%\vs{Don't use vague words like
%
%``some''. Be more concrete. Use a number. Here say ``four''. Also,
%  in other places in the paper.}  important
%\vs{\replaceWith{components to resolve each of the following
%    challenges}{design choices we need to address with respect to
%    modeling dialogue}}:
% \begin{enumerate}
% \item How to encode a dialogue and its utterances? (\S\ref{ssec:dialog_rep})
% \item How to capture important words for our tasks? (\S\ref{ssec:word_att})
% \item Can relevance between sentences help for our tasks? (\S\ref{ssec:sentence_att})
% \item How to handling label imbalance in our models ? (\S\ref{ssec:focal_loss})
% \end{enumerate}

% \vs{We spoke about the transfer learning point above.}
% \vs{Maybe these could be phrased as questions, and not in-para
% enum. That way, we could even add a reference to the subsection
% that discusses that question in parenthesis. The subsection titles
%  should use the words from the question. That way, it becomes easy
%  to read.}
%
%\begin{figure*}[t]
%\centering
%\includegraphics[width=0.3\textwidth]{hgru.png}\quad
%\includegraphics[width=0.3\textwidth]{gmlstm.png} \quad
%\includegraphics[width=0.3\textwidth]{multi-head.png} \quad
%\caption{\label{fig:models} (left) Hierarchical GRU, (middle) word-level Gated match-LSTM, (right) snt-level multi-head multi-hop self-attention. \vs{Do not use pngs if possible. Use pdf or eps}}
%\end{figure*}

\subsection{Encoding Dialogue}
\label{ssec:dialog_rep}

%\vs{See comment about subsection titles above}

%\vs{Start each subsection with a description of what the underlying problem is. Here, the problem is: To build any c%lassifier over a dialogue, we need to encode them into a vector form. Dialogues are sequences of utterences, which t% hemselves are sequences of tokens. This hierarchical structure has been widely studied in past work and the typical% approach is to use a hierarchical recurrent neural network [cite]...}

%Through extensive experiments in
%\S\ref{sec:experiments}, we find out the best combination of them for
%each task, and show what helps and what doesn't in each task with
%abalation studies in \S\ref{sec:analysis}.

Since both our tasks are classification tasks over a dialogue
history, our goal is to convert the sequence of utterences into a
single vector that serves as input to the final classifier.

We will use a hierarchical recurrent encoder~\cite[and
others]{li2015hierarchical,sordoni2015hierarchical,serban2016building}
to encode dialogues, specifically a hierarchical gated recurrent
unit (\HGRU) with an utterance and a dialogue encoder. We use a
bidirectional GRU over word embeddings to encode utterances. As is
standard, we represent an utterance $u_i$ by concatenating the final
forward and reverse hidden states. We will refer to this utterance
vector as $\bm{v}_i$. Also, we will use the hidden states of
each word as inputs to the attention components in
\S\ref{ssec:word_att}. We will refer to such contextual word
encoding of the $j^{th}$ word as $\bm{v}_{ij}$. The dialogue encoder
is a unidirectional GRU that operates on a concatenation of
utterance vectors $\bm{v}_i$ and a trainable vector representing the
speaker $s_i$.\footnote{For the dialogue encoder, we use a
  unidirectional GRU because the dialogue is incomplete. For words,
  since the utterances are completed, we can use a BiGRU.} The final
state of the GRU aggregates the entire dialogue history into a
vector $\bm{H}_n$.

% One way to encode a whole dialogue is concatenating all the dialogue
% utterances into a single long sentence and then encode it with a
% sequence encoder such as LSTM. However, it may lose important
% structure information of the dialogue after being flattened. Another
% widely used method is using hierarchical recurrent
% encoder~\cite{li2015hierarchical,
%   yang2016hierarchical,tang2015document, sordoni2015hierarchical,
%   serban2016building, tran2017hierarchical,
%   serban2017multiresolution}. In this paper, we adopt a hierarchical
%   GRU~(HGRU) as our model skeleton.

% An alternative to the \HGRU is to flatten the history into a
% sequence of utterences and use a single biGRU to encode this
% flattened history. We will refer to this strategy for encoding the
% dialogue history as \CON and denote the bidirectional encoding of
% the history as $\bm{C}$ and the hidden states of this BiGRU as
% $\bm{c}_1, \bm{c}_2, \cdots$.

The \HGRU skeleton can be optionally augmented with the
word and dialogue attention described next. All the models we will study are two-layer MLPs over the
vector $\bm{H}_n$ that use a ReLU hidden layer and a softmax layer
for the outputs.

%%% Local Variables:
%%% mode: latex
%%% TeX-master: "../../thesis-main.ltx"
%%% End:


\subsection{Word-level Attention}
\label{ssec:word_att}

Certain words in the utterance history are important to categorize or
forecast MISC labels. The identification of these words may depend on
the utterances in the dialogue. For example, to identify that an
utterance is a simple reflection (\RES) we may need to discover that
the therapist is mirroring a recent client utterance; the example
in table~\ref{tbl:misc} illustrates this. Word attention offers a
natural mechanism for discovering such patterns.

% Words are not equally important for our tasks. When classifying the
% current utterance $u_{n}$ in categorizing task, words in $u_{n}$ are
% usually more important than those in other utterances. For example,
% polarity and behavior related words ``quit'' in $u_{4}$ in Table
% ~\ref{tbl:example} are good indicators for \misc{NEG}, while most
% words in the dialogue history may be helpless. On the other side, only
% words in $u_{n}$ maybe not enough for identifying some codes. For
% example, only if words in $u_{n}$ are overlapping with previous
% utterances, then the label for $u_{n}$ is able about reflection~(e.g.,
% RES, REC, and REF). Hence, it is unclear whether word-attention can
% capture the above patterns.

We can unify a broad collection of attention mechanisms in NLP under
a single high level architecture~\cite{galassi2019attention}. We
seek to define attention over the word encodings $\bv_{ij}$ in the
history (called queries), guided by the word encodings in the anchor
$\bv_{nk}$ (called keys). The output is a sequence of
attention-weighted vectors, one for each word in the $i^{th}$
utterance.  The $j^{th}$ output vector $\ba_j$ is computed as a
weighted sum of the keys:
\begin{equation}
  \ba_{ij} = \sum_{k} \alpha^{k}_{j} \bv_{nk}
\label{eq:att_sum}
\end{equation}
The weighting factor $\alpha^k_j$ is the attention weight between
the $j^{th}$ query and the $k^{th}$ key, computed as
\begin{equation}
\label{eq:att_weight}
\alpha^{k}_{j} = \frac{\exp\left(f_{m}(\bv_{nk}, \bv_{ij})\right)}{\sum_{j^{\prime}} \exp\left(f_{m}(\bv_{nk}, \bv_{ij^\prime})\right)}
\end{equation}
Here, $f_m$ is a match scoring function between the corresponding
words, and different choices give us different attention mechanisms.

Finally, a combining function $f_{c}$ combines the original word
encoding $\bm{v}_{ij}$ and the above attention-weighted word vector
$\bm{a}_{ij}$ into a new vector representation $\bm{z}_{ij}$ as the final
representation of the query word encoding:
\begin{equation}
\bm{z}_{ij}= f_{c}(\bm{v}_{ij}, \bm{a}_{ij})
\end{equation}

The attention module, identified by the choice of the functions
$f_m$ and $f_c$, converts word encodings in each utterance
$\bv_{ij}$ into attended word encodings $\bm{z}_{ij}$. To use them
in the \HGRU skeleton, we will encode them a second time using a
BiGRU to produce attention-enhanced utterance vectors. For brevity,
we will refer to these vectors as $\bv_i$ for the utterance
$u_i$. If word attention is used, these attended vectors will be
treated as word encodings.


\begin{table}[t]
\begin{center}
  \setlength{\tabcolsep}{3pt}
  {\small
    \begin{tabular}{ll|l}
      \toprule
      Method                 & $f_{m} $                                    & $f_{c}$                                                             \\ \hline
      BiDAF                  & \multirow{2}{*}{$\bm{v}_{nk} {\bm{v}_{ij}^{T}}$}             & $[\bm{v}_{ij};~\bm{a}_{ij};  $                                      \\
                             &                                             & $~~\bm{v}_{ij} \odot \bm{a}_{ij};~\bm{v}_{ij}\odot \bm{a}^{\prime}]$ \\ \hline
      \multirow{2}{*}{GMGRU} & $\bm{w}^{e} \tanh(\bm{W}^{k}\bm{v}_{nk}$    & \multirow{2}{*}{$[\bm{v}_{ij};\bm{a}_{ij}]$}                        \\
                             & $~~+ \bm{W}^{q}[\bm{v}_{ij}; \bm{h}_{j-1}])$ &                                                                     \\ \hline
    \end{tabular}
  }
\end{center}
\caption{\label{tbl:word_att} Summary of word attention mechanisms.
  %
  We simplify BiDAF with multiplicative attention between  word
  pairs for $f_{m}$, while GMGRU uses additive attention
  influenced by the GRU hidden state.
  %
  The vector $\bm{w}_{e} \in\mathbb{R}^{d}$, and matrices
  $\bm{W}^{k}\in \mathbb{R}^{d \times d}$ and
  $\bm{W}^{q} \in\mathbb{R}^{2d \times 2d}$ are parameters of the BiGRU. The vector $\bm{h}_{j-1}$
  is the hidden state from the BiGRU in GMGRU at previous position
  $j-1$.  
  %
  For combination function, BiDAF concatenates bidirectional
  attention information from both the key-aware query vector
  $\ba_{ij}$ and a similarly defined query-aware key vector
  $\ba^{\prime}$. GMGRU uses simple concatenation for $f_c$.}
\end{table}

To complete this discussion, we need to instantiate the two
functions. We use two commonly used attention mechanisms:
BiDAF~\cite{bidaf} and gated matchLSTM~\cite{wang2017gated}. For
simplicity, we replace the sequence encoder in the latter with a
BiGRU and refer to it as GMGRU. Table~\ref{tbl:word_att} shows the
corresponding definitions of $f_{c}$ and $f_{m}$. We refer the
reader to the original papers for further details. In subsequent
sections, we will refer to the two attended versions of the \HGRU as
\BiDAFH and \GMGRUH.

% By specifying different sentence inputs as query and key, we summarize
% the inputs and outputs when using word-level attention over our
% skeleton CONCAT and HGRU dialogue encoding in Table
% \ref{tbl:word_att_q_key}. Worth to mention, we always involve the
% current utterance into dialogue history to weight with itself. Hence,
% it simultaneously captures the word interaction within the current
% utterance $n$ itself.

% \begin{table}[t]
% \begin{center}
% \setlength{\tabcolsep}{3pt}
% \begin{tabular}{lccc}
% \toprule
% Skeleton & Query   & Key  & Options\\ \hline 
% CON & $u_{n}$ & $C_{n}$ &  $\text{BiDAF}^{C}$,$\text{GMGRU}^{C}$ \\ 
% HGRU & $u_{i}$ & $u_{n}$ & $\text{BiDAF}^{H}$,$\text{GMGRU}^{H}
% $  \\ \bottomrule
% \end{tabular}
% \end{center}
% \caption{\label{tbl:word_att_q_key} For CON dialogue encoding, the
%   output of word attention is history-aware $u_{n}$ for
%   inference. While for HGRU, it output a query-aware history utterance
%   for each $u_{i}$. Then all the query-aware utterances will be
%   aggeragated by dialogue encoder to make a query-aware dialogue
%   representation $H_{n}$ for inference, }

% \end{table}

%In this paper, instead of inventing better attention mechanism for our tasks,
%we focused on to analysis how much word-level attention needed for our
%different task settings, and how it helps in our tasks. Hence, we
%choose to adopt a general gated extension from \cite{wang2017gated},
%where an extra gate was added to the input of the word attention
%aggregator to control how much attention in the vector $\bm{z}_{ij}$
%will be used when aggregating.
%
%\begin{equation}
%\bm{z}^{*}_{ij} = \text{sigmoid}(\bm{W}^{g}\bm{z}_{ij})\cdot\bm{z}_{ij}
%\end{equation}
%where $\bm{W}^{g} \in$

% % Match-LSTM is ported from textual % %entailment task to capture word-level
% attention between premise and %hypothesis.
% %In our model, we treat the response $r$
% as a premise and each %utterance in the
% dialogue history as a hypothesis. %


%\vs{I don't understand this paragraph at all. Please expand, with
%  equations if necessary. It should be accessible to those who
%  haven't read the match-LSTM paper.}

%\vs{Yes, equations will help. I don't understand what this is.}

%%% Local Variables:
%%% mode: latex
%%% TeX-master: "main"
%%% End:


\subsection{Utterance-Level Attention}
\label{ssec:sentence_att}
While we assume that the history of utterances is available for both
our tasks, not every utterance is relevant to decide a MISC label.
%
For categorization, the relevance of an utterance to the anchor may
be important. For example, a complex reflection (\REC) may depend on
the relationship of the current therapist utterance to one or more
of the previous client utterances. For forecasting, since we do not
have an utterance to label, several previous utterances may
be relevant. For example, in the conversation in
Table~\ref{tbl:snt:misc-example}, both $u_2$ and $u_4$ may be used to
forecast a complex reflection.

To model such utterance-level attention, we will employ the
multihead, multihop attention mechanism used in Transformer
networks~\citep{NIPS2017_7181}. As before, due to space constraints,
we refer the reader to the original work for details. We will use
the $(\bm{Q}, \bm{K}, \bm{V})$ notation from the original paper
here. These matrices represent a query, key and value,
respectively. The multihead attention is defined as:
%
\begin{equation}
\label{eq:multihead_attention}
{\small \text{Multihead}(\bm{Q},\bm{K},\bm{V}) = [\text{head}_{1};\cdots; \text{head}_{h}]\bm{W}^{O}}
\end{equation}
\begin{equation}
 \text{head}_{i} = \text{softmax}\left(\frac{\bm{Q}\bm{W}^{Q}_{i}\left(\bm{K}\bm{W}^{K}_{i}\right)^T}{\sqrt{d_{k}}}\right)\bm{V}\bm{W}^{V}_{i}
\end{equation}
The $\bm{W}_i$'s refer to projection matrices for the three inputs,
and the final $\bm{W}^o$ projects the concatenated heads into a
single vector.

The choices of the query, key and value defines the attention
mechanism. In our work, we compare two variants: {\em anchor-based
  attention}, and {\em self-attention}. The anchor-based attention
is defined by $Q = [\bm{v}_{n}]$ and
$K=V=[\bm{v}_{1} \cdots \bm{v}_{n}]$.  Self-attention is defined by
setting all three matrices to $[\bm{v}_{1} \cdots \bm{v}_{n}]$.
%
For both settings, we use four heads and stacking them for two hops,
and refer to them as $\self$ and $\anchor$.


%%% Local Variables:
%%% mode: latex
%%% TeX-master: "../../dissertation-main.ltx"
%%% End:



\subsection{Predicting and Training}
\label{ssec:inference_and_training}

From top to the bottom, every component will
produce some of useful representation for inferences in our tasks.
The dialogue encoding vector $\bm{H}_{n}$, as the final of the
unidirectional GRU, it is also the contextual utterance encoding
$\bm{h}_{u_{n}}$ of $u_{n}$. Hence $H_{n}$ can be directly used as a
representaion of $u_{n}$ for classification in annotating tasks, also
can be used as a represention of whole dialogue for forecasting
task. Hence in HGRU setting, we always use $\bm{H}_{n}$ as the base
option as input for inference.

However, for CON skeleton, the final state $C_{n}$ doest not exactly
represent the segement $u_{n}$ in the whole concatnated
dalogue. Hence, we concatenate the hidden state of the start
position~(0) and end position~(T) of $u_{n}$ into
$\bm{v}^{seg}_{n}=[h_{u^{0}_{n}};h_{u^{T}_{n}}]$, which is contextual
utterance encoding in CON mode.

Beside the above $\bm{H}_{n}$ and $\bm{C}_{n}$ contextual utterance
encoding in dialogue level, our components also produced the original
utterance encoding $\bm{v}_{n}$ from utterance encoder.  Whats'more,
in \textbf{CON} mode, we can use history-aware utterance encoding $\bm{v}^{wordatt}_{n}$
While in HGRU, it produced a self-attentive utterance encoding. We
denote it as $\bm{v}^{selfatt}_{n}$.

We summarize the option input encodings for inference
in~\autoref{tbl:inference_options}. There are two ways to scoring withthese
inputs, one is to score the concatenated those vectors together,
denoted as $\text{concat}(A, B)=\text{MLP}([A;B)$; The other one is
scoring each of them first and then add the scores up as the final
scores, such as $\text{add}(A,B)=\text{MLP}(A)+\text{MLP}(B)$.


\begin{table}[t]
\caption{\label{tbl:inference_options} Input options for annotating and forecasting tasks based on CON and HGRU skeletons.}
\begin{center}
\setlength{\tabcolsep}{3pt}
\begin{tabular}{lcc}
\toprule
Skeleton & Categorization & Forecasting  \\ \hline \hline
CON      & $\bm{v}^{seg}_{n}$,$\bm{v}^{{wordatt}_{n}}$, $\bm{v}_{n}$ & $\bm{C}_{n}$ \\ \hline
HGRU     & $\bm{H}_{n}$,$\bm{v}^{selfatt}_{n}$,$\bm{v}_{n}$          & $\bm{H}_{n}$ \\ \bottomrule
\end{tabular}
\end{center}
\end{table}


%%% Local Variables:
%%% mode: latex
%%% TeX-master: "../../dissertation-main.ltx"
%%% End:


\subsection{Addressing Label Imbalance}
\label{ssec:focal_loss}
From Table \ref{tbl:misc}, we see that both client and therapist
labels are imbalanced. Moreover, rarer labels are more important in
both tasks. For example, it is important to identify \CHANGE and
\SUSTAIN utterances. For therapists, it is crucial to flag MI
non-adherent (\MIN) utterances; seasoned therapists are trained to
avoid them because they correlate negatively with patient
improvements. If not explicitly addressed, the frequent but less
useful labels can dominate predictions.

% Previous
% work\cite{can2015dialog,tanana2016comparison, xiao2016behavioral}
% shows if not handing the label imbalance problem, results on client
% code may be dominated by most frequent label FN, and rarely predict POS
% and NEG; while for models on therapist codes, many of the rare labels,
% such as MIA, MIN will almost zero F1 score.
% One common solution for class imbalance is the $\alpha$-balanced
% cross entropy that weights different classes in standard cross
% entropy loss (CE) with $\alpha \in [0,1]$. Although easily
% classified labels may have small cross entropy value, the large
% number of them (for frequent labels) may still overwhelm the total
% cross entropy loss.

To address this, we extend the focal loss~\cite[FL][]{lin2017focal}
to the multiclass case. For a label $l$ with probability produced by
a model $p_t$, the loss is defined as 
\begin{equation}
 \label{eq:focal}
\text{FL}(p_{t}) = -\alpha_{t} {(1 -p_{t})}^{\gamma} \log(p_{t})
\end{equation}
In addition to using a label-specific balance weight $\alpha_t$, the
loss also includes a modulating factor ${(1-p_{t})}^{\gamma}$ to
dynamically downweight well-classified examples with
$p_{t}\gg0.5$. Here, the $\alpha_t$'s and the $\gamma$ are
hyperparameters. We use FL as the default loss function for all our
models.



%%% Local Variables:
%%% mode: latex
%%% TeX-master: "main"
%%% End:



%\subsection{Multiple Task Learning}
\label{ssec:mtl}

In this section, we studied whether the 4 settings in our tasks can benifit from each other.
To study the relation between annotating tasks ans forecasting task, we jointly learning the annotating tasks and forecasting task for the same speaker role as the C\_T and T\_JOINT.  To study, the relation between the client tasks and therapist task, we propose the alternative setting, our model alternative optimize for different tasks. Hence, we propose CT\_ANNO  and CT\_FORE for alternative learning for annotating tasks and forecasting task. Finally, we also proposed CT\_ALL for both alternative and jointly together, which learn our 4 settings all in one. We summerize our multple task learning settings in Table \ref{tbl:mtl}. Their experiment results are shown in Table \ref{tbl:mtl_result}

\begin{table}[tp]
\setlength{\tabcolsep}{3pt}
\begin{center}
\begin{tabular}{l|cc|cc}
\toprule
\hline
\multirow{2}{*}{ Method } & \multicolumn{2}{c|}{ Client } & \multicolumn{2}{c}{ Therapist } \\\cmidrule(lr){2-3}  \cmidrule(lr){4-5}
                          & Annotate                      & Forecast & Annotate & Forecast  \\ \midrule \midrule
C\_JOINT                  & Y                             & Y        &          &           \\
T\_JOINT                  &                               &          & Y        & Y         \\\midrule
CT\_ANNO                  & Y                             &          & Y        &           \\
CT\_FORE                  &                               & Y         &         & Y         \\ \midrule
CT\_ALL                   & Y                             & Y        & Y        & Y         \\ \hline
\bottomrule
\end{tabular}
\end{center}
\caption{\label{tbl:mtl} Summary of Multiple Task Learning on our paper}
\end{table}

%%% Local Variables:
%%% mode: latex
%%% TeX-master: "main"
%%% End:


%%% Local Variables:
%%% mode: latex
%%% TeX-master: "../../thesis-main.ltx"
%%% End:
