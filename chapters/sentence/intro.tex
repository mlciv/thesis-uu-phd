\section{Introduction}
\label{sec:intro}
Conversational agents have long been studied in the context of
psychotherapy, going back to chatbots such as
ELIZA~\cite{weizenbaum1966eliza} and
PARRY~\cite{colby1975artificial}. Research in modeling such dialogue
has largely sought to simulate a participant in the conversation.

In this paper, we argue for modeling dialogue \emph{observers}
instead of participants, and focus on psychotherapy. An observer
could help an \emph{ongoing} therapy session in several ways.
First, by monitoring fidelity to therapy standards, a helper could
guide both veteran and novice therapists towards better patient
outcomes. Second, rather than generating therapist utterances, it
could suggest the type of response that is appropriate. Third, it
could alert a therapist about potentially important cues from a
patient.
%
Such assistance would be especially helpful in the increasingly
prevalent online or text-based counseling services.\footnote{For
  example, Crisis Text Line (\url{https://www.crisistextline.org}),
  7 Cups (\url{https://www.7cups.com}), etc.}

%% putting the table here to make it in the second page.
\begin{table*}[ht]
  \begin{center}
\setlength{\tabcolsep}{4pt}
{\small
\begin{tabular}{llll}
  \toprule
  {\bf Code}            & {\bf Count}            & {\bf Description}                                                                                            & {\bf Examples}                                    \\
  \midrule \midrule
  \multicolumn{4}{c}{ \bf Client Behavioral Codes }                                                                                                                                                                 \\
  \midrule
  \multirow{2}{*}{\FN}  & \multirow{2}{*}{47715} & \multirow{2}{*}{\parbox{5.5cm}{Follow/ Neutral: unrelated to changing or sustaining behavior.}}              & ``You know, I didn't smoke for a while.''         \\
                        &                        &                                                                                                              & ``I have smoked for forty years now.''            \\ 
  \CHANGE               & 5099                   & Utterances about changing unhealthy  behavior.                                                                          & ``I want to stop smoking.''                       \\ 
  \SUSTAIN              & 4378                   & Utterances about sustaining unhealthy behavior.                                                                        & ``I really don't think I smoke too much.''        \\ \midrule
  \midrule
  \multicolumn{4}{c}{\bf Therapist Behavioral Codes }                                                                                                                                                               \\
  \midrule
  \FA                   & 17468                  & Facilitate conversation                                                                                      & ``Mm Hmm.'', ``OK.'',``Tell me more.''            \\
  \GI                   & 15271                  & Give information or feedback.                                                                                & ``I'm Steve.'', ``Yes, alcohol is a depressant.'' \\
  \multirow{2}{*}{\RES} & \multirow{2}{*}{6246}  & \multirow{2}{*}{\parbox{5.5cm}{Simple reflection about the client’s most recent utterance.}}                 & C: ``I didn't smoke last week''                   \\
                        &                        &                                                                                                              & T: ``Cool, you avoided smoking last week.''       \\ 
  \multirow{2}{*}{\REC} & \multirow{2}{*}{4651}  & \multirow{2}{*}{\parbox{5.5cm}{Complex reflection based on a client's history or the broader conversation.}} & C: ``I didn't smoke last week.''                  \\
                        &                        &                                                                                                              & T: ``You mean things begin to change''.           \\ 
  \QUC                  & 5218                   & Closed question                                                                                              & ``Did you smoke this week?''                      \\ 
  \QUO                  & 4509                   & Open question                                                                                                & ``Tell me more about your week.''                 \\ 
  \multirow{2}{*}{\MIA} & \multirow{2}{*}{3869}  & \multirow{2}{*}{\parbox{5.5cm}{Other MI adherent,\eg, affirmation, advising with permission, etc.}}          & ``You've accomplished a difficult task.''         \\
                        &                        &                                                                                                              & ``Is it OK if I suggested something?''            \\ 
  \multirow{2}{*}{\MIN} & \multirow{2}{*}{1019}  & \multirow{2}{*}{\parbox{5.5cm}{MI non-adherent, \eg, confrontation, advising without permission, etc.}}      & ``You hurt the baby's health for cigarettes?''    \\
                        &                        &                                                                                                              & ``You ask them not to drink at your house.''      \\\bottomrule
\end{tabular}}
\end{center}
\caption{Distribution, description and examples of MISC labels.} 
\label{tbl:misc}
\end{table*}

%%

We ground our study in a style of therapy called Motivational
Interviewing~\cite[MI,][]{miller2003motivational,miller2012motivational},
which is widely used for treating addiction-related problems.
%
To help train therapists, and also to monitor therapy quality,
utterances in sessions are annotated using a set of behavioral codes
called Motivational Interviewing Skill
Codes~\cite[MISC,][]{miller2003manual}. Table~\ref{tbl:misc} shows
standard therapist and patient (\ie, client) codes with
examples. Recent NLP work~\cite[][{\em inter
  alia}]{tanana2016comparison, xiao2016behavioral,
  perez2017predicting, huang2018modeling} has studied the problem of
using MISC to assess \emph{completed} sessions.  Despite its
usefulness, automated post hoc MISC labeling does not address the
desiderata for ongoing sessions identified above; such models use
information from utterances yet to be said. To provide real-time
feedback to therapists, we define two complementary dialogue
observers:
\begin{enumerate}[nosep] 
\item \textbf{Categorization}: Monitoring an ongoing session by
  predicting MISC labels for therapist and client utterances as they
  are made.
\item \textbf{Forecasting}: Given a dialogue history, forecasting
  the MISC label for the next utterance, thereby both alerting or
  guiding therapists.
\end{enumerate}
Via these tasks, we envision a helper that offers assistance to a
therapist in the form of MISC labels.

We study modeling challenges associated with these tasks related to:
\begin{inparaenum}[(1)]
\item representing words and utterances in therapy dialogue,
\item ascertaining relevant aspects of utterances and the dialogue
  history, and
\item handling label imbalance (as evidenced in
  Table~\ref{tbl:misc}).
\end{inparaenum}
We develop neural models that address these challenges in this
domain.

Experiments show that our proposed models outperform baselines by a
large margin. For the categorization task, our models even
outperform previous session-informed approaches that use information
from future utterances. For the more difficult forecasting task, we
show that even without having access to an utterance, the dialogue
history provides information about its MISC label.  We also report
the results of an ablation study that shows the impact of the
various design choices.\footnote{The code is available online at
\url{https://github.com/utahnlp/therapist-observer}.}.% Additionally, we will host
% a publicly accessible online demo of our best models.

In summary, in this paper, we
\begin{inparaenum}[(1)]
\item define the tasks of categorizing and forecasting Motivational
  Interviewing Skill Codes to provide real-time assistance to
  therapists,
\item propose neural models for both tasks that outperform several
  baselines, and
\item show the impact of various modeling choices via extensive
  analysis.
\end{inparaenum}


%%% Local Variables:
%%% mode: latex
%%% TeX-master: "main"
%%% End:
