\subsection{Can We Suggest Empathetic Responses?}
\label{ssec:sentence:good-MI}
% In therapy, there is no correct answer to what constitutes the best
% type of response for a specific client statement. However, 
% We can use examples from sessions that expert human raters
% identified as better to demonstrate what might be suitable types of
% utterances to follow a client statement.

Our forecasting models are trained on regular MI sessions, according
to the label distribution on~\autoref{tbl:bg:misc}, there are both MI
adherent or nonadherent data. Hence, our models are trained to show
how the therapist usually respond to a given statement.

To show whether our model can mimic {\em good} MI policies, we
selected 35 MI sessions from our test set which were rated 5 or
higher on a 7-point scale empathy or spirit. On these sessions, we
still achieve a recall@3 of 76.9, suggesting that we can learn good
MI policies by training on all therapy sessions. These results
suggest that our models can help train new
therapists who may be uncertain about how to respond to a client.

%  For Recall@3 metric, our model can still achieve
% 76.9 comparing to 77.0 in our test set, if we always choose the most
% frequent label \FA, \GI, \RES, it only can achieve 66.6
% Recall@3. \footnote{label distribution and the detailed performance
%   on each of the theraspit codes are in the suppelementary
%   material}.

% In the field of psychotherapy, one of the major difficulties is that
% new therapists do not always know how to respond to a client.  It is
% not always appropriate to respond to every statement with an open
% question or a reflection (though both are desirable from the
% perspective of motivational interviewing).  This type of model could
% be used as scaffolding to help train new psychotherapists to learn the
% types of responses an expert might give to a specific statement.


%%% Local Variables:
%%% mode: latex
%%% TeX-master: "../../dissertation-main.ltx"
%%% End:
