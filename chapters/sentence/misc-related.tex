\section{Related Work}
\label{sec:sentential:related}
In this section, we review the related work on MISC code prediction,
different MISC code clustering strategies, and recent advances on
dialogue representation and attention mechanism.

\subsection{MISC Code Prediction}
\label{ssec:sentential:misc-related}

MEMM and CRF with handcrafted features are firstly proposed by
\citet{can2012case, can2015dialog} for the MISC code prediction task
. Then \citet{tanana2015recursive} improved the model by incorporating
richer dependency relation features, which even outperformed a
proposed recursive neural network baseline model. Recently,
\cite{xiao2016behavioral} proposed to use GRU with domain-specific word
embedding and weighted cross-entropy loss to resolve the label
imbalance problem. This paper studies the solutions drawn from
recent work in dialogue representation, memory attention, and
imbalanced classification. Besides that, other improvements exist,
such as topic models based domain adaptio~\cite{atkins2014scaling,
  huang2018modeling}, and prosodic features \cite{weber2002using} have
been proposed to improve the prediction tasks.

\section{Different Clustering Strategies for MISC}
\label{sec:misc_clustering}

\begin{table}[!h]
  \begin{center}
\setlength{\tabcolsep}{4pt}
{\small
\begin{tabular}{llll}
  \toprule
{\bf Code}           & {\bf Count}            & {\bf Description}                                                                                                                                                                                                     & {\bf Examples}                                      \\ \hline \hline
\multirow{6}{*}{\MIA} & \multirow{6}{*}{3869}  & \multirow{6}{*}{\parbox{5.5cm}{Group of MI Adherent codes : Affirm(\misc{AF}); Reframe(\misc{RF}); Emphasize Control(\misc{EC}); Support(\misc{SU}); Filler(\misc{FI}); Advise with permission(\misc{ADP}); Structure(\misc{ST}); Raise concern with permission(\misc{RCP})}} & ``You've accomplished a difficult task.''~(\misc{\misc{AF}})      \\
                     &                        &                                                                                                                                                                                                                       & ``It’s your decision whether you quit or not''~(\misc{EC}) \\
                     &                        &                                                                                                                                                                                                                       & ``That must have been difficult.''~(\misc{SU})             \\
                     &                        &                                                                                                                                                                                                                       & ``Nice weather today!''~(\misc{FI})                        \\
                     &                        &                                                                                                                                                                                                                       & ``Is it OK if I suggested something?''~(\misc{ADP})        \\
                     &                        &                                                                                                                                                                                                                       & ``Let's go to the next topic''~(\misc{ST})                 \\
                     &                        &                                                                                                                                                                                                                       & ``Frankly, it worries me.''~(\misc{RCP})                   \\  \hline
\multirow{5}{*}{\MIN} & \multirow{5}{*}{1019}  & \multirow{5}{*}{\parbox{5.5cm}{Group of MI Non-adherent codes: Confront(\misc{CO}); Direct(\misc{DI}); Advise without permission(\misc{ADW}); Warn(\misc{WA}); Raise concern without permission(\misc{RCW})}}                                            & ``You hurt the baby's health for cigarettes?''~(\misc{CO}) \\
                     &                        &                                                                                                                                                                                                                       & ``You need to xxx.''~(\misc{DI})                           \\
                     &                        &                                                                                                                                                                                                                       & ``You ask them not to drink at your house.''~(\misc{ADW})  \\
                     &                        &                                                                                                                                                                                                                       & ``You will die if you don't stop smoking.''~(\misc{WA})    \\
                     &                        &                                                                                                                                                                                                                       & ``You may use it again with your friends.''~(\misc{RCW})   \\ \bottomrule
\end{tabular}}
\end{center}
\caption{\label{tbl:misc_mia_min} Label distribution, description and exmaples for \MIA and \MIN}
\end{table}

\Paragraph{MISC-28} The original MISC description of
\citet{miller2003manual} included 28 labels (9 client, 19
therapist).

\Paragraph{MISC-8} Due to data scarcity and label confusion, some
labels were merged into a coarser set.  \citet{can2015dialog} retain 6
original labels \FA, \GI, \QUC, \QUO, \REC, \RES, and merge remaining
13 rare labels into a single \misc{COU} label, they merge all 9 client
codes into a single \misc{CLI} label.

\Paragraph{MISC-15} Instead, \citet{tanana2016comparison} merge only 8
of rare labels in therapist codes into a \misc{OTHER} label and they
cluster client codes into 3 labels according to the valence of
changing, sustaining or being neutral on the addictive
behavior~\cite{atkins2014scaling}.

\Paragraph{MISC-11} Then \citet{xiao2016behavioral} combine and
improve above two clustering strategies by splitting the all 13 rare
labels according to whether the code represents MI-adherent(\MIA) and
MI-nonadherent (\MIN) We show more details about the original labels
in \MIA and \MIN in Table \ref{tbl:misc_mia_min}

%%% Local Variables:
%%% mode: latex
%%% TeX-master: "../../thesis-main.ltx"
%%% End:


\subsection{Dialogue Representation and Hierarchical Encoder }
\label{ssec:sentential:dialogue-encoder}

End-to-End neural networks and attention mechanisms have been widely
used in many natural language tasks, such as text classification,
question answering, dialogue systems, etc.  RNN, CNN, and
Transformer has been quite effective for modeling the sequence of
words~\cite{ kim14cnn,zhang2015character}. Combinations of them have
also been used to model the hierarchical structure of document and
dialogue context. They first use CNN or LSTM to get a sentence vector,
and then a BiGRU to compose a sentence vector to get a document
vector \citep{tang2015document, li2015hierarchical,
  yang2016hierarchical,sordoni2015hierarchical, serban2016building,
  serban2017multiresolution}. In our paper, we use hierarchical GRU as
skeletons. Other hierarchical combinations may also help, we leave it
for future studies.

\subsection{Attention Mechanism}
\label{ssec:sentential:attention}

Attention mechanism was first proposed by \citet{bahdanau2014neural}
in machine translation, then various of extensions has been invented
and widely used in other tasks, especially for question answering and
dialogue
system\cite{matchlstm,bidaf,sukhbaatar15mnet,fei17gmnet,P18-1157}.
Attention on sentence-level representation also helps in many recent
works such as abstractive summarization~\cite{P18-1013}, dialogue
state tracking~\cite{zhou2018multi,zhou2016multi}, document
classification~\cite{yang2016hierarchical}.

Previous work on hierarchical attention\cite{yang2016hierarchical}
tries always to use both word-level and sentence-level attention in a
model, In this chapter, we argue that, for different tasks, we need
different levels attentions. Always adding two-level attention may be
not necessary. Recently, multi-heads multi-hop attention used in
Transformer \cite{NIPS2017_7181} became a more attractive attention
mechanism. All the above advances in NLP inspired us to systematically
analyze whether or how they impact our tasks.

%%% Local Variables:
%%% mode: latex
%%% TeX-master: "../../dissertation-main.ltx"
%%% End:
