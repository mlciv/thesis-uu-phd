\section{Related Work}
\label{sec:sentential:related}
In this section, we review the related work on MISC code prediction,
different MISC code clustering strategies, and recent advances on
dialogue representation and attention mechanism.

\subsection{MISC Code Prediction}
\label{ssec:sentential:misc-related}
MEMM and CRF with handcrafted features are firstly proposed by
\citet{can2012case, can2015dialog} for MISC code prediction task
. Then \citet{tanana2015recursive} improved the model by incorporating
richer dependency relation features, which even outperformed a
proposed recursive neural network baseline model. Recently,
\cite{xiao2016behavioral} propose to use GRU with domain-specific word
embedding, and weighted cross-entropy loss to resolve the label
imbalance problem. In this paper, we study the solutions draw from
recent work in dialogue representation, memory attention, and
imbalanced classification. Besides that, another improvement exists,
such as topic models based domain adaptio~\cite{atkins2014scaling,
  huang2018modeling}, and prosodic features \cite{weber2002using} has
been proposed to improve the prediction tasks.

\section{Main Clustering Strategies}
\label{sec:misc:other-clustering}
This section summarizes the existing clustering stategies of MISC
codes as follows:
\begin{itemize}
\item \kw{MISC-28:} The original MISC description of
  \citet{miller2003manual} included 28 labels (9 client codes, 19
  therapist codes). However, among them, there are 13 therapist codes
  are too rare to learn an efficient model on predicting them.

\item \kw{MISC-8:} Hence, \citet{can2015dialog} retain 6 original
  codes \FA, \GI, \QUC, \QUO, \REC, \RES, and merge the remaining 13
  rare therapist codes into a single \misc{COU} label.  Furthermore,
  they merge all 9 client codes into a single \misc{CLI} label. In
  total, they have 7 labels for therapists and 1 label for clients.

\item \kw{MISC-15:} Instead, \citet{tanana2016comparison} merge only 8
  of 13 rare labels in therapist codes into a customized label named
  \misc{OTHER}, and then they cluster client codes into 3 labels
  according to the valence of changing, sustaining or being neutral on
  the addictive behavior~\citep{atkins2014scaling}. Hence, in total,
  they have 12 labels for therapists and 3 labels for clients.

\item \kw{MISC-11:} Then \citet{xiao2016behavioral} combine and
  improve above the two clustering strategies~(MISC-8 and MISC-15) by
  splitting all the 13 rare labels according to whether the code
  represents MI-adherent(\MIA) and MI-nonadherent (\MIN). In total,
  they have 8 therapist codes and 3 client codes as shown
  in~\autoref{tbl:bg:misc}~(\autoref{sssec:bg:dialogue-act}). We show
  more details about the original labels in the group \MIA and \MIN
  in~\autoref{tbl:misc_mia_min}~(\autoref{sec:misc:our-clustering}).
\end{itemize}

\section{Details of MISC-11}
\label{sec:misc:our-clustering}
In \kw{MISC-11}, 13 rare labels are grouped into two set: 8 labels
representing MI-adherent(\MIA) and 5 labels representing
MI-nonadherent (\MIN). \autoref{tbl:misc_mia_min} shows in the
distribution of the grouped \MIA and \MIN labels. Compared to the
other labels in~\autoref{tbl:bg:misc}, the grouped \MIA and \MIN are
still relative less frequent. Hence, we use the focal loss to resolve
the label imbalance issue, and improve the overall performance. The
examples in~\autoref{tbl:misc_mia_min} will help to understand the
meanings of the rare labels. For more details about each of the MISC
code, please refer to the latest MISC annotation
guideline~\citep{houck2012motivational}.
\begin{table}[!h]
\caption{\label{tbl:misc_mia_min} Label distribution, description and exmaples for \MIA and \MIN.}
  \begin{center}
\setlength{\tabcolsep}{3pt}
{\scriptsize
\begin{tabular}{llll}
  \toprule
 \hline
{\bf Code}           & {\bf Count}            & {\bf Description}                                                                                                                                                                                                     & {\bf Examples}                                      \\ \hline \hline
\multirow{6}{*}{\MIA} & \multirow{6}{*}{3869}  & \multirow{6}{*}{\parbox{5.5cm}{Group of 8 MI Adherent codes : Affirm(\misc{AF}); Reframe(\misc{RF}); Emphasize Control(\misc{EC}); Support(\misc{SU}); Filler(\misc{FI}); Advise with permission(\misc{ADP}); Structure(\misc{ST}); Raise concern with permission(\misc{RCP})}} & ``You've accomplished a difficult task.''~(\misc{\misc{AF}})      \\
                     &                        &                                                                                                                                                                                                                       & ``It's your decision whether you quit or not.''~(\misc{EC}) \\
                     &                        &                                                                                                                                                                                                                       & ``That must have been difficult.''~(\misc{SU})             \\
                     &                        &                                                                                                                                                                                                                       & ``Nice weather today!''~(\misc{FI})                        \\
                     &                        &                                                                                                                                                                                                                       & ``Is it OK if I suggested something?''~(\misc{ADP})        \\
                     &                        &                                                                                                                                                                                                                       & ``Let's go to the next topic.''~(\misc{ST})                 \\
                     &                        &                                                                                                                                                                                                                       & ``Frankly, it worries me.''~(\misc{RCP})                   \\  \hline
\multirow{5}{*}{\MIN} & \multirow{5}{*}{1019}  & \multirow{5}{*}{\parbox{5.5cm}{Group of 5 MI-nonadherent codes: Confront(\misc{CO}); Direct(\misc{DI}); Advise without permission(\misc{ADW}); Warn(\misc{WA}); Raise concern without permission(\misc{RCW})}}                                            & ``You hurt the baby's health for cigarettes?''~(\misc{CO}) \\
                     &                        &                                                                                                                                                                                                                       & ``You need to xxx.''~(\misc{DI})                           \\
                     &                        &                                                                                                                                                                                                                       & ``You ask them not to drink at your house.''~(\misc{ADW})  \\
                     &                        &                                                                                                                                                                                                                       & ``You will die if you don't stop smoking.''~(\misc{WA})    \\
                     &                        &                                                                                                                                                                                                                       & ``You may use it again with your friends.''~(\misc{RCW})   \\ \hline
\bottomrule
\end{tabular}}
\end{center}
\end{table}

%%% Local Variables:
%%% mode: latex
%%% TeX-master: "../../dissertation-main.ltx"
%%% End:


\subsection{Dialogue Representation and Hierarchical Encoder }
\label{ssec:sentential:dialogue-encoder}
End-to-End neural networks and attention mechanism has been widely
used in many natural language tasks, such as text classification,
question answering, dialogue system and so on.  RNN, CNN, and
Transformer has been quite effective for modeling sequence of
words~\cite{ kim14cnn,zhang2015character}. Combinations of them have
also been used to model the hierarchical structure of document, and
dialogue context. They first use CNN or LSTM to get a sentence vector,
and then a BiGRU to compose a sentence vectors to get a document
vector \cite{tang2015document, li2015hierarchical,
  yang2016hierarchical,sordoni2015hierarchical, serban2016building,
  serban2017multiresolution}. In our paper, we use hierarchical GRU as
skeletons. Other hierarhical combination may also help, we leave it
for future studies.

\subsection{Attention Mechanism}
\label{ssec:sentential:attention}
Attention mechanism was first proposed by \citet{bahdanau2014neural}
in machine translation, then various of extensions has been invented
and widely used in other tasks, especially for question answering and
dialogue
system\cite{matchlstm,bidaf,sukhbaatar15mnet,fei17gmnet,P18-1157}.
Attention on sentence-level representation also helps in many recent
works such as abstractive summarization~\cite{P18-1013}, dialogue
state tracking~\cite{zhou2018multi,zhou2016multi}, document
classification~\cite{yang2016hierarchical}.

Previous work on hierarchical attention\cite{yang2016hierarchical}
tries always to use both word-level and sentence -level attention in a
model, In this paper, we argue that, for different tasks, we need
different levels attentions. Always adding two-level attention may be
not necessary.  Recently, multi-heads multi-hops attention used in
Transformer \cite{NIPS2017_7181} became a more attractive attention
mechanism. All above advances in NLP inspired us to do systematic
analysis on whether or how they impact on our tasks.

%%% Local Variables:
%%% mode: latex
%%% TeX-master: "../../thesis-main.ltx"
%%% End:
