\subsection{Word-Level Attention}
\label{ssec:word_att}

Certain words in the utterance history are important to categorize or
forecast MISC labels. The identification of these words may depend on
the utterances in the dialogue. For example, to identify that an
utterance is a simple reflection (\RES) we may need to discover that
the therapist is mirroring a recent client utterance; the example
in~\autoref{tbl:bg:misc} illustrates this. Word attention offers a
natural mechanism for discovering such patterns.

% Words are not equally important for our tasks. When classifying the
% current utterance $u_{n}$ in categorizing task, words in $u_{n}$ are
% usually more important than those in other utterances. For example,
% polarity and behavior related words ``quit'' in $u_{4}$ in Table
% ~\ref{tbl:example} are good indicators for \misc{NEG}, while most
% words in the dialogue history may be helpless. On the other side, only
% words in $u_{n}$ maybe not enough for identifying some codes. For
% example, only if words in $u_{n}$ are overlapping with previous
% utterances, then the label for $u_{n}$ is able about reflection~(e.g.,
% RES, REC, and REF). Hence, it is unclear whether word-attention can
% capture the above patterns.

We can unify a broad collection of attention mechanisms in NLP under
a single high level architecture~\cite{galassi2019attention}. We
seek to define attention over the word encodings $\bv_{ij}$ in the
history (called queries), guided by the word encodings in the anchor
$\bv_{nk}$ (called keys). The output is a sequence of
attention-weighted vectors, one for each word in the $i^{th}$
utterance.  The $j^{th}$ output vector $\ba_j$ is computed as a
weighted sum of the keys:
\begin{equation}
  \ba_{ij} = \sum_{k} \alpha^{k}_{j} \bv_{nk}
\label{eq:att_sum}
\end{equation}

The weighting factor $\alpha^k_j$ is the attention weight between
the $j^{th}$ query and the $k^{th}$ key, computed as
\begin{equation}
\label{eq:att_weight}
\alpha^{k}_{j} = \frac{\exp\left(f_{m}(\bv_{nk}, \bv_{ij})\right)}{\sum_{j^{\prime}} \exp\left(f_{m}(\bv_{nk}, \bv_{ij^\prime})\right)}
\end{equation}

Here, $f_m$ is a match scoring function between the corresponding
words, and different choices give us different attention mechanisms.

Finally, a combining function $f_{c}$ combines the original word
encoding $\bm{v}_{ij}$ and the above attention-weighted word vector
$\bm{a}_{ij}$ into a new vector representation $\bm{z}_{ij}$ as the final
representation of the query word encoding:
\begin{equation}
\bm{z}_{ij}= f_{c}(\bm{v}_{ij}, \bm{a}_{ij})
\end{equation}

The attention module, identified by the choice of the functions
$f_m$ and $f_c$, converts word encodings in each utterance
$\bv_{ij}$ into attended word encodings $\bm{z}_{ij}$. To use them
in the \HGRU skeleton, we will encode them a second time using a
BiGRU to produce attention-enhanced utterance vectors. For brevity,
we will refer to these vectors as $\bv_i$ for the utterance
$u_i$. If word attention is used, these attended vectors will be
treated as word encodings.


\begin{table}[t]
\caption{\label{tbl:word_att} Summary of word attention mechanisms.}
\begin{center}
  \setlength{\tabcolsep}{3pt}
  {
    \begin{tabular}{c|c|c}
      \toprule
      \hline
      Method & $f_{m} $                                                                          & $f_{c}$                                                 \\ \hline
      BiDAF  & $\bm{v}_{nk} {\bm{v}_{ij}^{T}}$                                  & $[\bm{v}_{ij};~\bm{a}_{ij};~\bm{v}_{ij} \odot \bm{a}_{ij};~\bm{v}_{ij}\odot \bm{a}^{\prime}]$ \\ \hline
      GMGRU  & $\bm{w}^{e} \tanh(\bm{W}^{k}\bm{v}_{nk} + \bm{W}^{q}[\bm{v}_{ij}; \bm{h}_{j-1}])$ & $[\bm{v}_{ij};\bm{a}_{ij}]$            \\ \hline
      \bottomrule
    \end{tabular}
  }
\end{center}
\end{table}

To complete this discussion, we need to instantiate the two
functions. We use two commonly used attention mechanisms:
BiDAF~\cite{bidaf} and gated matchLSTM~\cite{wang2017gated}. For
simplicity, we replace the sequence encoder in the latter with a BiGRU
and refer to it as GMGRU. Table~\ref{tbl:word_att} shows the
corresponding definitions of $f_{c}$ and $f_{m}$. We simplify BiDAF
with multiplicative attention between word pairs for $f_{m}$, while
GMGRU uses additive attention influenced by the GRU hidden state.  The
vector $\bm{w}_{e} \in\mathbb{R}^{d}$, and matrices
$\bm{W}^{k}\in \mathbb{R}^{d \times d}$ and
$\bm{W}^{q} \in\mathbb{R}^{2d \times 2d}$ are parameters of the BiGRU. The
vector $\bm{h}_{j-1}$ is the hidden state from the BiGRU in GMGRU at
previous position $j-1$.  For combination function, BiDAF concatenates
bidirectional attention information from both the key-aware query
vector $\ba_{ij}$ and a similarly defined query-aware key vector
$\ba^{\prime}$. GMGRU uses simple concatenation for $f_c$.We refer the
reader to the original papers for further details. In subsequent
sections, we will refer to the two attended versions of the \HGRU as
\BiDAFH and \GMGRUH.

% By specifying different sentence inputs as query and key, we summarize
% the inputs and outputs when using word-level attention over our
% skeleton CONCAT and HGRU dialogue encoding in Table
% \ref{tbl:word_att_q_key}. Worth to mention, we always involve the
% current utterance into dialogue history to weight with itself. Hence,
% it simultaneously captures the word interaction within the current
% utterance $n$ itself.

% \begin{table}[t]
% \begin{center}
% \setlength{\tabcolsep}{3pt}
% \begin{tabular}{lccc}
% \toprule
% Skeleton & Query   & Key  & Options\\ \hline 
% CON & $u_{n}$ & $C_{n}$ &  $\text{BiDAF}^{C}$,$\text{GMGRU}^{C}$ \\ 
% HGRU & $u_{i}$ & $u_{n}$ & $\text{BiDAF}^{H}$,$\text{GMGRU}^{H}
% $  \\ \bottomrule
% \end{tabular}
% \end{center}
% \caption{\label{tbl:word_att_q_key} For CON dialogue encoding, the
%   output of word attention is history-aware $u_{n}$ for
%   inference. While for HGRU, it output a query-aware history utterance
%   for each $u_{i}$. Then all the query-aware utterances will be
%   aggeragated by dialogue encoder to make a query-aware dialogue
%   representation $H_{n}$ for inference, }

% \end{table}

%In this paper, instead of inventing better attention mechanism for our tasks,
%we focused on to analysis how much word-level attention needed for our
%different task settings, and how it helps in our tasks. Hence, we
%choose to adopt a general gated extension from \cite{wang2017gated},
%where an extra gate was added to the input of the word attention
%aggregator to control how much attention in the vector $\bm{z}_{ij}$
%will be used when aggregating.
%
%\begin{equation}
%\bm{z}^{*}_{ij} = \text{sigmoid}(\bm{W}^{g}\bm{z}_{ij})\cdot\bm{z}_{ij}
%\end{equation}
%where $\bm{W}^{g} \in$

% % Match-LSTM is ported from textual % %entailment task to capture word-level
% attention between premise and %hypodissertation.
% %In our model, we treat the response $r$
% as a premise and each %utterance in the
% dialogue history as a hypodissertation. %


%\vs{I don't understand this paragraph at all. Please expand, with
%  equations if necessary. It should be accessible to those who
%  haven't read the match-LSTM paper.}


%%% Local Variables:
%%% mode: latex
%%% TeX-master: "../../dissertation-main.ltx"
%%% End:
