\section{Background and Motivation} \label{sec:background}
Motivational Interviewing (MI) is a style of psychotherapy that
seeks to resolve a client's ambivalence towards their problems,
thereby motivating behavior change. Several meta-analyses and
empirical studies have shown the high efficacy and success of MI in
psychotherapy~\cite{burke2004emerging, martins2009review,
  lundahl2010meta}. However, MI skills take practice to master and
require ongoing coaching and feedback to
sustain~\cite{Schwalbe2014}.  Given the emphasis on using specific
types of linguistic behaviors in MI (\eg,
open questions and reflections), fine-grained behavioral coding
plays an important role in MI theory and training.

Motivational Interviewing Skill Codes (MISC, table~\ref{tbl:misc})
is a framework for coding MI sessions. It facilitates evaluating
therapy sessions via utterance-level labels that are akin to
dialogue acts~\cite{stolcke2000dialogue,jurafsky2018speech}, and are designed to examine therapist and client
behavior in a therapy session.\footnote{The original MISC description of
  \citet{miller2003manual} included 28 labels (9 client, 19
  therapist). Due to data scarcity and label confusion, various
  strategies are proposed to merge the labels into a coarser set.
  We adopt the grouping proposed by~\citet{xiao2016behavioral}; the
  appendix gives more details.}
% Like dialogue acts in general
% goal-oriented dialogue~\cite{jurafsky2018speech}, therapist codes
% characterize counselor and client behavior in a.
% Additionally, some therapist codes also mark the adherence to or
% against the spirit of MI. Client behavior
% codes are designed to characterize behavior change rather than the
% function of an utterance in the dialogue.

As Table~\ref{tbl:misc} shows,
client labels mark utterances as discussing changing or sustaining
problematic behavior (\CHANGE and \SUSTAIN, respectively) or being
neutral (\FN). Therapist utterances are grouped into eight labels,
some of which (\RES, \REC) correlate with improved outcomes, while
MI non-adherent (\MIN) utterances are to be avoided.  MISC labeling
was originally done by trained annotators performing multiple passes
over a session recording or a transcript.  Recent NLP work speeds up
this process by automatically annotating a completed MI
session~\cite[\eg,][]{tanana2016comparison, xiao2016behavioral,
  perez2017predicting}.

\emph{Instead of providing feedback to a therapist after the
  completion of a session, can a dialogue observer provide online
  feedback?} While past work has shown the helpfulness of post hoc
evaluations of a session, prompt feedback would be more helpful,
especially for MI non-adherent responses.  Such feedback opens up
the possibility of the dialogue observer influencing the therapy
session. It could serve as an assistant that offers suggestions to a
therapist (novice or veteran) about how to respond to a client
utterance. Moreover, it could help alert the therapist to
potentially important cues from the client (specifically, \CHANGE or
\SUSTAIN).