
\subsection{Preprocessing and Model Setup}
\label{ssec:snt:exp_setup}
% The standard Motivational Interview Skill Codes (MISC)
% ~\cite{miller2003manual, houck2012motivational} contains 28 codes
% including 9 client codes and 19 therapist codes.  But some of them are
% too rare to build reasonable models. To address this issue, in
% previous works, several merging strategies have been proposed to group
% them into new categories \cite{can2015dialog, tanana2016comparison,
%   xiao2016behavioral}. In this paper, we follow the grouping strategy
% from \citet{xiao2016behavioral}. As shown in Table \ref{tbl:misc}, our
% dataset includes 3 client codes and 8 therapist codes.

%As shown in Table~\ref{tbl:misc}, for
%client codes, based on whether they are towards or away from to Target
%Behavior Change~(TBC), they are categorized into three codes: POS, NEG,
%and FN. Respectively, they represent the valence of changing,
%sustaining or following or neutral to the addictive behaviors
%~\cite{atkins2014scaling}. For therapist codes, we retain six original
%MISC code~(FA, GI, RES, REC, QUC, QUO). For other 13 rare codes,
%according to whether they represent MI adherent and MI non-adherent in
%the original MISC code annotation guideline, we cluster 8 of them into
%MIA and the other five into MIN. Table \ref{tbl:misc} shows the brief
%explanation and examples for final label set we used in our dataset
%and their distribution in our training corpus.
%% \vs{Too complicated para. Split somehow}

An MI session contains about 500 utterances on average. We use a
sliding window of size $N=8$ utterances with padding for the initial
ones. We assume that we always know the identity of the speaker for
all utterances. Based on this, we split the sliding windows into a
client and therapist windows to train separate models.
%
We tokenized and lower-cased utterances using
spaCy~\cite{spacy2}. To embed words, we concatenated 300-dimensional
Glove embeddings~\cite{pennington2014glove} with ELMo
vectors~\cite{Peters:2018}.

We use 300-dimensional Glove embeddings pretrained on 840B tokens
from Common Crawl~\cite{pennington2014glove}. We do not update the
embedding during training. Tokens not covered by Glove are using a
randomly initialized UNK embedding. We also use character-level deep
contextualized embedding ELMo 5.5B model by concatenating the
corresponding ELMo word encoding after the word embedding vector. For
speaker information, we randomly initialize them with 8 dimensional
vectors and update them during training. We used a dropout rate of 0.3
for the embedding layers.

We trained all models using Adam~\cite{kingma2014adam} with learning
rate chosen by cross validation between $[1e^{-4}, 5*1e^{-4}]$,
gradient norms clipping from at $[1.0, 5.0]$, and minibatch sizes of
32 or 64. We use the same hidden size for both utterance encoder,
dialogue encoder and other attention memory hidden size; it has been
selected from $\{64, 128, 256, 512\}$. We set a smaller dropout 0.2
for the final two fully connected layers. All the models are trained
for 100 epochs with early-stoping based on macro $\text{F}_{1}$ over
development results.


% To
% simplify our model setting and simulate the realtime dialogue
% setting, for every dialogue utternace, imagining it as the most
% current utterance as dilogue goes on, then we only select its
% preceding $N$ utterances with their speaker and label information as
% our dialogue history(We use N= 8 for our main models, and use
% $N \leq 16$ for ablation study). We add padding empty utterances
% with special padding speaker and padding label ahead of the original
% context utterances when there is not enough preceding utterances.

% The speaker for every utterance is always given information, including
% the goal utterances, which are the last utterance to be categorized
% aid the future utterance to be forecasted\footnote{In real practise,
%   we don't know the future speaker, instead we forecast with models
%   for both client and therapist setting }. According to the speaker
% label of the goal utterances, we further split our dataset into client
% dataset and therapist dataset, and build seperate models for client
% and therapist on each of them respectively, which leads to reduced
% label confusion.

%%% Local Variables:
%%% mode: latex
%%% TeX-master: "../../dissertation-main.ltx"
%%% End:
