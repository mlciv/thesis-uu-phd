\section{Background and Motivation}
\label{sec:snt:background}
As Table~\ref{tbl:bg:misc} shows, client labels mark utterances as
discussing changing or sustaining problematic behavior (\CHANGE and
\SUSTAIN, respectively) or being neutral (\FN). Therapist utterances
are grouped into eight labels, some of which (\RES, \REC) correlate
with improved outcomes, while MI non-adherent (\MIN) utterances are to
be avoided.  MISC labeling was originally done by trained annotators
performing multiple passes over a session recording or a transcript.
Recent NLP work speeds up this process by automatically annotating a
completed MI session~\citep[\eg,][]{tanana2016comparison,
  xiao2016behavioral, perez2017predicting}.

\emph{Instead of providing feedback to a therapist after the
  completion of a session, can a dialogue observer provide online
  feedback?} While past work has shown the helpfulness of post hoc
evaluations of a session, prompt feedback would be more helpful,
especially for MI non-adherent responses.  Such feedback opens up
the possibility of the dialogue observer influencing the therapy
session. It could serve as an assistant that offers suggestions to a
therapist (novice or veteran) about how to respond to a client
utterance. Moreover, it could help alert the therapist to
potentially important cues from the client (specifically, \CHANGE or
\SUSTAIN). % From a technical perspective, such real-time feedback
% poses similar challenges as the task of predicting dialogue intent
% in goal-oriented dialogue systems.



% In this paper, we are curious about whether NLP technologies can
%help in the above real-time setting. Although global ratings must be
%performed after an MI session finished, however, the utterance-level
%MISC code can be annotated in real-time as NLU component used to
%annotate dialogue intent in a goal-oriented dialogue system. Inspired
%but distinguished from current goal-oriented dialogue agent, we don't
%want to build a bot to substitute the therapist to chat with clients
%directly. Instead, we propose to build a novel third-party assistant
%agent, who not only can give rational judgement and alert to the
%therapist by monitoring what just happened in current dialogue, but
%also it can offer suggestion by forecasting the future
%conversation. According to the motivation above, we will first
%decompose and formulate the monitoring and forecasting ability into
%two seperate NLP tasks in \S\ref{sec:task}, then we describe our
%proposed models in \S\ref{sec:devices} and \S\ref{sec:models}.

%According to the manual of MISC~\cite{miller2003manual,
%houck2012motivational}, the original goal of MISC is to evaluate the
%quality of MI session, beyond the function of each utterance, the
%behavior codes also emphasize on 6 counselor rating dimensions:
%Acceptance, Empathy, Direction, Autonomy support, Collaboration, and
%Evocation, while the client MISC code are used to evaluate the target
%behavior change(TBC).\vs{The last sentence is too long. And it seems
%somewhat orthogonal}

%%% Local Variables:
%%% mode: latex
%%% TeX-master: "../../dissertation-main.ltx"
%%% End:
