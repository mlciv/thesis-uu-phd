\section{Independent Factorization for Motivational Interviewing}
\label{sec:snt:task}
In this section, we will formally define the two NLP tasks
corresponding to the vision in~\S\ref{sec:snt:background} using the
conversation in~\autoref{tbl:snt:misc-example} as a running
example. Suppose we have an ongoing MI session with utterances
$u_1, u_2,\cdots, u_n$: together, the dialogue history $H_n$.  Each
utterance $u_i$ is associated with its speaker $s_i$, either C
(client) or T (therapist). We will define two classification tasks
over a fixed dialogue history with $n$ elements ---
\emph{categorization}~\S\ref{ssec:snt:task-categorization} and
\emph{forecasting}~\S\ref{ssec:snt:task-forcasting}.  Then we make a
comparative analysis on the independent factorization for the two
tasks~\S\ref{ssec:snt:comparative-task}.

%We will refer to the last utterance $u_n$ as the \emph{anchor} for the
%corresponding subtasks.  As the conversation progresses, the history
%will be updated with a sliding window.  Since the therapist and client
%codes share no overlap, we will design separate models for the two
%speakers, giving us four settings in all.

\begin{table}[tp]
  \begin{center}
    \setlength{\tabcolsep}{3pt}
    {
      \begin{tabular}{crll}
        \toprule
        $i$        & $s_{i}$ & $u_{i}$                             & $l_{i}$  \\ \hline
        1        & T:      & Have you used drugs recently?       & \QUC     \\
        2        & C:      & I stopped for a year, but relapsed. & \FN      \\
        3        & T:      & You will suffer if you keep using. & \MIN     \\
        4        & C:      & Sorry, I just want to quit.         & \CHANGE  \\
        $\cdots$ &         & $\cdots$                            & $\cdots$ \\ \bottomrule
      \end{tabular}
    }
  \end{center}
  \caption{\label{tbl:snt:misc-example} An example of ongoing therapy session}
\end{table}

\subsection{Task 1: Categorization}
\label{ssec:snt:task-categorization}
The goal of this task is to provide real-time feedback to a
therapist during an ongoing MI session. In the running example, the
therapist's confrontational response in the third utterance is not
MI adherent (\MIN); an observer should flag it as such to bring the
therapist back on track. The client's response, however, shows an
inclination to change their behavior (\CHANGE). Alerting a therapist
(especially a novice) can help guide the conversation in a direction
that encourages it.

In previous post-hoc dialogue analysis setting, we have the following
definition for the categorization task: \emph{Given the dialogue
  history $H_n=\{u_{1},u_{2}, ..., u_{n}\}$ which includes the speaker
  information, output a sequence of MISC label
  $L_n=\{l_{1}, l_{2}, ..., l_{n}\}$ for each utterance $u_i$.}

In essence, we have the following real-time classification task:
\emph{Given the dialogue history $H_n$ which includes the speaker
  information, predict the MISC label $l_n$ for the last utterance
  $u_n$.}

The key difference from previous work in predicting MISC labels is
that we are restricting the input to the real-time setting. As a
result, models can only use the dialogue history to predict the label,
and in particular, we can not use models such as a conditional random
field or a bi-directional LSTM that need both past and future inputs.

% \jc{I added this for comparing to
% previous dialogue act classification}
% \vs{I removed the content because it repeats information from
% section 3}

% For the client codes, it is similar to the sentiment analysis but it
% towards more on behavior changes, e.g., the client may feel happy
% and confident that smoking may help to relax, but this is \misc{NEG}
% in our cases. For therapist codes, this task can be considered as
% dialogue intent classification but emphasize professional counseling
% skills and spirits, such as empathy, support, and evocation.


\subsection{Task 2: Forecasting}
\label{ssec:snt:task-forcasting}
A real-time therapy observer may be thought of as an expert
therapist who guides a session with suggestions to the therapist.
For example, after a client discloses their recent drug use
relapse, a novice therapist may respond in a confrontational manner
(which is not recommended, and hence coded \MIN). On the other
hand, a seasoned therapist may respond with a complex reflection
(\REC) such as \emph{``Sounds like you really wanted to give up and
  you're unhappy about the relapse.''}
Such an expert may also anticipate important cues
from the client.% , expressing their intention to change or sustain
% their behavior
% .
The MISC forecasting task is a previously unstudied problem in the
posthoc dialogue analysis.

The forecasting task seeks to mimic the intent of such a seasoned
therapist: \emph{Given a dialogue history $H_n$ and the next speaker's
  identity $s_{n+1}$, predict the MISC code $l_{n+1}$ of the yet
  unknown next utterance $u_{n+1}$.}
% We
% argue that it is possible to learn to anticipate information about
% future utterances; this forms the basis of chatbot
% development.
We argue that forecasting the type of the next utterance, rather than
selecting or generating its text as has been the focus of several
recent lines of work~\citep[\eg,][]{schatzmann2005quantitative,ubuntu,DSTC7},
allows the human in the loop~(the therapist) the freedom to
creatively participate in the conversation within the parameters
defined by the seasoned observer, and perhaps even rejecting
suggestions. Such an observer could be especially helpful for
training therapists~\citep{imel2017technology}.
%
The forecasting task is also related to recent work on detecting
antisocial comments in online
conversations~\citep{zhang2018conversations} whose goal is to
provide an early warning for such events.


\subsection{Independent Factorization for Categorization and
  Forcasting Task}
\label{ssec:snt:comparative-tasks}
Take the dialogue segment in~\autoref{tbl:snt:misc-example} as a
running example, we compare the categorization and forcaseting task
for each turn, with respect to the independent factorization for the
dialogue history and predicting target. Both categorization and
forcasting tasks are taking a sequence of dialogue utterances as
input, and then predict a sequence of MISC labels as output. They
share the similar sequence labelling structures, and seems sharing the
same independent factorization. However, they have different
predicting goals when considering it for each dialogue turn.
\autoref{tbl:snt:task-comparasion} shows the detailed differences when
we choose dialogue history window size as 3.

\Paragraph{Output Decomposition} The output of both tasks are a
squence of MISC labels. Because each MISC label is naturally
associated with a corresponding utterance, and each of them represent
different functions as shown in \autoref{tbl:misc}. Hence for
independent factorization, it is natural to decompose the whole
squence prediction into a set of continous single MISC label
prediction task as defined previously. In such way, each MISC
prediction in categorization or forcasting task are independent from
the MISC prediction in the previous dialogue turn. However, they have
different goals for each dialogue turn. When the dialogue goes to the
turn 3, both the categorization and forcasting task will the observe
the current dialog history window as input $X=H_{n}$. However, the key
difference is as follows: the categorization task is to predict the
MISC code $l_{n}$ for the last seen utterance $u_{n}$, while the
forcasting task is to guess the future MISC code $l_{n+1}$ for the
unseen utterance $u_{n+1}$.

\Paragraph{Input Decomposition and Alignments Discovery}
For the input decomposition in previous lexical-anchoring and
phrasal-anchoring, the corresponding output label can be directly
derived from each input segment. For example, in lexical anchoring,
each concept or subgraph in an AMR graph is aligned to the each token
or special entity. In phrasal anchoring, a non-terminal intent and
slot label in TOP is aligned to a phrase in the input
sentence. However, in MISC code prediction, 

Due to the realtime setting, the input dialogue is naturually
decomposed by time steps, and forms a sequence of incremental dialgoue
history $\{u_1\}$, $\{u_1,u_2\}$, and $\{u_1,u_2,u_3\}$.  Worth to
mention, when the dialogue goes to the turn 4, the dialog history will
slide to the next window of size 3, as $H_{n}=\{u_2,u_3,u_4\}$. The
$u_{1}$ will be truncated due the sliding window. We limit the
dialogue window because the whole therapy dialogue session may last
for 500 utterances, where current neural models such as RNN and
transformer can not handle the long context well.  In this thesis
study~\autoref{ssec:snt:abl_context_attention}, we compare the window
size as 8 and 16 for our models. We leave the extrem long dialogue
context encoding as the future work.

\begin{table}[!h]
\begin{center}{
\setlength{\tabcolsep}{3pt}
\begin{tabular}{l|c|c|c}
\toprule
\hline
\multirow{2}{*}{Turn} & \multirow{2}{*}{$X=H_{n}$} & Categorization & Forcasting  \\
                      &                            & $Y=l_{n}$      & $Y=l_{n+1}$ \\ \hline
 1                    & $\{u_1\}$                  & {\QUC}         & {\FN}       \\
 2                    & $\{u_1,u_2\}$              & {\FN}          & {\MIN}      \\
 3                    & $\{u_1,u_2,u_3\}$          & {\MIN}         & {\CHANGE}   \\
 4                    & $\{u_2,u_3,u_4\}$          & {\CHANGE}      & {\RES}      \\ \hline
  \bottomrule
\end{tabular}}
\end{center}
\caption{\label{tbl:snt:task-comparasion} Differences between the
  categorization task and the forcasting task, when choosing a window
  size as 3 to factorize the dialog sequential flow}
\end{table}


%%% Local Variables:
%%% mode: latex
%%% TeX-master: "../../thesis-main.ltx"
%%% End:
