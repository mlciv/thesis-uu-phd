\section{Task Definitions}
\label{sec:snt:task}

In this section, we will formally define the two NLP tasks
corresponding to the vision in \S\ref{sec:background} using the
conversation in table~\ref{tbl:example} as a running example.

Suppose we have an ongoing MI session with utterances
$u_1, u_2,\cdots, u_n$: together, the dialogue history $H_n$.  Each
utterance $u_i$ is associated with its speaker $s_i$, either C
(client) or T (therapist). Each utterance is also associated with
the MISC label $l_i$, which is the object of study. We will refer to
the last utterance $u_n$ as the \emph{anchor}.

We will define two classification tasks over a fixed dialogue
history with $n$ elements --- \emph{categorization} and
\emph{forecasting}. As the conversation progresses, the history will
be updated with a sliding window.  Since the therapist and client
codes share no overlap, we will design separate models for the two
speakers, giving us four settings in all.

\begin{table}[tp]
  \begin{center}
    \setlength{\tabcolsep}{2pt}
    {\small
      \begin{tabular}{crll}
        \toprule
        $i$        & $s_{i}$ & $u_{i}$                             & $l_{i}$  \\ \hline
        1        & T:      & Have you used drugs recently?       & \QUC     \\
        2        & C:      & I stopped for a year, but relapsed. & \FN      \\
        3        & T:      & You will suffer if you keep using. & \MIN     \\
        4        & C:      & Sorry, I just want to quit.         & \CHANGE  \\
        $\cdots$ &         & $\cdots$                            & $\cdots$ \\ \bottomrule
      \end{tabular}
    }
  \end{center}
  \caption{\label{tbl:example} An example of ongoing therapy session}
\end{table}

\Paragraph{Task 1: Categorization}
%\label{ssec:task_categorizing}
The goal of this task is to provide real-time feedback to a
therapist during an ongoing MI session. In the running example, the
therapist's confrontational response in the third utterance is not
MI adherent (\MIN); an observer should flag it as such to bring the
therapist back on track. The client's response, however, shows an
inclination to change their behavior (\CHANGE). Alerting a therapist
(especially a novice) can help guide the conversation in a direction
that encourages it.

In essence, we have the following real-time classification task:
\emph{Given the dialogue history $H_n$ which includes the speaker
  information, predict the MISC label $l_n$ for the last utterance
  $u_n$.}

The key difference from previous work in predicting MISC labels is
that we are restricting the input to the
real-time setting. As a result, models can only use the dialogue
history to predict the label, and in particular, we can not use models
such as a conditional random field or a bi-directional LSTM that need both past and future inputs. 
% \jc{I added this for comparing to
% previous dialogue act classification}
% \vs{I removed the content because it repeats information from
% section 3}

% For the client codes, it is similar to the sentiment analysis but it
% towards more on behavior changes, e.g., the client may feel happy
% and confident that smoking may help to relax, but this is \misc{NEG}
% in our cases. For therapist codes, this task can be considered as
% dialogue intent classification but emphasize professional counseling
% skills and spirits, such as empathy, support, and evocation.


\Paragraph{Task 2: Forecasting}
% \label{ssec:task_forcasting}
A real-time therapy observer may be thought of as an expert
therapist who guides a session with suggestions to the therapist.
For example, after a client discloses their recent drug use
relapse, a novice therapist may respond in a confrontational manner
(which is not recommended, and hence coded \MIN). On the other
hand, a seasoned therapist may respond with a complex reflection
(\REC) such as \emph{``Sounds like you really wanted to give up and
  you're unhappy about the relapse.''}
Such an expert may also anticipate important cues
from the client.% , expressing their intention to change or sustain
% their behavior
% .

The forecasting task seeks to mimic the intent of such a seasoned therapist:
\emph{Given a dialogue history $H_n$ and the next speaker's identity
  $s_{n+1}$, predict the MISC code $l_{n+1}$ of the yet unknown next
  utterance $u_{n+1}$.}

The MISC forecasting task is a previously unstudied problem. %  We
% argue that it is possible to learn to anticipate information about
% future utterances; this forms the basis of chatbot
% development.
We argue that forecasting the type of the next utterance, rather than
selecting or generating its text as has been the focus of several
recent lines of work~\cite[\eg,][]{schatzmann2005quantitative,ubuntu,DSTC7},
allows the human in the loop (the therapist) the freedom to
creatively participate in the conversation within the parameters
defined by the seasoned observer, and perhaps even rejecting
suggestions. Such an observer could be especially helpful for
training therapists~\cite{imel2017technology}.
%
The forecasting task is also related to recent work on detecting
antisocial comments in online
conversations~\cite{zhang2018conversations} whose goal is to
provide an early warning for such events.

% Recent work on patient-like conversational agent also
% helps for psyctherapy training. 
% We leave this text generation tasks
% for our future work. \todo{@Mike @Vivke Please help to check this citation and related reference text}
% Furthermore, from an operational perspective, we can use
% standard multi-class evaluations instead of addressing the problem
% of evaluating generated text.

%\begin{numcases}{}
%\operatorname*{arg\,max}_{l \in L_{C}} p(l_{n+1}=l|H_{n}, s_{n+1}=C) \label{eq:fc}\\
%\operatorname*{arg\,max}_{l \in L_{T}} p(l_{n+1}=l|H_{n}, s_{n+1}=T) \label{eq:ft}
%\end{numcases}
%
%\subsection{Task Extentions}
%\label{ssec:task_ext}
%What's more, as our model annotate current utterance at each turn, we
%also can incorprate the predicted MISC label $l^{\prime}_{i}$ into our
%dialogue history, then we have labeled dialogue history
%$H^{l}_{n}$. For $i < n$, instead of a pair $(u_{i}, s_{i})$, we have a
%triple $(u_{i}, s_{i}, l^{\prime}_{i})$. Hence, the labeled dialogue
%history $H^{l}_{n}$ can substitute the unlabeled dialogue history
%$H_{n}$ for all above 4 settings. We will discuss this extenions to
%explore how those predicted labels helps in our tasks.


%\vs{Needs a good segue into the next para. Maybe a paragraph saying
%  that we need to monitor an ongoing therapy session, as opposed to
%  analyze a session after it is done.}

%\vs{Do we need to introduce the symbols $n$ and $T$ here? If they
%  are not used anywhere, we can leave it dots. Something like
%  $\{a_1, a_2, \cdots\}$. Also use $\cdots$ or $\ldots$ for ellipses
%  instead of three dots.}

% \vs{We need to say that the focus of this paper is the NLP question
%  of whether such online prediction and suggestion is even possible,
%  not the therapeutic aspects.}

%%% Local Variables:
%%% mode: latex
%%% TeX-master: "main"
%%% End:
