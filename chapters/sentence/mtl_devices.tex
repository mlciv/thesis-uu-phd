\subsection{Multiple Task Learning}
\label{ssec:mtl}

In this section, we studied whether the 4 settings in our tasks can benifit from each other.
To study the relation between annotating tasks ans forecasting task, we jointly learning the annotating tasks and forecasting task for the same speaker role as the C\_T and T\_JOINT.  To study, the relation between the client tasks and therapist task, we propose the alternative setting, our model alternative optimize for different tasks. Hence, we propose CT\_ANNO  and CT\_FORE for alternative learning for annotating tasks and forecasting task. Finally, we also proposed CT\_ALL for both alternative and jointly together, which learn our 4 settings all in one. We summerize our multple task learning settings in Table \ref{tbl:mtl}. Their experiment results are shown in Table \ref{tbl:mtl_result}

\begin{table}[tp]
\setlength{\tabcolsep}{3pt}
\begin{center}
\begin{tabular}{l|cc|cc}
\toprule
\multirow{2}{*}{ Method } & \multicolumn{2}{c|}{ Client } & \multicolumn{2}{c}{ Therapist } \\\cmidrule(lr){2-3}  \cmidrule(lr){4-5}
                          & Annotate                      & Forecast & Annotate & Forecast  \\ \midrule \midrule
C\_JOINT                  & Y                             & Y        &          &           \\
T\_JOINT                  &                               &          & Y        & Y         \\\midrule
CT\_ANNO                  & Y                             &          & Y        &           \\
CT\_FORE                  &                               & Y         &         & Y         \\ \midrule
CT\_ALL                   & Y                             & Y        & Y        & Y         \\ \midrule
\bottomrule
\end{tabular}
\end{center}
\caption{\label{tbl:mtl} Summary of Multiple Task Learning on our paper}
\end{table}

%%% Local Variables:
%%% mode: latex
%%% TeX-master: "main"
%%% End:
