\section{Conclusion}
\label{sec:snt:conclusion}

We addressed the question of providing real-time assistance to
therapists and proposed the tasks of categorizing and forecasting MISC
labels for an ongoing therapy session. By developing a modular family
of neural networks for these tasks, we show that our models outperform
several baselines by a large margin.  Ablation studies on history
size, word-level and sentence-level attention, focal loss and generic
or domain-specific embeddings, shows how each of them helps in our
tasks. Extensive analysis shows that our model can decrease the label
confusion compared to previous work, especially for reflections and
rare labels. The experimental results show that word-level and sentence-level attention
mechanism can help our independent factorization to discover the
relevant parts for our local factor modelling without the hard
alignment.

Furthermore, our model trained on regular MI sessions with both MI
adherent and non-adherent utterances can still achieve a recall@3 of
76.9 on selected good MI sessions. Hence our model potentially helps
for therapist training on psychotherapy domain. Beyond the strategy of
mimic good therapy dialogue session, in the future, we also can extend
the study to other strategies. For example, one possibility is to use
some sort of an external idea of checklist on how the dialogue should
go. In this way, the checklist can supervise the dialogue flow on the
going. Another possiblity is that we can use reinforcement learning to
model the reward and policy towards the final goal of motivational
interview: facilitating behaviour change.

%%% Local Variables:
%%% mode: latex
%%% TeX-master: "../../dissertation-main.ltx"
%%% End:
