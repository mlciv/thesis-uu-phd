\section{Analysis on Phrasal Anchoring in UCCA and TOP}
\label{sec:phr:anchor-analysis}

As a result, the semantic concepts in these meaning
representations can be anchored to the individual lexical units of the
sentence. \texttt{Flavor-1} includes Elementary Dependency
Structures~\cite[EDS,][]{oepen2006discriminant} and Universal
Conceptual Cognitive Annotation
framework~\cite[UCCA,][]{abend2013universal}, which shows an explicit,
many-to-many anchoring of semantic concepts onto sub-strings of the
underlying sentence.

However, different from the lexical anchoring without overlapping,
nodes in EDS and UCCA may align to larger overlapped word spans which
involves syntactic or semantic pharsal structure. Nodes in UCCA do not
have node labels or node properties, but all the nodes are anchored to
the spans of the underlying sentence. Furthermore, the nodes in UCCA
are linked into a hierarchical structure, with edges going between
parent and child nodes. With certain exceptions~(e.g. remote edges),
the majority of the UCCA graphs are tree-like structures. According to
the position as well as the anchoring style, nodes in UCCA can be
classified into the following two types:

\begin{inparaenum}
\item \textbf{Terminal nodes} are the leaf semantic
  concepts anchored to individual lexical units in the sentence

\item \textbf{Non-terminal nodes} are usually anchored to a span with
  more than one lexical units, thus usually overlapped with the
  anchoring of terminal nodes.
\end{inparaenum}

The similar classification of anchoring nodes also applies to the
nodes in EDS, although they do not regularly form a recursive tree
like UCCA. As the running example in Figure
\ref{fig:phrasal-anchoring}, most of the nodes belongs to terminal
nodes, which can be explicitly anchored to a single token in the
original sentence. However, those bold non-teriminal nodes are
anchored to a large span of words. For example, the node
\texttt{\char`\"undef\_q\char`\"} with span \texttt{<53:100>} is
aligned to the whole substring starting from ``other crops" to the
end; The abstract node with label \texttt{imp\_conj} are corresponding
to the whole coordinate structure between \texttt{soybeans and rice}

In summary, by treating AMR as an implicitly lexically anchored MR, we
propose a simplified taxonomy for parsing the five meaning
representation in this shared task.

\begin{itemize}
\item \texttt{Lexical-anchoring}: DM, PSD, AMR
\item \texttt{Phrasal-anchoring}: EDS, UCCA
\end{itemize}

%%% Local Variables:
%%% mode: latex
%%% TeX-master: "main"
%%% End:
