\section{Supplemental Material}
\label{sc:supplemental}

\subsection{Experiment Setups}
\label{ssec:exp}

\begin{table}[!h]
\begin{center}{\small
\setlength{\tabcolsep}{3pt}
\begin{tabular}{ll}
\toprule
\textbf{Glove.840B.300d}     & \\
dim       & 300       \\ \hline
\textbf{POS}       & \\
dim       &             \\ \hline
\textbf{CharCNN} & \\
dim & 22.6       \\
filters & 28.3       \\
53.5       & 15.0       \\ \bottomrule
\end{tabular}}
\end{center}
\caption{\label{tbl:model_setup} Model setup, to fix}
\end{table}


\subsection{Evaluation Metrics}
\label{ssec:eval}

\begin{itemize}
 \item \textbf{MCES} \\
MCES(Maximum Common Edge Subgraph Isomorphism) is a universal metric proposed for the shared MRP representation for all five MRs, which targets for the following goals for all 5 meaming representations:
\begin{inparaenum}[a)]
\item compatable with different flavors of semantic graphs
\item supporting labeled and unlabeled variants
\item supporting comparasion with or without corresponding node anchoring
\item minimize concerns about non-deterministic approximations
\item take advantage of anchoring information when available.
\end{inparaenum}
\item \textbf{Frame-specific metric} \\
\begin{itemize}
\item EDS: EDM (Elementary Dependency Match)
\item DM, PSD: SDP Labeled and Unlabeled Dependency F1
\item AMR:  SMATCH Precision, Recall, and F1
\item UCCA: UCCA Labeled and Unlabeled Dependency F1
\end{itemize}
\end{itemize}




%%% Local Variables:
%%% mode: latex
%%% TeX-master: "main"
%%% End:
