\section{Symbolic Representations for Natural Language}
\label{sec:bg:symbolic}

This is a chapter to indtroduce the background of meaning representations and related works on each of them.

\subsection{Broad-coverage Semantic Representation}
\label{ssec:bg:broad-mr}

For linguistic analysis, structures have been studied from
subword-level morphology~\cite{beesley2003finite}, word-level lexicon
semantics~\cite{miller1998wordnet}, to single-sentence syntax/semantic
representations~\cite{baker1998berkeley,palmer2005proposition,collins2003head},
and multi-sentences discourse
analysis~\cite{carlson2003building,wolf2005representing,prasad2008penn}. This
thesis covers broad-coverage graph-based semantic representations in
different anchoring types. For lexical anchoring, we cover the
DELPH-IN MRS Bi-lexical Dependencies~\cite[DM,][]{ivanova2012did} and
Prague Semantic
Dependencies~\cite[PSD,][]{hajic2012announcing,miyao2014house},
Abstract Meaning Representation~\cite[AMR,][]{Ban:Bon:Cai:13}; While
for phrasal-anchoring, we study Universal Conceptual Cognitive
Annotation~\cite[UCCA,][]{Abe:Rap:13b}.

\subsubsection{Bi-lexical Semantic Dependency Parsing}

\subsubsection{Abstract Meaning Representation}

\subsubsection{Universal Conceptual Cognitive Anotation}

\subsection{Application-specific Representation: Dialog}

In this thesis, we ground the study on application-specific
representation on dialogue.  In utterance-level, dialogue acts are
designed to represent the function of each utterance in the
dialogue. Similarly, Motivational Interview Skill Codes
~\cite[MISC,][]{miller2003motivational,miller2012motivational} are
also proposed to represent the functions of each client and therapist
utterance in the psychotherapy dialogue. Frame-based representation
for dialogue is popular in modern dialogue systems, where the state of
the dialogue is represented as the combinations of intent and
slots. Recently, conversational parsing has attract much attention to
represent the dialogue state in a more compositional way, such as the
hierarchical tree structure in the Task-Oriented dialogue Parsing
~\cite[TOP,][]{gupta-etal-2018-semantic-parsing} and so on.

\subsubsection{Dialog Act}

\subsubsection{Dialog State Tracking}

\subsubsection{Conversational Semantic Representation}

%%% Local Variables:
%%% mode: latex
%%% TeX-master: "../thesis-main.ltx"
%%% End:
