%%% -*-LaTeX-*-
\chapter{Supplementary Training and Description Styles}
\label{chap:appendix:schema-analysis}
This appendix shows more analysis and results on how to model the
natural language description in schema-guided dialogue, including
supplimentary
training~\autoref{sec:sgd:appendices-results-sup-training},
homogeneous and heterogeneous evaluation on different description
styles~\autoref{sec:sgd:more-desc-results}.

\section{More Results of Supplementary Training}
\label{sec:sgd:appendices-results-sup-training}
Table \ref{tbl:sup-training-details} shows the detailed performance
when using different intermediate tasks as supplementary training.  For
SNLI tasks, as the pretrained model is uncased model~(textattack/
bert-base-uncased-snli), hence, we first train different models with
BERT-base-uncased, then compare the performance with SNLI pretrained
model. For SQuAD2, we use deepset/bert-base-cased-SQuAD2 model, hence,
we compare it all cased model. To fairly compare with our original
\CE, we add extra speaker tokens \emph{[user:]} and \emph{[system:]} for encoding
the multiturn dialog histories.

To evaluate show supplementary training impact on different styles, we
also show the detailed results when we apply supplementary training of
SQuAD2 on the \NSL~tasks~(\autoref{ssec:sgd:squad2_homo}).

\begin{sidewaystable}[!tbp]
\caption{\label{tbl:sup-training-details} Results of different supplementary training on \sgdst~and \multiwoz~datasets.}
\begin{center}
\setlength{\tabcolsep}{2pt}
\begin{tabular}{c|c|ccc|ccc|ccc|ccc|ccc|ccc}
\toprule
  \hline
 &  & \multicolumn{12}{c|}{ sgd }   & \multicolumn{6}{c}{ multiwoz }                                                                                                                      \\ \cline{3-20}
 &  & \multicolumn{3}{c|}{ intent } & \multicolumn{3}{c|}{ req } & \multicolumn{3}{c|}{ cat } & \multicolumn{3}{c|}{ noncat } & \multicolumn{3}{c|}{ cat } & \multicolumn{3}{c}{ noncat } \\ \hline

\multirow{2}{*}{ snli }  & uncased              & 93.31       & 95.4  & 92.62       & 98.62 & 99.34 & 98.37 & 75.66 & 93.39 & 69.98 & 80.38      & 90.93 & 76.87      & 51.77 & 84.93 & 59.40 & 56.47      & 82.39 & 61.62       \\
                         & snli                 & 93.82       & 95.42 & 93.3        & 98.43 & 99.72 & 97.99 & 74.03 & 90.52 & 68.75 & 75.68      & 90.83 & 70.62      & 53.82 & 85.53 & 58.70 & 60.11      & 83.44 & 66.46       \\
                         & $\Delta_{\textbf{SNLI}}$  & {\bf +0.51} & +0.02 & {\bf +0.68} & -0.19 & +0.38 & -0.38 & -1.63 & -2.87 & -1.23 & -4.7       & -0.1  & -6.25      & +2.05 & +0.6  & --0.7  & {\bf 3.64} & +1.05 & {\bf +4.84} \\ \hline
\multirow{2}{*}{ squad } & cased                & 93.01       & 95.51 & 92.2        & 98.59 & 99.59 & 98.26 & 74.51 & 92.1  & 71.23 & 75.76      & 93.52 & 69.84      & 52.19 & 85.74 & 56.49 & 57.2       & 83.67 & 61.39       \\
                         & squad                & 91.2        & 95.34 & 90.88       & 98.34 & 99.58 & 97.93 & 71.64 & 89.08 & 66.06 & 77.75      & 91.73 & 73.09      & 52.23 & 85.03 & 56.90 & 59.13      & 81.46 & 65.66       \\
                         & $\Delta_{\textbf{SQuAD}}$ & -1.81       & -0.17 & -1.32       & -0.25 & -0.01 & -0.33 & -2.87 & -3.02 & -5.17 & {\bf 1.99} & -1.79 & {\bf 3.25} & +0.04 & -0.71 & +0.41 & {\bf 1.93} & -2.21 & {\bf +4.27} \\ \hline
\bottomrule
\end{tabular}
\end{center}
\end{sidewaystable}

\subsection[More details on SQuAD2 Results on Different Styles]{More details on SQuAD2 Results on \\Different Styles}
\label{ssec:sgd:squad2_homo}
For homogeneous evaluation,
\autoref{tbl:squad2-results-question-details} shows the detailed
performance when we apply SQuAD2-finetuned BERT on our \NSL~models.
We found that supplementary training generally helps the \NSL task in
most setting except for \NAMEONLY on \sgdst dataset. Furthermore, the
SQuAD2 supplementary training helps more on the rich description than
name-based description, because the questions in SQuAD2 share more
distribution similary with natural language descriptions than the
intent or slot names. Besides that, another observation is that
supplementary training helps more on unseen services, while slightly
hurting the performance on the seen services.

\begin{table}[!tbp]
\caption{\label{tbl:squad2-results-question-details} Results on
  different description style on \sgdst~and \multiwoz dataset, when
  performing SQuAD2 supplementary training.}
\begin{center}
\setlength{\tabcolsep}{3pt}
\begin{tabular}{c|ccc|ccc}
  \toprule
  \hline
                         \multirow{2}{*}{Style/Dataset} & \multicolumn{3}{c|}{\sgdst} & \multicolumn{3}{c}{\multiwoz}                     \\ \cline{2-7}
                                                        & all                         & seen  & unseen      & all   & seen  & unseen      \\ \hline
\multirow{3}{*}{\ORIGIN}                                & 75.76                       & 93.52 & 69.84       & 57.2  & 83.67 & 61.39       \\
                                                        & 77.75                       & 91.73 & 73.09       & 59.13 & 81.46 & 65.66       \\
                                                        & +1.99                       & -1.79 & {\bf +3.25} & +1.93 & -2.21 & {\bf +4.27} \\ \hline
\multirow{3}{*}{\QARICH}                                & 76.60                       & 92.86 & 71.18       & 57.80 & 82.45 & 62.45       \\
                                                        & 82.73                       & 90.85 & 80.02       & 58.86 & 81.17 & 65.51       \\
                                                        & +6.13                       & -2.01 & {\bf +8.84} & +1.06 & -1.28 & {\bf +3.06} \\ \hline
\multirow{3}{*}{\NAMEONLY}                              & 75.63                       & 88.90 & 71.21       & 56.18 & 81.68 & 61.30       \\
                                                        & 75.18                       & 87.41 & 71.10       & 57.93 & 82.26 & 63.07       \\
                                                        & -0.45                       & -1.49 & {\bf --0.11} & +1.75 & +0.58 & {\bf +1.77} \\ \hline
\multirow{3}{*}{\QANAMEONLY}                            & 74.86                       & 91.78 & 69.22       & 56.17 & 81.19 & 60.47       \\
                                                        & 74.91                       & 88.8  & 70.26       & 56.13 & 80.87 & 61.72       \\
                                                        & +0.05                       & -2.98 & {\bf +1.04} & -0.04 & -0.32 & {\bf +1.25} \\ \hline
  \bottomrule
\end{tabular}
\end{center}
\end{table}

\section[Homogeneous and Heterogeneous Evaluation on Different Styles]{Homogeneous and Heterogeneous \\Evaluation on Different Styles}
\label{sec:sgd:more-desc-results}
\autoref{ssec:desc-styles} introduces the benchmarking description
styles used in our dissertation. This appendix section shows more
detailed evaluation results on both homogeneous and heterogeneous
setting.


\subsection[More Results On Homogeneous and Heterogeneous Evalution]{More Results On Homogeneous and \\Heterogeneous Evalution}
\label{ssec:sgd:more-hete-results}
We list the detailed results for our evaluation across different
styles. We use {\it italic} to show the homogeneous evaluation, where
the results are shown in the diagonal of each table, and we
\underline{underline} the best homogeneous results in the diagonal. We
use {\bf bold} to show the best heterogeneous performance and the best
performance gap in the last two columns.

\subsubsection{Results on Intent Classification}
\label{ssec:sgd:results-ic}
The results on \sgdst dataset are shown in Table
\ref{tbl:heter-intent}. Because there are very few intents in
\multiwoz~dataset, we don't conduct intent classification on
\multiwoz. All performance get dropped when evaluating on
heterogeneous descriptions styles. For both heterogeneous and
homogeneous evaluation, adding rich description on intent
classification tasks seems not bring much benefits than simply using
the named-based description. As the discussion
in~\autoref{ssec:sgd:homo-eval}, we believe the name template is good
enough to describe the core functionality of an intent in \sgdst
dataset.

\begin{table}[!ht]
  \caption{\label{tbl:heter-intent} Accuracy of intent classification
    subtask with different description styles on unseen services. We
    train the models on \sgdst~dataset for each description in each
    row, then evaluating on 4 different descriptions styles. The mean
    are average performance of the remaining 3 descriptions
    styles. The $\Delta$ means the performance gap between the mean and the
    homogeneous performance.}
\begin{center}
\setlength{\tabcolsep}{2pt}
\begin{tabular}{c|cccc|cc}
  \toprule
  \hline
  Style       & \NAMEONLY               & \QANAMEONLY & \ORIGIN     & \QARICH     & mean        & $\Delta$        \\ \hline
  \NAMEONLY   & \underline{{\it 93.94}} & 78.27       & 93.18       & 75.95       & 82.47       & -11.47     \\
  \QANAMEONLY & 93.18                   & {\it 92.69} & 93.26       & 93.36       & {\bf 93.27} & {\bf +0.58} \\
  \ORIGIN     & 81.57                   & 66.42       & {\it 92.17} & 90.43       & 79.47       & -12.70     \\
  \QARICH     & 81.48                   & 79.04       & 93.19       & {\it 92.81} & 84.57       & -8.24      \\
  \hline
  \bottomrule
\end{tabular}
\end{center}
\end{table}

\subsubsection{Requested Slot}
\label{sssec:sgd:results-req}

Table \ref{tbl:heter-req}~shows the results on \sgdst~dataset for the
requested slots subtask. We ignore the requested slots in
\multiwoz~dataset due to its sparsity. Overall, the requested slot
subtask are relatively easy, performances on heterogeneous styles
still drops but not much. For both heterogeneous and homogeneous
evaluation, the performance are not sensible to rich description.

\begin{table}[!ht]
\caption{\label{tbl:heter-req} F1 Score of requested slot
  classification subtask with different description styles on unseen
  services. We train the model on \sgdst~dataset for the description
  style in each row, then evaluate on 4 different descriptions
  styles. The mean are average performance of the remaining 3
  descriptions styles. The $\Delta$ means the performance gap between the
  mean and the homogeneous performance.}
\begin{center}
\setlength{\tabcolsep}{2pt}
\begin{tabular}{c|cccc|cc}
 \toprule
  \hline
Style       & \NAMEONLY   & \QANAMEONLY             & \ORIGIN     & \QARICH     & mean        & $\Delta$         \\ \hline
\NAMEONLY   & {\it 98.56} & 96.01                   & 97.2        & 97.54       & 96.92       & -1.64       \\
\QANAMEONLY & 98.37       & \underline{{\it 98.64}} & 97.8        & 97.48       & {\bf 97.88} & -0.76       \\
\ORIGIN     & 97.95       & 95.78                   & {\it 98.16} & 98.52       & 97.42       & {\bf -0.74} \\
\QARICH     & 97.24       & 95.85                   & 97.00       & {\it 98.15} & 96.70       & -1.45       \\
  \hline
  \bottomrule
\end{tabular}
\end{center}
\end{table}

\subsubsection{Categorical Slot}
\label{sssec:sgd:results-cat}
The results on \sgdst ~and \multiwoz~dataset are shown in Table
\ref{tbl:heter-cat}. When creating
\multiwoz~\cite{zang-etal-2020-multiwoz}, the slots with less than 50
different slot values are classified as categorical slots. We noticed
that this leads inconsistent results with \sgdst~dataset. It is hard
to draw a consistent conclusion on the two datasets. According to the
definition, we believe \sgdst~are more suitable for categorical slot
subtasks, we can further verify our guess when more datasets are
created for the research of schema-guided dialog in the future.
\begin{table}[!ht]
  \caption{\label{tbl:heter-cat} Joint accuracy of categorical slot
    subtask with different description styles on unseen services. We
    train the different models on \sgdst ~and~\multiwoz~datasets,
    respectively, for each description style in each row. Then we
    evaluate on all 4 descriptions styles. The mean are the average
    performance of the remaining 3 descriptions styles. The $\Delta$ means
    the performance gap between the mean and the homogeneous
    performance.}
\begin{center}
\setlength{\tabcolsep}{2pt}
\begin{tabular}{c|cccc|cc}
 \toprule
  \hline
Style       & \NAMEONLY   & \QANAMEONLY             & \ORIGIN     & \QARICH                 & mean        & $\Delta$        \\ \hline
\multicolumn{7}{c}{\sgdst}                                                                                             \\ \hline
\NAMEONLY   & {\it 68.09} & 58.41                   & 63.49       & 62.21                   & 61.37       & -6.72      \\
\QANAMEONLY & 69.01       & {\it 68.29}             & 68.53       & 68.12                   & 68.55       & {\bf +0.26} \\
\ORIGIN     & 70.19       & 65.91                   & {\it 68.88} & 69.64                   & {\bf 68.58} & -0.30      \\
\QARICH     & 69.98       & 65.97                   & 69.26       & \underline{{\it 71.29}} & 68.40       & -2.89      \\
  \hline
\multicolumn{7}{c}{\multiwoz}                                                                                          \\ \hline
\NAMEONLY   & {\it 59.24} & 59.32                   & 59.12       & 59.29                   & 59.24       & 0.00       \\
\QANAMEONLY & 58.64       & \underline{{\it 59.74}} & 58.49       & 59.43                   & 58.85       & -0.89      \\
\ORIGIN     & 59.26       & 59.91                   & {\it 56.49} & 58.97                   & {\bf 59.38} & {\bf +2.89} \\
\QARICH     & 60.00       & 60.70                   & 51.18       & {\it 58.95}             & 57.29       & -1.66      \\
  \hline
  \bottomrule
\end{tabular}
\end{center}
\end{table}

\subsubsection{Noncategorical Slot}
\label{sssec:sgd:results-noncat}
We conduct noncategorical slot identification subtasks on both
\sgdst~and \multiwoz~dataset. The results are shown in Table
\ref{tbl:heternoncat}. Overall, the rich description performs better on
both homogeneous and heterogeneous evaluations.
\begin{table}[!ht]
  \caption{\label{tbl:heternoncat} Joint accuracy of noncategorical
    slot subtask with different description styles on unseen
    services. For each description style in each row, we train
    different models on \sgdst ~and~\multiwoz~datasets,
    respectively. Then we evaluate each model trained from one style
    on all 4 different description styles. The mean are the average
    performance of the remaining 3 description styles. The $\Delta$ means
    the performance gap between the mean and the homogeneous
    performance.}
\begin{center}
\setlength{\tabcolsep}{2pt}
\begin{tabular}{c|cccc|cc}
 \toprule
  \hline
Style       & \NAMEONLY               & \QANAMEONLY & \ORIGIN     & \QARICH                 & mean        & $\Delta$         \\ \hline
\multicolumn{7}{c}{\sgdst}                                                                                              \\ \hline
\NAMEONLY   & \underline{{\it 71.21}} & 49.85       & 59.8        & 59.95                   & 56.53       & -14.68      \\
\QANAMEONLY & 66.32                   & {\it 69.22} & 61.67       & 60.77                   & 62.92       & -6.30       \\
\ORIGIN     & 78.73                   & 51.57       & {\it 69.84} & 69.87                   & {\bf 66.72} & {\bf -3.12} \\
\QARICH     & 62.6                    & 36.44       & 69.49       & \underline{{\it 71.18}} & 56.18       & -15.00      \\ \hline
\multicolumn{7}{c}{\multiwoz}                                                                                           \\ \hline
\NAMEONLY   & {\it 61.30}             & 57.88       & 61.51       & 64.05                   & 61.15       & -0.15       \\
\QANAMEONLY & 60.62                   & {\it 60.47} & 60.6        & 62.58                   & 61.27       & +0.80        \\
\ORIGIN     & 61.77                   & 65.4        & {\it 61.39} & 62.4                    & {\bf 63.19} & {\bf +1.80}  \\
\QARICH     & 61.29                   & 60.6        & 62.46       & \underline{{\it 62.45}} & 61.45       & -1.00       \\
  \hline
  \bottomrule
\end{tabular}
\end{center}
\end{table}

\begin{sidewaystable}[!t]
\caption{\label{tbl:heter-quali}We analyze the confusion matrix of
  above slots before and after using the paraphrased name. We summarize
  the extra impact for using each paraphrased name.}
\begin{center}
\setlength{\tabcolsep}{3pt}
\begin{tabular}{ccc|p{7cm}}
  \bottomrule
  \hline
Service Name    & Original Name             & Paraphrased Name                      & Extra Impact by Paraphrased Name                                                    \\ \hline
\kw{Travel\_1}  & \kw{location}             & \kw{attraction\_location}             & Confused with other \kw{attraction} prefixed slots, \eg, \kw{attraction \_name}         \\
\kw{Movies\_1}  & \kw{genre}                & \kw{movie\_genre}                     & Confused with \kw{movie\_name}                                                          \\
\kw{Movies\_1}  & \kw{price}                & \kw{ticket\_price}, \kw{total\_price} & No impact                                                                               \\ \hline
\kw{Bueses\_3}  & \kw{to\_city}             & \kw{target\_city}                     & The synonyms \kw{target} is not understood well by model, confused with \kw{from\_city} \\
\kw{Movies\_1}  & \kw{movie\_name}          & \kw{film\_name}                       & The synonyms \kw{film} is not understood well, getting wrong with \kw{theather\_name}   \\
\kw{Hotels\_2}  & \kw{where\_to}            & \kw{house\_loc}                       & Improved by specific "house" keywords                                                   \\ \hline
\kw{Flights\_4} & \kw{origin\_airport}      & \kw{orig\_city\_airport}              & More frequently predicted to slot \kw{destination\_airport}                             \\
\kw{Flights\_4} & \kw{destination\_airport} & \kw{dest\_city\_airport}              & More frequently predicted to slot  \kw{origin\_airport}                                 \\
\kw{Media\_3}   & \kw{subtitle\_language}   & \kw{sub\_lang}                        & Missing keyword \kw{subtitle} make the slot inactive                                    \\
\kw{Flights\_4} & \kw{number\_of\_tickets}  & \kw{num\_of\_tickets}            & No impact                                                                       \\ \hline
\bottomrule
\end{tabular}
\end{center}
\end{sidewaystable}


\subsection[Qualitative Analysis On Heterogeneous
Evaluation]{Qualitative Analysis On Heterogeneous \\Evaluation}
\label{ssec:qualitative-analysis}
We conduct qualitative analysis on heterogeneous evaluation on
name-based descriptions. Table \ref{tbl:heter-quali} shows how
paraphrasing the name-based descriptions impacts on the categorical and
noncategorical slot prediction tasks.

The first three rows at the top are showing the cases of adding modifiers
to the name. When the added extra modifiers are keywords in other
slots, \eg,~\kw{attraction} are the keywords also used
in~\kw{attraction\_name.}~The first
shows~\kw{attraction\_location} may wrongly predicted
as~\kw{attraction\_name.}~It seems the model does not understand
the compound nouns well, and they seems just pay attention to each key
words~\kw{attraction} and \kw{movie} here.

The three rows in the middle are showing the cases of using
synonyms. Changing \kw{to} to \kw{target,}~and changing \kw{movie} to \kw{film} will
cause extra confusion, which shows the model may fail to the synonyms.

The last four rows at the bottom is showing using abbreviations. Changing
\kw{number} to \kw{num} will not impact the model, while
changing \kw{subtitle} to \kw{sub} may let the model miss
the key meaning of subtitle.  The performance drop in the later case
may be due to the misuse of the \kw{sub} prefix, in English, it
usually means \kw{secondary, less important, parts.}~We also
found the \kw{orig} and \kw{dest} abbreviations may also understand well by
the model. The above abbreviations seems reasonable paraphrases people
will use for naming, while the are not understood well in the given
context. Hence, in the design of schema-guided dialog, if using
named-based description, we should be careful for about abbreviations
used in the naming.

\begin{table}[!tbp]
\caption{\label{tbl:hete-style-results-sgd2} Results on unseen service with heterogeneous description styles on \sgdst dataset. More results and qualitative analysis are in the appendix~\ref{sec:sgd:more-desc-results}}
\begin{center}
\setlength{\tabcolsep}{1pt}
\begin{tabular}{c|c|c|c|c}
  \toprule
  \hline
\multirow{3}{*}{Style\textbackslash{Task}} & \multicolumn{4}{c}{ \sgdst }                                                                                                                   \\ \cline{2-5}
                                           & \multicolumn{1}{c|}{ Intent(Acc)} & \multicolumn{1}{c|}{Req(F1)} & \multicolumn{1}{c|}{Cat(Joint Acc)} & \multicolumn{1}{c}{NonCat(Joint Acc)} \\ \cline{2-5}
                                           & $\Delta$                               & $\Delta$                          & $\Delta$                                 & $\Delta$                                   \\ \hline
\NAMEONLY                                  & -11.47                            & -1.64                        & -5.54                               & -14.68                                \\
\QANAMEONLY                                & +0.58                             & -0.76                        & +2.63                               & -6.30                                 \\
\ORIGIN                                    & -12.70                            & -0.74                        & -0.3                                & -3.11                                 \\
\QARICH                                    & -8.24                             & -1.45                        & -2.89                               & -15.00                                \\
  \hline
                                           & $\Delta$                               & $\Delta$                          & $\Delta$                                 & $\Delta$                                   \\ \hline
\NAMEONLY                                  & -1.74                             & -0.87                        & -0.7                                & -4.04                                 \\
\ORIGIN                                    & {\bf -0.63}                       & {\bf +0.26}                  & {\bf +2.86}                         & {\bf -2.16}                           \\ \hline
  \bottomrule
\end{tabular}
\end{center}
\end{table}


%%% Local Variables:
%%% mode: latex
%%% TeX-master: "../dissertation-main.ltx"
%%% End:
