\chapter{More Analysis on Schema-Guided Dialogue State Tracking}
\label{chap:appendix:schema}
%Appendices are material that can be read, and include lemmas, formulas, proofs, and tables that are not critical to the reading and understanding of the paper. 
%Appendices should be \textbf{uploaded as supplementary material} when submitting the paper for review.
%Upon acceptance, the appendices come after the references, as shown here.
\section{Composition of Descriptions}
\label{sec:sgd:com-desc}
For each subtask, the key description element must be included, e.g.,
intent description for intent task, and value for categorical slot
tasks. However, besides the description about intents and slots, we
also have service description and the names for intents and
slots. What other information will help? and what kind of composition
will offer better performance. In this section, we will mainly answer
the questions with the following experiments on compostion of
descriptions.

\subsection{Composition Settings}
\label{ssec:sgd:com-settings}
 To show how each component helps schema-guided dialog state
tracking, we incrementally add richer schema component one by one.

\Paragraph{\bf ID} This is the least informative case: we only use
  meaningless intent/slot identifiers, e.g. Intent\_4, Slot\_2. It means we don't use
  description from any schema component. We want to investigate how a
  simple identifier-based description performs in schema-guided dialog
  modeling, and the performance lower-bound on transferring to unseen services.

\Paragraph{\bf I/S Desc} Only using the original intent/slot description
  of intent/slot in \sgdst and \multiwoz dataset for corresponding
  tasks.

\Paragraph{\bf Service + I/S Desc} Adding a service description to the above
  original description. Service description summarize the
  functionalities of the whole service, hence may offer extra background
  information for intent and slots.
For categorical slot value detection, we simply add the
value after each of the above composition.

\begin{table}[!t]
\begin{center}{\scriptsize
\setlength{\tabcolsep}{2pt}
\begin{tabular}{l|cccc|cc}
\toprule
\hline
Model\textbackslash{Task} & \multicolumn{4}{c|}{SG-DST} & \multicolumn{2}{c}{MultiWOZ}                                        \\
                          & Intent                      & Req         & Cat         & NonCat      & Cat         & NonCat      \\ \hline
  \multicolumn{7}{c}{{\bf Seen Service}}                                                                                      \\ \hline
Identifer                 & 92.76                       & 99.70       & 87.86       & 88.38       & 58.46       & 77.29       \\
I/S Desc                  & {\bf 95.35}                 & {\bf 99.74} & 92.10       & {\bf 93.52} & {\bf 85.84} & {\bf 83.67} \\
Service + I/S Desc        & 95.28                       & {\bf 99.74} & {\bf 93.19} & 92.34       & 85.07       & 80.56       \\ \hline
  \multicolumn{7}{c}{{\bf Unseen Service}}                                                                                    \\ \hline
Identifier                & 50.63                       & 88.74       & 54.34       & 10.77       & 53.05       & 56.18       \\
I/S Desc                  & {\bf 92.17}                 & {\bf 98.16} & {\bf 68.88} & 69.84       & 56.49       & {\bf 61.39} \\
Service + I/S Desc        & 86.95                       & 97.99       & 67.08       & {\bf 71.30} & {\bf 60.58} & 59.63       \\
\hline
\bottomrule
\end{tabular}}
\end{center}
\caption{\label{tbl:schema-component-results} Models using different
  composition of schema, results on test set of \sgdst and our remixed \multiwoz %For zero-shot performance on MultiWOZ, we simply merge the zero-shot performance on each domain.
  }
\end{table}

\begin{table}[!t]
\begin{center}{\scriptsize
\setlength{\tabcolsep}{2pt}
\begin{tabular}{c|c|ccc|ccc|ccc|ccc|ccc|ccc}
\toprule
  \hline
 &  & \multicolumn{12}{c|}{ sgd }   & \multicolumn{6}{c}{ multiwoz }                                                                                                                      \\ \cline{3-20}
 &  & \multicolumn{3}{c|}{ intent } & \multicolumn{3}{c|}{ req } & \multicolumn{3}{c|}{ cat } & \multicolumn{3}{c|}{ noncat } & \multicolumn{3}{c|}{ cat } & \multicolumn{3}{c}{ noncat } \\ \hline

\multirow{2}{*}{ snli }  & uncased              & 93.31       & 95.4  & 92.62       & 98.62 & 99.34 & 98.37 & 75.66 & 93.39 & 69.98 & 80.38      & 90.93 & 76.87      & 51.77 & 84.93 & 59.40 & 56.47      & 82.39 & 61.62       \\
                         & snli                 & 93.82       & 95.42 & 93.3        & 98.43 & 99.72 & 97.99 & 74.03 & 90.52 & 68.75 & 75.68      & 90.83 & 70.62      & 53.82 & 85.53 & 58.70 & 60.11      & 83.44 & 66.46       \\
                         & $\Delta_{\textbf{SNLI}}$  & {\bf +0.51} & +0.02 & {\bf +0.68} & -0.19 & +0.38 & -0.38 & -1.63 & -2.87 & -1.23 & -4.7       & -0.1  & -6.25      & +2.05 & +0.6  & --0.7  & {\bf 3.64} & +1.05 & {\bf +4.84} \\ \hline
\multirow{2}{*}{ squad } & cased                & 93.01       & 95.51 & 92.2        & 98.59 & 99.59 & 98.26 & 74.51 & 92.1  & 71.23 & 75.76      & 93.52 & 69.84      & 52.19 & 85.74 & 56.49 & 57.2       & 83.67 & 61.39       \\
                         & squad                & 91.2        & 95.34 & 90.88       & 98.34 & 99.58 & 97.93 & 71.64 & 89.08 & 66.06 & 77.75      & 91.73 & 73.09      & 52.23 & 85.03 & 56.90 & 59.13      & 81.46 & 65.66       \\
                         & $\Delta_{\textbf{SQuAD}}$ & -1.81       & -0.17 & -1.32       & -0.25 & -0.01 & -0.33 & -2.87 & -3.02 & -5.17 & {\bf 1.99} & -1.79 & {\bf 3.25} & +0.04 & -0.71 & +0.41 & {\bf 1.93} & -2.21 & {\bf +4.27} \\ \hline
\bottomrule
\end{tabular}
}
\end{center}
\caption{\label{tbl:sup-training-details} Results of different supplementary training on \sgdst~and \multiwoz~dataset}
\end{table}

\subsection{Results on Description Compositions}
\label{ssec:sgd:com-results}
Table~\ref{tbl:schema-component-results} shows the results of using
different description compositions. First, there are consistent
findings across datasets and subtasks:
\begin{inparaenum}[(1)]
 \item using meaningless identifier as intent/slot description shows
  the worse performance on all tasks of both datasets, and can
  not generalize well to unseen services.
 \item using intent/slot descriptions can largely boost the performance,
  especially on unseen services.
\end{inparaenum}

However, the impact of service description varies by tasks. For
example, it largely hurts performance on intent classification task,
but does not impact %or slightly overfitting on
requested slot and categorical slot tasks. According to manual
analysis of \sgdst~and~\multiwoz dataset, we found that service
description consists of the main functions of the service, especially
the meaning of the supported intents. Hence, using service description
for intent causes confusion between the intent description information
and other supported intents. Moreover, in categorical slot value
prediction task, the most important information is the slot
description and value.  When adding extra information from service
description, it improves marginally on seen service while not
generalizing well on unseen services, which indicates the model learns
artifacts that are not general useful for unseen services.

Finally, on non-categorical slot tasks, the impact of service
description may also varies on datasets. On \sgdst, there are 16
domains and more than 30 services, the rich background context from
service description contains both domain and service-specific
information, which seems to help both seen and unseen
services. However, on \multiwoz, it hurts the performance on seen
service {\it restaurant} the most, while improving the performance on
the unseen service {\it hotel} by 4 points. In this case, it works
like a regularizer rather than a definitive clues. Because in
\multiwoz, there are only 8 domains, and one service per domain, thus
service descriptions just contain domain related information without
much extra information, it will not help the model to detect the span
for the slot.
%
%\subsection{Statistics on Schema Name and Description}
%\label{ssec:appendices-schema-statistics}
%
%We analyze the structure pattern of the schema description in SG-DST and
%MultiWOZ datasets, and summerize them in Table \ref{tbl:desc_stat}
%
%\begin{table}[!h]
%\begin{center}{\small
%\setlength{\tabcolsep}{3pt}
%\begin{tabular}{l|cc|c}
%\hline
%\multicolumn{4}{c}{Desc in SG-DST}        \\ \hline
%        & Freq Patterns & Count & Example \\ \hline
%Service &               &       & \\
%Intent  &               &       & \\
%Slot    &               &       & \\ \hline
%\multicolumn{4}{c}{Desc in MultiWOZ}      \\ \hline
%Service &               &       & \\
%Intent  &               &       & \\
%Slot    &               &       & \\
%\hline
%\end{tabular}}
%\end{center}
%\caption{\label{tbl:desc_stat} Service/Intent/Slot names in SG-DST and MultiWOZ datasets}
%\end{table}

\section{More Results of Supplementary Training}
\label{sec:sgd:appendices-results-sup-training}
Table \ref{tbl:sup-training-details} shows the detailed performance
when using different intermediate tasks as supplementary training.  For
SNLI tasks, as the pretrained model is uncased model~(textattack/
bert-base-uncased-snli), hence, we first train different models with
BERT-base-uncased, then compare the performance with SNLI pretrained
model. For SQuAD2, we use deepset/bert-base-cased-SQuAD2 model, hence,
we compare it all cased model. To fairly compare with our original
\CE, we add extra speaker tokens [user:] and [system:] for encoding
the multi-turn dialog histories.

\section{Homogeneous and Hetergenuous Evaluation on Different Styles}
\label{sec:sgd:more-desc-results}

\subsection{More details on SQuAD2 Results on Different Styles}
\label{ssec:sgd:squad2_homo}
For homogeneous evaluation, Table
\ref{tbl:squad2-results-question-details} shows the detailed
performance when we apply SQuAD2-finetuned BERT on our models.
\begin{table}[t]
\begin{center}{\small
\setlength{\tabcolsep}{3pt}
\begin{tabular}{c|ccc|ccc}
  \toprule
  \hline
                         \multirow{2}{*}{Style/Dataset} & \multicolumn{3}{c|}{\sgdst} & \multicolumn{3}{c}{\multiwoz}                     \\ \cline{2-7}
                                                        & all                         & seen  & unseen      & all   & seen  & unseen      \\ \hline
\multirow{3}{*}{\ORIGIN}                                & 75.76                       & 93.52 & 69.84       & 57.2  & 83.67 & 61.39       \\
                                                        & 77.75                       & 91.73 & 73.09       & 59.13 & 81.46 & 65.66       \\
                                                        & +1.99                       & -1.79 & {\bf +3.25} & +1.93 & -2.21 & {\bf +4.27} \\ \hline
\multirow{3}{*}{\QARICH}                                & 76.60                       & 92.86 & 71.18       & 57.80 & 82.45 & 62.45       \\
                                                        & 82.73                       & 90.85 & 80.02       & 58.86 & 81.17 & 65.51       \\
                                                        & +6.13                       & -2.01 & {\bf +8.84} & +1.06 & -1.28 & {\bf +3.06} \\ \hline
\multirow{3}{*}{\NAMEONLY}                              & 75.63                       & 88.90 & 71.21       & 56.18 & 81.68 & 61.30       \\
                                                        & 75.18                       & 87.41 & 71.10       & 57.93 & 82.26 & 63.07       \\
                                                        & -0.45                       & -1.49 & {\bf --0.11} & +1.75 & +0.58 & {\bf +1.77} \\ \hline
\multirow{3}{*}{\QANAMEONLY}                            & 74.86                       & 91.78 & 69.22       & 56.17 & 81.19 & 60.47       \\
                                                        & 74.91                       & 88.8  & 70.26       & 56.13 & 80.87 & 61.72       \\
                                                        & +0.05                       & -2.98 & {\bf +1.04} & -0.04 & -0.32 & {\bf +1.25} \\ \hline
  \bottomrule
\end{tabular}
}
\end{center}
\caption{\label{tbl:squad2-results-question-details} Results on
  different description style on \sgdst~and \multiwoz dataset, when
  performing SQuAD2 supplementary training }
\end{table}


\subsection{More Results On Homogeneous and Heterogeneous Evalution}
\label{ssec:sgd:more-hete-results}
We list the detailed results for our evaluation across different
styles. We use {\it italic} to show the homogeneous evaluation, where
the results are shown in the diagonal of each table, and we
\underline{underline} the best homogeneous results in the diagonal. We
use {\bf bold} to show the best heterogeneous performance and the best
performance gap in the last two columns

\Paragraph{Intent} The results on \sgdst dataset are shown in Table
\ref{tbl:heter-intent}. Because there are very few intents in
\multiwoz~dataset, we don't conduct intent classification on
\multiwoz. All performance get dropped when evaluating on
heterogeneous descriptions styles. For both heterogeneous and
homogeneous evaluation, adding rich description on intent classification
tasks seems not bring much benefits than simply using the named-based
description. As the discussion in \S\ref{sssec:homo-eval}, we believe
the name template is good enough to describe the core functionality of
an intent in \sgdst dataset.

\begin{table}[!ht]
\begin{center}{\scriptsize
\setlength{\tabcolsep}{2pt}
\begin{tabular}{c|cccc|cc}
  \toprule
  \hline
  Style       & \NAMEONLY               & \QANAMEONLY & \ORIGIN     & \QARICH     & mean        & $\Delta$        \\ \hline
  \NAMEONLY   & \underline{{\it 93.94}} & 78.27       & 93.18       & 75.95       & 82.47       & -11.47     \\
  \QANAMEONLY & 93.18                   & {\it 92.69} & 93.26       & 93.36       & {\bf 93.27} & {\bf +0.58} \\
  \ORIGIN     & 81.57                   & 66.42       & {\it 92.17} & 90.43       & 79.47       & -12.70     \\
  \QARICH     & 81.48                   & 79.04       & 93.19       & {\it 92.81} & 84.57       & -8.24      \\
  \hline
  \bottomrule
\end{tabular}
}
\end{center}
\caption{\label{tbl:heter-intent} Accuracy of intent classification
  subtask with different description styles on unseen services. Train
  the model on \sgdst dataset for each description in each row, then
  evaluating on 4 different descriptions styles. The mean are average
  performance of the remaining 3 descriptions styles. The $\Delta$ means
  the performance gap between the mean and the homogeneous
  performance}
\end{table}

\Paragraph{Requested Slot} Table \ref{tbl:heter-req}~shows the results
on \sgdst~dataset for the requested slots subtask. We ignore the
requested slots in \multiwoz~dataset due to its sparsity. Overall, the
requested slot subtask are relatively easy, performances on
heterogeneous styles still drops but not much. For both heterogeneous
and homogeneous evaluation, the performance are not sensible to rich
description.

\begin{table}[!ht]
\begin{center}{\scriptsize
\setlength{\tabcolsep}{2pt}
\begin{tabular}{c|cccc|cc}
 \toprule
  \hline
Style       & \NAMEONLY   & \QANAMEONLY             & \ORIGIN     & \QARICH     & mean        & $\Delta$         \\ \hline
\NAMEONLY   & {\it 98.56} & 96.01                   & 97.2        & 97.54       & 96.92       & -1.64       \\
\QANAMEONLY & 98.37       & \underline{{\it 98.64}} & 97.8        & 97.48       & {\bf 97.88} & -0.76       \\
\ORIGIN     & 97.95       & 95.78                   & {\it 98.16} & 98.52       & 97.42       & {\bf -0.74} \\
\QARICH     & 97.24       & 95.85                   & 97.00       & {\it 98.15} & 96.70       & -1.45       \\
  \hline
  \bottomrule
\end{tabular}
}
\end{center}
\caption{\label{tbl:heter-req} F1 Score of requested slot
  classification subtask with different description styles on unseen
  services. We train the model on \sgdst~dataset for the description
  style in each row, then evaluate on 4 different descriptions
  styles. The mean are average performance of the remaining 3
  descriptions styles. The $\Delta$ means the performance gap between the
  mean and the homogeneous performance}
\end{table}

\Paragraph{Categorical Slot}
The results on \sgdst ~and \multiwoz~dataset are shown in Table
\ref{tbl:heter-cat}. When creating
\multiwoz~\cite{zang-etal-2020-multiwoz}, the slots with less than 50
different slot values are classified as categorical slots. We noticed
that this leads inconsistent results with \sgdst~dataset. It is hard
to draw a consistent conclusion on the two datasets. According to the
definition, we believe \sgdst~are more suitable for categorical slot
subtasks, we can further verify our guess when more datasets are
created for the research of schema-guided dialog in the future.
\begin{table}[!ht]
\begin{center}{\scriptsize
\setlength{\tabcolsep}{2pt}
\begin{tabular}{c|cccc|cc}
 \toprule
  \hline
Style       & \NAMEONLY   & \QANAMEONLY             & \ORIGIN     & \QARICH                 & mean        & $\Delta$        \\ \hline
\multicolumn{7}{c}{\sgdst}                                                                                             \\ \hline
\NAMEONLY   & {\it 68.09} & 58.41                   & 63.49       & 62.21                   & 61.37       & -6.72      \\
\QANAMEONLY & 69.01       & {\it 68.29}             & 68.53       & 68.12                   & 68.55       & {\bf +0.26} \\
\ORIGIN     & 70.19       & 65.91                   & {\it 68.88} & 69.64                   & {\bf 68.58} & -0.30      \\
\QARICH     & 69.98       & 65.97                   & 69.26       & \underline{{\it 71.29}} & 68.40       & -2.89      \\
  \hline
\multicolumn{7}{c}{\multiwoz}                                                                                          \\ \hline
\NAMEONLY   & {\it 59.24} & 59.32                   & 59.12       & 59.29                   & 59.24       & 0.00       \\
\QANAMEONLY & 58.64       & \underline{{\it 59.74}} & 58.49       & 59.43                   & 58.85       & -0.89      \\
\ORIGIN     & 59.26       & 59.91                   & {\it 56.49} & 58.97                   & {\bf 59.38} & {\bf +2.89} \\
\QARICH     & 60.00       & 60.70                   & 51.18       & {\it 58.95}             & 57.29       & -1.66      \\
  \hline
  \bottomrule
\end{tabular}
}
\end{center}
\caption{\label{tbl:heter-cat} Joint accuracy of categorical slot
  Subtask with different description styles on unseen services. Train
  the model on \sgdst ~and~\multiwoz~datasets respectively for each
  description style in each row, then evaluate on all 4 descriptions
  styles. The mean are the average performance of the remaining 3
  descriptions styles. The $\Delta$ means the performance gap between the
  mean and the homogeneous performance}
\end{table}

\Paragraph{Non-categorical Slot}
We conduct non-categorical slot identification sub-tasks on both
\sgdst~and \multiwoz~dataset. The results are shown in Table
\ref{tbl:heter-noncat}. Overall, the rich description performs better on
both homogeneous and heterogeneous evaluations.
\begin{table}[!ht]
\begin{center}{\scriptsize
\setlength{\tabcolsep}{2pt}
\begin{tabular}{c|cccc|cc}
 \toprule
  \hline
Style       & \NAMEONLY               & \QANAMEONLY & \ORIGIN     & \QARICH                 & mean        & $\Delta$         \\ \hline
\multicolumn{7}{c}{\sgdst}                                                                                              \\ \hline
\NAMEONLY   & \underline{{\it 71.21}} & 49.85       & 59.8        & 59.95                   & 56.53       & -14.68      \\
\QANAMEONLY & 66.32                   & {\it 69.22} & 61.67       & 60.77                   & 62.92       & -6.30       \\
\ORIGIN     & 78.73                   & 51.57       & {\it 69.84} & 69.87                   & {\bf 66.72} & {\bf -3.12} \\
\QARICH     & 62.6                    & 36.44       & 69.49       & \underline{{\it 71.18}} & 56.18       & -15.00      \\ \hline
\multicolumn{7}{c}{\multiwoz}                                                                                           \\ \hline
\NAMEONLY   & {\it 61.30}             & 57.88       & 61.51       & 64.05                   & 61.15       & -0.15       \\
\QANAMEONLY & 60.62                   & {\it 60.47} & 60.6        & 62.58                   & 61.27       & +0.80        \\
\ORIGIN     & 61.77                   & 65.4        & {\it 61.39} & 62.4                    & {\bf 63.19} & {\bf +1.80}  \\
\QARICH     & 61.29                   & 60.6        & 62.46       & \underline{{\it 62.45}} & 61.45       & -1.00       \\
  \hline
  \bottomrule
\end{tabular}
}
\end{center}
\caption{\label{tbl:heter-noncat} Joint accuracy of non-categorical
  slot Subtask with different description styles on unseen
  services. We train the model on \sgdst ~and~\multiwoz~datasets
  respectively for the description style in each row, then evaluate
  on all 4 different descriptions styles. The mean are the average
  performance of the remaining 3 descriptions styles. The $\Delta$ means
  the performance gap between the mean and the homogeneous
  performance}
\end{table}

\begin{table}[!t]
\begin{center}{\scriptsize
\setlength{\tabcolsep}{3pt}
\begin{tabular}{ccc|l}
  \bottomrule
  \hline
Service Name & Original Name        & Paraphrased Name            & Extra impact by the paraphrased name                                            \\ \hline
Travel\_1    & location             & attraction\_location        & Confused with other "attraction" prefixed slots, e.g. attraction \_name         \\
Movies\_1    & genre                & movie\_genre                & Confused with movie\_name                                                       \\
Movies\_1    & price                & ticket\_price, total\_price & No impact                                                                       \\ \hline
Bueses\_3    & to\_city             & target\_city                & The synonyms "target" is not understood well by model, confused with from\_city \\
Movies\_1    & movie\_name          & film\_name                  & The synonyms "film" is not understood well, getting wrong with theather\_name   \\
Hotels\_2    & where\_to            & house\_loc                  & Improved by specific "house" keywords                                           \\ \hline
Flights\_4   & origin\_airport      & orig\_city\_airport         & More frequently predicted to slot "destination\_airport"                             \\
Flights\_4   & destination\_airport & dest\_city\_airport         & More frequently predicted to slot  "origin\_airport"                                  \\
Media\_3     & subtitle\_language   & sub\_lang                   & Missing keyword "subtitle" make the slot inactive                               \\
Flights\_4   & number\_of\_tickets  & num\_of\_tickets            & No impact                                                                       \\ \hline
\bottomrule
\end{tabular}
}
\end{center}
\caption{\label{tbl:heter-quali}We analyze the confusion matrix of
  above slots before and after using the paraphrased name. We summarize
  the extra impact for using each paraphrased name.  }
\end{table}


\subsection{Qualitative Analysis On Heterogeneous Evaluation}
\label{ssec:qualitative-analysis}
We conduct qualitative analysis on heterogeneous evaluation on
named-based description. Table \ref{tbl:heter-quali} shows how paraphrasing the
named-based description impact on the categorical and non-categorical
slot prediction tasks.

The first 3 rows at the top are showing the cases of adding modifiers
to the name. When the added extra modifiers are keywords in other
slots, e.g. "attraction" are the keywords also used in
"attraction\_name". The first shows "attraction\_location" may wrongly
predicted as "attraction\_name". It seems the model does not understand
the compound nouns well, and they seems just pay attention to each key
words "attraction" and "movie" here.

The 3 rows in the middle are showing the cases of using
synonyms. Changing "to" to "target", and changing "movie" to "film" will
cause extra confusion, which shows the model may fail to the synonyms.

The last 4 rows at the bottom is showing using abbreviations. Changing
"number" to "num" will not impact the model, while changing "subtitle"
to "sub" may let the model miss the key meaning of subtitle.  The
performance drop in the later case may be due to the misuse of the
"sub" prefix, in English, it usually means "secondary, less important,
parts". We also found the "orig" and "dest" abbreviations may also
understand well by the model. The above abbreviations seems reasonable
paraphrases people will use for naming, while the are not understood
well in the given context. Hence, in the design of schema-guided
dialog, if using named-based description, we should be careful for
about abbreviations used in the naming.

\begin{table}[!t]
\begin{center}{\scriptsize
\setlength{\tabcolsep}{2pt}
\begin{tabular}{c|cc|cc}
  \toprule
  \hline
                       & \multicolumn{4}{c}{ \multiwoz }                                                                                                                                                                                                                      \\ \cline{2-5}
                       & \multicolumn{2}{c|}{ cat } & \multicolumn{2}{c}{ noncat } \\ \cline{2-4}
                       & seen                       & unseen & seen  & unseen      \\ \hline
  $\Delta_{\textbf{SNLI}}$  & +0.6                       & --0.7   & +1.05 & {\bf +4.84} \\ \hline
  $\Delta_{\textbf{SQuAD}}$ & -0.71                      & +0.41  & -2.21 & {\bf +4.27} \\ \hline
  \bottomrule
\end{tabular}
}
\end{center}
\caption{\label{tbl:sup-training3} Relative performance improvement of different supplementary training on \sgdst~and \multiwoz~dataset}
\end{table}


\begin{table}[!t]
\begin{center}{\small
\setlength{\tabcolsep}{3pt}
\begin{tabular}{c|cc|cc}
  \toprule
  \hline
                         \multirow{2}{*}{Style/Dataset} & \multicolumn{2}{c}{\sgdst}  & \multicolumn{2}{c}{\multiwoz}  \\ \cline{2-5}
                                                        &  seen  & unseen      & seen  & unseen      \\ \hline
 \multirow{1}{*}{\ORIGIN}                               &  -1.79 & {\bf +3.25} & -2.21 & {\bf +4.27} \\ \
 \multirow{1}{*}{\QARICH}                               &  -2.01 & {\bf +8.84} & -1.28 & {\bf +3.06} \\ \hline
 \multirow{1}{*}{\NAMEONLY}                             &  -1.49 & -0.11       & +0.58 & {\bf +1.77}       \\
 \multirow{1}{*}{\QANAMEONLY}                           &  -2.98 & {\bf +1.04} & -0.32 & {\bf +1.25} \\ \hline
  \bottomrule
\end{tabular}
}
\end{center}
\caption{\label{tbl:squad2-results-question2} Performance changes when
  using BERT finetuned on SQuAD2 dataset to further finetuning on our
  \NSL~task. }
\end{table}


\begin{table}[!ht]
\begin{center}{\scriptsize
\setlength{\tabcolsep}{1pt}
\begin{tabular}{c|c|c|c|c}
  \toprule
  \hline
\multirow{3}{*}{Style\textbackslash{Task}} & \multicolumn{4}{c}{ \sgdst }                                                                                                                   \\ \cline{2-5}
                                           & \multicolumn{1}{c|}{ Intent(Acc)} & \multicolumn{1}{c|}{Req(F1)} & \multicolumn{1}{c|}{Cat(Joint Acc)} & \multicolumn{1}{c}{NonCat(Joint Acc)} \\ \cline{2-5}
                                           & $\Delta$                               & $\Delta$                          & $\Delta$                                 & $\Delta$                                   \\ \hline
\NAMEONLY                                  & -11.47                            & -1.64                        & -5.54                               & -14.68                                \\
\QANAMEONLY                                & +0.58                             & -0.76                        & +2.63                               & -6.30                                 \\
\ORIGIN                                    & -12.70                            & -0.74                        & -0.3                                & -3.11                                 \\
\QARICH                                    & -8.24                             & -1.45                        & -2.89                               & -15.00                                \\
  \hline
                                           & $\Delta$                               & $\Delta$                          & $\Delta$                                 & $\Delta$                                   \\ \hline
\NAMEONLY                                  & -1.74                             & -0.87                        & -0.7                                & -4.04                                 \\
\ORIGIN                                    & {\bf -0.63}                       & {\bf +0.26}                  & {\bf +2.86}                         & {\bf -2.16}                           \\ \hline
  \bottomrule
\end{tabular}
}
\end{center}
\caption{\label{tbl:hete-style-results-sgd2} Results on unseen service with heterogeneous description styles on \sgdst dataset. More results and qualitative analysis are in the appendix~\ref{sec:sgd:more-desc-results}}
\end{table}

%%% Local Variables:
%%% mode: latex
%%% TeX-master: "../thesis-main.ltx"
%%% End:
