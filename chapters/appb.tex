\chapter[Combinations of Schema Descriptions]{Combinations of
  Schema Descriptions}
\label{chap:appendix:schema}
In schema-guided dialogue modeling, as shown
in~\autoref{fig:schema-dst} and ~\autoref{sec:sgd:decompose-y}, it
shows three main schema components: service, intent, slot. Natural
language descriptions are also added to the three components. This
appendix includes the ablation study on the combinations of the three
descriptions.

\section{Combinatory Settings}
\label{sec:sgd:com-desc}
For each subtask, the key description element must be included, \eg,
intent description for intent task, and value for categorical slot
tasks. However, besides the description about intents and slots, we
also have service description and the names for intents and
slots. What other information will help? and what kind of composition
will offer better performance. In this section, we will mainly answer
the questions with the following experiments on compostion of
descriptions.

To show how each component helps schema-guided dialog state tracking,
we incrementally add richer schema component one by one.

\begin{itemize}
\item \textbf{ID:}~This is the least informative case: we only use
  meaningless intent/slot identifiers, \eg, \kw{Intent\_4}, \kw{Slot\_2}. It means we don't use
  description from any schema component. We want to investigate how a
  simple identifier-based description performs in schema-guided dialog
  modeling, and the performance lower-bound on transferring to unseen services.

\item \textbf{I/S Desc:}~Only using the original intent/slot descriptions
  of intent/slot in \sgdst~and~\multiwoz~dataset for corresponding
  tasks.

\item \textbf{Service + I/S Desc:}~Adding a service description to the above
  original intent/slot descriptions. Service descriptions summarize the
  functionalities of the whole service, hence may offer extra background
  information for intent and slots.
\end{itemize}

For categorical slot value detection, we simply add the
value after each of the above composition.

\begin{table}[!t]
  \caption{\label{tbl:schema-component-results} Models using different
    composition of schema, results on test set of \sgdst and our
    remixed \multiwoz. For zero-shot performance on MultiWOZ, we
    simply merge the zero-shot performance on each domain.}
\begin{center}
\setlength{\tabcolsep}{2pt}
\begin{tabular}{l|cccc|cc}
\toprule
\hline
Model\textbackslash{Task} & \multicolumn{4}{c|}{SG-DST} & \multicolumn{2}{c}{MultiWOZ}                                        \\
                          & Intent                      & Req         & Cat         & NonCat      & Cat         & NonCat      \\ \hline
  \multicolumn{7}{c}{{\bf Seen Service}}                                                                                      \\ \hline
Identifer                 & 92.76                       & 99.70       & 87.86       & 88.38       & 58.46       & 77.29       \\
I/S Desc                  & {\bf 95.35}                 & {\bf 99.74} & 92.10       & {\bf 93.52} & {\bf 85.84} & {\bf 83.67} \\
Service + I/S Desc        & 95.28                       & {\bf 99.74} & {\bf 93.19} & 92.34       & 85.07       & 80.56       \\ \hline
  \multicolumn{7}{c}{{\bf Unseen Service}}                                                                                    \\ \hline
Identifier                & 50.63                       & 88.74       & 54.34       & 10.77       & 53.05       & 56.18       \\
I/S Desc                  & {\bf 92.17}                 & {\bf 98.16} & {\bf 68.88} & 69.84       & 56.49       & {\bf 61.39} \\
Service + I/S Desc        & 86.95                       & 97.99       & 67.08       & {\bf 71.30} & {\bf 60.58} & 59.63       \\
\hline
\bottomrule
\end{tabular}
\end{center}
\end{table}
\section{Results on Description Combinations}
\label{sec:sgd:com-results}
Table~\ref{tbl:schema-component-results} shows the results of using
different description compositions. First, there are consistent
findings across datasets and subtasks:
\begin{inparaenum}[(1)]
 \item using meaningless identifier as intent/slot description shows
  the worse performance on all tasks of both datasets, and can
  not generalize well to unseen services.
 \item using intent/slot descriptions can largely boost the performance,
  especially on unseen services.
\end{inparaenum}

However, the impact of service description varies by tasks. For
example, it largely hurts performance on intent classification task,
but does not impact %or slightly overfitting on
requested slot and categorical slot tasks. According to manual
analysis of \sgdst~and~\multiwoz dataset, we found that service
description consists of the main functions of the service, especially
the meaning of the supported intents. Hence, using service description
for intent causes confusion between the intent description information
and other supported intents. Moreover, in categorical slot value
prediction task, the most important information is the slot
description and value.  When adding extra information from service
description, it improves marginally on seen service while not
generalizing well on unseen services, which indicates the model learns
artifacts that are not general useful for unseen services.

Finally, on noncategorical slot tasks, the impact of service
description may also varies on datasets. On \sgdst, there are 16
domains and more than 30 services, the rich background context from
service description contains both domain and service-specific
information, which seems to help both seen and unseen
services. However, on \multiwoz, it hurts the performance on seen
service {\it restaurant} the most, while improving the performance on
the unseen service {\it hotel} by 4 points. In this case, it works
like a regularizer rather than a definitive clues. Because in
\multiwoz, there are only 8 domains, and one service per domain, thus
service descriptions just contain domain related information without
much extra information, it will not help the model to detect the span
for the slot.
%
%\subsection{Statistics on Schema Name and Description}
%\label{ssec:appendices-schema-statistics}
%
%We analyze the structure pattern of the schema description in SG-DST and
%MultiWOZ datasets, and summerize them in Table \ref{tbl:desc_stat}
%
%\begin{table}[!h]
%\begin{center}{\small
%\setlength{\tabcolsep}{3pt}
%\begin{tabular}{l|cc|c}
%\hline
%\multicolumn{4}{c}{Desc in SG-DST}        \\ \hline
%        & Freq Patterns & Count & Example \\ \hline
%Service &               &       & \\
%Intent  &               &       & \\
%Slot    &               &       & \\ \hline
%\multicolumn{4}{c}{Desc in MultiWOZ}      \\ \hline
%Service &               &       & \\
%Intent  &               &       & \\
%Slot    &               &       & \\
%\hline
%\end{tabular}}
%\end{center}
%\caption{\label{tbl:desc_stat} Service/Intent/Slot names in SG-DST and MultiWOZ datasets}
%\end{table}

%%% Local Variables:
%%% mode: latex
%%% TeX-master: "../dissertation-main.ltx"
%%% End:
