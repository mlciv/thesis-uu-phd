\section{Experiments and Results}
\label{sec:lex-phr:exp}

\subsection{Dataset and Evaluation}
\label{ssec:data_eval}

For DM, PSD, we split the training set by taking WSJ section (00-19)
as training, and section 20 as dev set. For other datasets, when
developing and parameter tuning, we use splits with a ratio of
25:1:1. In our submitted model, we did not use multitask learning for
training. Following the unified MRP metrics in the shared tasks, we
train our model based on the development set and finally evaluate on
the private test set.  For more details of the metrics, please refer
to the summarization of the MRP~2019 task~\cite{Oep:Abe:Haj:19},
\begin{table}[!ht]
\begin{center}
\small
\begin{tabular}{l|ccc}
\toprule
\hline
                                 & \multicolumn{3}{c}{{\bf Lexicon Anchoring}}                         \\ \cline{2-4}
                                 & DM                    & PSD               & AMR                     \\ \hline
Root                             & 1                     & $ \geq 1$ (11.56\%)  & 1                       \\ \hline
Node Label                       & Lemma                 & Lemma(*)          & Lemma(*) + NeType(143+) \\ \hline
\multirow{2}{*}{Node Properties} & POS                   & POS               & constant values         \\
                                 & SEM-I(160*)\_args(25) & wordid\_sense(25) & polarity, Named entity  \\ \hline
Edge Label                       & (45)                  & (91)              & (94+)                   \\ \hline
Edge Properties                  & N/A                   & N/A               & N/A                     \\ \hline
Connectivity                     & True                  & True              & True                    \\ \hline
Training Data                    & 35656                 & 35656             & 57885                   \\ \hline
Test Data                        & 3269                  & 3269              & 1998                    \\ \hline \bottomrule
\end{tabular}
\end{center}
\caption{The detailed lex-anchoring classifiers in our model. The
  round bracket means the number of ouput classes of our classifiers,
  * means copy mechanism is used in our classifier. }
\label{tbl:lex:impl_lex}
\end{table}


\begin{table}[!ht]
\caption{The detailed lex-anchoring classifiers in our model. The round bracket means the number of ouput classes of our classifiers, * means copy mechanism is
  used in our classifier.}
\label{tbl:phr:impl_phrasal}
\begin{center}
\begin{tabular}{l|cc}
\toprule
\hline
                                 & \multicolumn{2}{c}{{\bf Phrase Anchoring}} \\ \cline{2-3}
                                 & UCCA      & TOP                            \\ \hline
Root                             & 1         & 1                              \\ \hline
Node Label                       & N/A       & Intent(25), Slot(36)           \\ \hline
\multirow{2}{*}{Node Properties} & N/A       & N/A                            \\
                                 & N/A       & N/A                            \\ \hline
Edge Label                       & (15)      & N/A                            \\ \hline
Edge Properties                  & ``remote" & N/A                            \\ \hline
Connectivity                     & True      & True                           \\ \hline
Training Data                    & 6485      & 35741                          \\ \hline
Test Data                        & 1131      & 9042                           \\ \hline \bottomrule
\end{tabular}
\end{center}
\label{tbl:phr:impl_phrasal}
\end{table}

\subsection{Summary of Implementation}
\label{ssec:impl_summary}

We summarize our implementation for lexical-anchoring
in~\autoref{tbl:lex:impl_lex}, and phrasal-anchoring
in~\autoref{tbl:phr:impl_phrasal}. As we mentioned in the previous
sections, we use latent-alignment graph-based parsing for lexical
anchoring MRs~(DM, PSD, AMR), and use CKY-based constituent parsing
phrasal anchoring in MRs~(UCCA, TOP). This section gives information
about various decision for our models.

\subsubsection{Root}
\label{sssec:lex:root}

The first row ``Root" shows the numbers of root nodes in the graph.
We can see that for PSD, 11.56\% of graphs with more than 1 top
nodes. In our system, we predict one and only one root node with a
N~(N is size of already identified nodes) way classifier, and then fix
this with a post-processing strategy. When our model predicts one node
as the root node, and if we find additional coordination nodes with
it, we add the coordination node also as the top node.

\subsubsection{Node}
\label{sssec:lex:node}

Except for UCCA, all other four MRs have labeled
nodes, the row ``Node Label" shows the templates of a node label. For
DM and PSD, the node label is usually the lemma of its underlying
token. But the lemma is neither the same as one in the given companion
data nor the predicted by Stanford Lemma Annotators. One common
challenge for predicting the node labels is the open label set
problem. Usually, the lemma is one of the morphology derivations of
the original word. But the derivation rule is not easy to create
manually. In our experiment, we found that handcrafted rules for lemma
prediction only works worse than classification with copy mechanism,
except for DM.

For AMR, there are other components in the node labels beyond the
lemma. Especially, the node label for AMR also contains more than 143
fine-grained named entity types. In addition to the node label, the
properties of the label also need to be predicted. Among them, node
properties of DM are from the SEMI sense and arguments handler, while
for PSD, senses are constrained the senses in the predefined the
vallex lexicon.

\subsubsection{Edge}
\label{sssec:lex:edge}

Edge predication is another challenge in our task
because of its large label set (from 45 to 94) as shown in row ``Edge
Label", the round bracket means the number of output classes of our
classifiers. For Lexical anchoring MRs, edges are usually connected
between two tokens, while phrasal anchoring needs extra effort to
figure out the corresponding span with that node. For example, in UCCA
parsing, To predict edge labels, we first predicted the node spans,
and then node labels based that span, and finally we transform back
the node label into edge label.

\subsubsection{Connectivity}
\label{sssec:lex:connectivity}

Beside the local label classification for
nodes and edges, there are other global structure constraints for all
five MRs: All the nodes and edges should eventually form a connected
graph. For lexical anchoring, we use MSCG algorithm to find the
maximum connected graph greedily; For phrasal anchoring, we use
dynamic programming to decoding the constituent tree then
deterministically transforming back to a connected UCCA Graph
\footnote{Due to time
  constraint, we ignored all the discontinuous span and remote edges
  in UCCA}

\subsection{Model Setup}
\label{ssec:exp_setup}

For lexical-anchoring model setup, our network mainly consists of node
and edge prediction model. For AMR, DM, and PSD, they all use one
layer Bi-directional LSTM for input sentence encoder, and two layers
Bi-directional LSTM for head or dependent node encoder in the
bi-affine classifier. For every sentence encoder, it takes a sequence
of word embedding as input (We use 300 dimension Glove here), and then
their output will pass a softmax layer to predicting output
distribution. For the latent AMR model, to model the posterior
alignment, we use another Bi-LSTM for node sequence encoding. For
phrasal-anchoring model setup, we follow the original model set up in
\citet{kitaev2018constituency}, and we use 8-layers 8-headers
transformer with position encoding to encode the input sentence.

For all sentence encoders, we also use the character-level CNN model
as character-level embedding without any pre-trained deep
contextualized embedding model. Equipping our model with Bert or
multi-task learning is promising to get further improvement. We leave
this as our future work.

Our models are trained with Adam~\cite{kingma2014adam}, using a batch
size 64 for a graph-based model, and 250 for CKY-based
model. Hyper-parameters were tuned on the development set, based on
labeled F1 between two graphs. We exploit early-stopping to avoid
over-fitting.

\subsection{Results}
\label{ssec:results}
At the time of official evaluation in MRP'2019 shared task, we
submitted the unified lexical anchoring parser for DM, PSD and AMR,
and then we submitted another phrasal-anchoring model for UCCA parsing
during post-evaluation stage, and we leave EDS parsing as future
work. The following sections are the official results and error
breakdowns for lexical-anchoring and phrasal-anchoring
respectively. For the results of TOP Parsing, we use a seperate table
for it.

\Paragraph{Official Results on Lexical Anchoring}
Table \ref{tbl:results_rank} shows the official results for our
lexical-anchoring models on AMR, DM, PSD.  By using our latent
alignment based AMR parser, our system ranked top 1 in the AMR subtask,
and outperformed the top 5 models in large margin. Our parser on PSD
ranked 6, but only 0.02\% worse then the top 5 model. However, official
results on DM and PSD shows that there is still around 2.5 points
performance gap between our model and the top 1 model.

\begin{table}[!h]
\caption{\label{tbl:results_rank} Official results overview on unified MRP metric, we selected the performance from top 1/3/5 system(s) for comparison}
\centering
\begin{tabular}{lll}
\toprule
MR     & Ours~(P/R/F1) & Top 1/3/5~(F1)  \\ \hline
AMR(1) & {\bf 75/71/73.38}   & {\bf 73.38/71.97/71.72} \\
PSD(6) & 89/89/{\bf 88.75}   & 90.76/89.91/{\bf 88.77} \\
DM(7)  & 93/92/92.14   & 94.76/94.32/93.74 \\ \hline
%UCCA   & 76/68/71.65   & 81.67/77.80/73.22 \\
%EDS    & N/A           & 94.47/90.75/89.10 \\ \bottomrule
\end{tabular}
\end{table}

\Paragraph{Official Results on Phrasal Anchoring~(UCCA)} Table
\ref{tbl:ucca_results_rank} shows that our span-based CKY model for
UCCA can achieve 74.00 F1 score on official test set, and ranked
5th. When adding ELMo~\cite{peters2018deep} into our model, it can further improve almost 3
points on it.
\begin{table}[!h]
  \small
\centering
\begin{tabular}{lll}
\toprule
MR     & Ours~(P/R/F1) & Top 1/3/5~(F1)  \\ \hline
UCCA(5)   & 80.83/73.42/\textbf{76.94}   & 81.67/77.80/73.22 \\
% EDS    & N/A                 & 94.47/90.75/89.10 \\ \bottomrule
\end{tabular}
\caption{\label{tbl:ucca_results_rank} Official results overview on
  unified MRP metric, we selected the performance from top 1/3/5
  system(s) for comparison. It shows our UCCA model for post-evluation
  can rank 5th}
\end{table}


\Paragraph{Results on Phrasal Anchoring~(TOP)} \autoref{tbl:top_results_rank} shows the results on TOP parsing. We use our CKY model skeleton but with different underlying contextualized span representation from ELMo, Roberta, and SpanBERT. Below the line are the baseline models for TOP parsing. Our plain model with glove-based transformer model can outperform all the previous baseline models. Contualized presentation from pretrained language model can boost the performance further.

\begin{table}[!h]
  \small
\centering
\begin{tabular}{cccccc}
  \toprule
  Model           & Exact Match & P     & R     & F1    \\ \hline
  Ours            & 80.85       & 92.77 & 91.05 & 91.9  \\
  Ours + ELMo     & 83.04       & 93.57 & 92.37 & 92.97 \\
  Ours + Roberta  & 85.89       & 96.07 & 93.08 & 94.92 \\
  Ours + SpanBERT & 85.82       & 95.55 & 94.44 & 94.99 \\ \hline
  RNNG                & 78.51       & 90.62 & 89.84 & 90.23 \\
  Seq2Seq-CNN         & 75.87       & 89.25 & 87.88 & 88.56 \\
  Seq2Seq-LSTM        & 75.31       & 88.35 & 87.03 & 87.69 \\
  Seq2Seq-Transformer & 72.2        & 87.09 & 86.11 & 86.6  \\
  \bottomrule
\end{tabular}
\caption{\label{tbl:top_results_rank} Results on TOP Parsing}
\end{table}

\subsection{Error Breakdown}
\label{ssec:error_breakdown}
Table \ref{tbl:results_amr}, \ref{tbl:results_dm},
\ref{tbl:results_psd} and \ref{tbl:results_ucca} shows the detailed
error breakdown of AMR, DM, PSD and UCCA respectively. Each column in
the table shows the F1 score of each subcomponent in a graph: top
nodes, node lables, node properties, node anchors, edge labels, and
overall F1 score. No anchors for AMR, and no node label and propertis
for UCCA. We show the results of MRP metric on two datasets. ``all"
denotes all the examples for that specific MR, while lpps are a set of
100 sentences from \texttt{The Little Prince}, and annotated in all five
meaning representations. To better understand the performance, we also
reported the official results from two baseline models
TUPA~\cite{Her:Arv:19} and ERG~\cite{Oep:Fli:19}.

\begin{table}[!ht]
\small
\centering
\setlength{\tabcolsep}{3pt}
\begin{tabular}{lcccccc}
\toprule
                          & data & tops              & labels & prop  & edges & all   \\ \hline
\multirow{2}{*}{ \parbox{1cm}{TUPA
single} }                 & all  & 63.95             & 57.20  & 22.31 & 36.41 & 44.73 \\
                          & lpps & 71.96             & 55.52  & 26.42 & 36.38 & 47.04 \\ \hline
\multirow{2}{*}{ \parbox{1cm}{TUPA
multi} }                  & all  & 61.30             & 39.80  & 27.70 & 27.35 & 33.75 \\
                          & lpps & 72.63             & 50.11  & 20.25 & 33.12 & 43.38 \\ \hline
\multirow{2}{*}{ Ours(1)} & all  & \underline{65.92} & 82.86  & {\bf 77.26} & 63.57 & {\bf 73.38} \\
                          & lpps & \underline{72.00} & 78.71  & 58.93 & 63.96 & 71.11 \\ \hline
\multirow{2}{*}{ Top 2}  & all  & 78.15             & 82.51  & 71.33 & 63.21 & 72.94 \\
                          & lpps & 83.00             & 76.24  & 51.79 & 60.43 & 69.03 \\ \bottomrule
\end{tabular}
\caption{\label{tbl:results_amr} Our parser on AMR ranked 1st. This table shows the error breakdown when comparing to the baseline TUPA model and top 2~\cite{Che:Dou:Xu:19} in official results}
\end{table}


\begin{table}[!h]
\small
\centering
\setlength{\tabcolsep}{2.5pt}
\begin{tabular}{cccccccc}
  \toprule
                            & data & tops              & labels & prop  & anchors & edges & all   \\ \hline
  \multirow{2}{*}{ ERG }    & all  & 91.83             & 98.22  & 95.25 & 98.82   & 90.76 & 95.65 \\
                            & lpps & 95.00             & 97.32  & 97.75 & 99.46   & 92.71 & 97.03 \\
  \multirow{2}{*}{Top 1}    & all  & 93.23             & 94.14  & 94.83 & 98.40   & 91.55 & 94.76 \\
                            & lpps & 96.48             & 91.85  & 94.36 & 99.04   & 93.28 & 94.64 \\
  \multirow{2}{*}{ Ours(7)} & all  & \underline{70.95} & 93.96  & 92.13 & 97.25   & 86.45 & 92.14 \\
                            & lpps & \underline{84.00} & 90.55  & 91.91 & 97.96   & 87.24 & 91.82 \\ \bottomrule
\end{tabular}
\caption{\label{tbl:results_dm} Our parser on DM ranked 7th. This table shows the error breakdown when comparing to the model ranked Top 1~\cite{Li:Zha:Zha:19} in official results}
\end{table}



\begin{table}[!h]
\small
\centering
\setlength{\tabcolsep}{2.5pt}
\begin{tabular}{cccccccc}
\toprule
                          & data & tops              & labels & prop  & anchors & edges & all   \\ \hline
%\multirow{2}{*}{ TUPA
%single }                 & all  & 53.07             & 64.58  & 43.30 & 79.50   & 32.96 & 52.97 \\
%                         & lpps & 59.90             & 66.44  & 42.46 & 81.45   & 38.11 & 55.73 \\
%\multirow{2}{*}{ TUPA
%multi }                  & all  & 53.07             & 64.58  & 43.30 & 79.50   & 32.96 & 52.97 \\
%                         & lpps & 59.90             & 66.44  & 42.46 & 81.45   & 38.11 & 55.73 \\
\multirow{2}{*}{Top 1}   & all  & 93.45             & 94.68  & 91.78 & 98.35   & 77.79 & 90.76 \\
                          & lpps & 93.33             & 91.73  & 84.37 & 98.40   & 77.63 & 88.34 \\
\multirow{2}{*}{ Ours(6)} & all  & \underline{82.01} & 94.18  & 91.28 & 96.94   & 72.40 & 88.75 \\
                          & lpps & \underline{85.85} & 90.48  & 82.63 & 95.97   & 73.60 & 85.83 \\ \bottomrule
\end{tabular}
\caption{\label{tbl:results_psd} Our parser on PSD ranked 6th. This table shows the error breakdown when comparing to the model ranked top 1~\cite{Don:Fow:Gro:19} in official results}
\end{table}

\subsubsection{Error Analysis on Lexical-Anchoring}
As shown in Table \ref{tbl:results_amr}, our AMR parser is good at
predicting node properties and consistently perform better than other
models in all subcomponent, except for top prediction. Node properties
in AMR are usually named entities, negation, and some other quantity
entities. In our system, we recategorize the graph fragements into a
single node, which helps for both alignments and structured inference
for those special graph fragments. We see that all our 3 models
perform almost as good as the top 1 model of each subtask on node
label prediction, but they perform worse on top and edge
prediction. It indicates that our bi-affine relation classifier are
main bottleneck to improve. Moreover, we found the performance gap
between node labels and node anchors are almost consistent, it
indicates that improving our model on predicting \texttt{NULL} nodes may
further improve node label prediction as well.  Moreover, we believe
that multi-task learning and pre-trained deep models such as
BERT~\cite{devlin2018bert} may also boost the performance of our paser
in future.

\subsubsection{Error Analysis on Phrasal-Anchoring}

According to Table \ref{tbl:results_ucca}, our model with ELMo works
slightly better than the top 1 model on anchors prediction. It means
our model is good at predicting the nodes in UCCA and we belive that
it is also helpful for prediction phrasal anchoring nodes in EDS.

\begin{table}[!h]
\small
\centering
\setlength{\tabcolsep}{2.5pt}

\begin{tabular}{ccccccccc}
\toprule
                              & data & tops  & anchors & edge  & attr  & all   \\ \hline
\multirow{2}{*}{ TUPA
single }                      & all  & 78.73 & 69.17   & 16.96 & 15.18 & 27.56 \\
                              & lpps & 86.03 & 76.26   & 28.32 & 24.00 & 40.06 \\
\multirow{2}{*}{ TUPA
multi }                       & all  & 84.92 & 65.74   & 12.99 &  9.07 & 23.65 \\
                              & lpps & 88.89 & 77.76   & 26.45 & 18.32 & 41.04 \\
\multirow{2}{*}{\cite{Che:Dou:Xu:19}}       & all  & 1.00  & 95.36   & 72.66 & 61.98 & 81.67 \\
                              & lpps & 1.00  & 96.99   & 73.08 & 48.37 & 82.61 \\ \hline
\multirow{2}{*}{ Ours(*5)}    & all  & 98.85 & 94.92   & 60.17 & 0.00  & {\bf 74.00} \\
                              & lpps & 96.00 & 96.75   & 60.20 & 0.00  & 75.17 \\
\multirow{2}{*}{ Ours + ELMo} & all  & 99.38 & 95.70   & 64.88 & 0.00  & {\bf 76.94} \\
                              & lpps & 98.00 & 96.84   & 66.63 & 0.00  & 78.77 \\ \bottomrule
\end{tabular}
\caption{\label{tbl:results_ucca} Our UCCA parser in post-evaluation ranked 5th according to the original official evaluation results. This table shows the error breakdown when comparing to the model ranked top 1~\cite{Che:Dou:Xu:19} in official results. * denotes the ranking of post-evaluation results }
\end{table}

However, when predicting the edge and edge attributes, our model
performs 7-8 points worse than the top 1 model. In UCCA, an edge label
means the relation between a parent nodes and its children. In our
UCCA transformation, we assign edge label as the node label of its
child and then predict with only child span encoding. Thus it actually
misses important information from the parent node. Hence, in future,
more improvement can be done to use both child and parent span
encoding for label prediction, or even using another span-based
bi-affine classifier for edge prediction, or remote edge recovering.

%%% Local Variables:
%%% mode: latex
%%% TeX-master: "../../dissertation-main.ltx"
%%% End:
