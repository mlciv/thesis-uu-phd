\section{Chapter Summary}
\label{sec:conclusion}

In summary, by analyzing the AMR alignments, we show that implicit AMR
anchoring is actually lexical anchoring, where each output node can
implict be aligned to a token or entity in the input sentence. Thus we
propose to regroup five representations as two groups: lexical
anchoring~(DM,PSD,AMR) and phrasal anchoring~(UCCA,TOP). Furthermore,
based on the assumption of independent factorization and the
structural inductive biases about different anchoring-types, we
proposed two parsing framework to support lexical and phrasal
anchoring respectively.  For lexical anchoring, we suggest to parse
DM, PSD, and AMR in a unified parsing framework based on latent
alignment model, which supports both~\textbf{explicit
  lexical-anchoring} and \textbf{implicit lexical anchoring}. Our
submission ranked top 1 in AMR sub-task, ranked 6th and 7th in PSD and
DM tasks. For phrasal anchoring, by reducing UCCA and TOP graph into
constituent tree structures, and we use the same span-based CKY parser
to predict their diverse tree structures. Our phrasal-anchoring parser
ranked 5th in UCCA official post-evaluation stage, and outperform
serveral baselines on TOP parsing.

%%% Local Variables:
%%% mode: latex
%%% TeX-master: "../../dissertation-main.ltx"
%%% End:
