The design and implementation of broad-coverage and linguistically
motivated meaning representation frameworks for natural language is
attracting growing attention in recent years. With the advent of deep
neural network-based machine learning techniques, we have made
significant progress to automatically parse sentences intro structured
meaning
representation~\citep{Oep:Kuh:Miy:14,Oep:Kuh:Miy:15,May:2016wc,hershcovich-etal-2019-semeval}. Moreover,
the differences between various representation frameworks has a
significant impact on the design and performance of the parsing
systems.

Due to the abstract nature of semantics, there is a diverse set of
meaning representation frameworks in the
literature~\citep{abend2017state}. In some application scenario,
tasks-specific formal representations such as database queries and
arithmetic formula have also been proposed. However, primarily the
study in computational semantics focuses on frameworks that are
theoretically grounded on formal semantic theories, and sometimes also
with assumptions on underlying syntactic structures.

Anchoring is crucial in graph-based meaning representation
parsing. Training a statistical parser typically starts with a
conjectured alignment between tokens/spans and the semantic graph
nodes to help to factorize the supervision of graph structure into
nodes and edges. In this chapter, with evidence from previous research
on AMR
alignments~\citep{Pourdamghani:2014aligning,Flanigan:2014vc,Wang:2017vt,chen2017unsupervised,szubert2018structured,lyu2018amr},
we propose a uniform handling of three meaning representations from
\texttt{Flavor-0} (DM, PSD) and \texttt{Flavor-2} (AMR) into a new
group referred to as the \textbf{lexical-anchoring} MRs. It supports
both explicit and implicit anchoring of semantic concepts to tokens.
The other two meaning representations from \texttt{Flavor-1} (EDS,
UCCA) is referred to the group of \textbf{phrasal-anchoring} MRs where
the semantic concepts are anchored to phrases as well.

To support the simplified taxonomy, we named our parser as
LAPA~({\textbf{L}exical-\textbf{A}nchoring and
\textbf{P}hrasal-\textbf{A}nchoring)\footnote{The code is available
online at \url{https://github.com/utahnlp/lapa-mrp}}. We proposed a
graph-based parsing framework with a latent-alignment mechanism to
support both explicit and implicit lexicon anchoring. According to
official evaluation results, our submission for this group ranked 1st
in the AMR subtask, 6th on PSD, and 7th on DM respectively, among 16
participating teams. For phrasal-anchoring, we proposed a CKY-based
constituent tree parsing algorithm to resolve the anchor in UCCA, and
our post-evaluation submission ranked 5th on UCCAsubtask.
%%% Local Variables:
%%% mode: latex
%%% TeX-master: "../../thesis-main.ltx"
%%% End:
