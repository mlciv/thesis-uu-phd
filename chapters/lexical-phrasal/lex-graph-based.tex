\section[Lexical Anchoring: Latent Alignment Model for Graph-Based
Parsing]{Lexical Anchoring: Latent Alignment \\Model for Graph-Based Parsing}
\label{sec:lex-phr:graph-based}

In this section, we will first introduce the two-stage framework for parsing
the DM, PSD, and AMR graphs~(\autoref{ssec:lex-phr:two-stage}). Then we
resolve the alignment problem with a latent alignment model~(\autoref{ssec:lex-phr:latent-alignment})


According to the linguistic analysis on anchoring, we have shown that
DM, PSD and AMR belong to the lexical-anchoring, which indicates that
their nodes in the output graph are explicitly or implicitly anchored
to the lexical units of the corresponding input sentence.  Before
showing the unified design of independent factorization, let us
introduce the fundamental concepts and notations.

We refer to words in a sentence as
$x=(x_{0};x_{1};...x_{i};...;x_{n})$, where $n$ is the sentence
length.  For all the graph-based representations in our
lexical-anchoring category, we decompose the whole graph into nodes
and edges. We denotes the labelled nodes~(concepts) as
$C = \{c_{i}\mid i \in [0,m]\}$, where $m$ is the number of concepts.
While the labelled edges~(relations) between the $m$ concepts are
denoted as $R = \{r_{ij}\mid i \in [0, m], j \in [0, m]\}$.  $r_{ij}$ means
the directional relation from the node $i$ to the node $j$.  We use
$r_{ij}=\Phi$ to indicate no edge from the node $i$ to the node $j$.

In this part, based on the above notations, we introduce how to do
independent factorization on lexical-anchoring by addressing the three
main chellenges in \textbf{Output
  Decomposition}~(\autoref{sssec:lex-phr:lex-output-decomposition}),
\textbf{Input Decomposition and Alignments
  Discovery}~(\autoref{sssec:lex-phr:lex-input-decomposition}), and
\textbf{Factor Modelling}~(Two-stage
Parsing~(\autoref{ssec:lex-phr:two-stage} and Latent Alignment
Model~\autoref{ssec:lex-phr:latent-alignment})

\subsection{Independent Factorization on Lexical-Anchoring}
\label{ssec:lex-phr:lex-factorization-analysis}
In the following, we focus on the first two main challenges in
independent factorization: \textbf{output decomposition} and
\textbf{input decomposition} on lexical-anchoring. We leave the
\textbf{factor modelling} into the next part.  We take a more
complicated AMR graph in~\autoref{fig:bg:amr} as an example, for the
sentence, \kw{Pierre Vinken, 61 years old, will join the board as a
  nonexecutive director Nov.29.} We introduce the details of
independent factorization for AMR and other lexical-anchoring
representations.

%\begin{figure}[t]
%\centering
%\includegraphics[width=.96\textwidth,trim=.1cm .1cm .05cm .05cm, clip]{eds-white.pdf}
%\caption{\label{fig:phrasal-anchoring}Phrasal-anchoring in
%  EDS[wsj\#0209013], for the sentence \texttt{\char`\"A similar
%    technique is almost impossible to apply to other crops, such as
%    cotton, soybeans and rice.\char`\"}. Bold nodes are similar to the
%  non-terminal nodes in UCCA, which are anchored multiple tokens, thus
%  overlapping with the anchors of other nodes.}
%\end{figure}

\subsubsection{Output Decomposition}
\label{sssec:lex-phr:lex-output-decomposition}
For the lexical-anchoring graph-based representations~(DM, PSD, AMR),
the target graph can be decomposed into independent nodes or
subgraphs. For DM and PSD, each node in the target graph is strictly
aligned to the corresponding token. Hence, we simply decompose the DM
and PSD graph by nodes. However, when we introduce the necessary of
structural inductive biases for independent
factorization~(\autoref{sssec:intro:structural-biases}), the AMR
parsing running examples shows that we need to handle those special
entities in AMR graph.  From the training data alone, we cannot easily
tell how to segment the AMR graphs. However, according to the
annotation guideline of AMR and previous work on subgraph
templating~\cite{Werling:2015up} or abstract concept
label~\cite{Wang:2017vt}, the main intuition of grouping is to ensure
that concepts are rarely lexically triggered (e.g., \tquoted{person}
and \dquoted{have-org-role-91} in~\autoref{fig:bg:amr} get grouped
together with lexically triggered nodes~(e.g., \tquoted{Pierre
  Vinken}, and \dquoted{director}). Finally, we adopt the templates
proposed by~\citet{corro2019learning} including:
\tquoted{thing}~(e.g., \tquoted{(opine-01 :ARG-01 thing)} for
\tquoted{opinion}), \tquoted{person}~(e.g., \tquoted{(play-01 :ARG1
  person)} for \tquoted{player}), \tquoted{more},
\tquoted{most}~(e.g., \tquoted{(have-degree-91 :ARG2 good-02) :ARG3
  more)} for \tquoted{better}),
\tquoted{date-entity}~(e.g.,\tquoted{(date-entity :month 2)} for
\tquoted{February}), and all the special quantity entities,
\tquoted{xx-quantity}~(e.g., \tquoted{(monetary-quantity :quant 20
  :unit dollar)} for \tquoted{\$20}).~\footnote{Please refer to the
  paper of~\citet{lyu2018amr} for more details about the recategorization.}

AMR output decomposition will be the rectangled segments shown in
\autoref{fig:lex-phr:amr-decomposition}.
\begin{figure}[!tbp]
  \centering
  \includegraphics[width=0.90\textwidth]{pierre-decomposition.pdf}
  \caption{\label{fig:lex-phr:amr-decomposition} AMR output
    decomposition for the sentence \#20001001}
\end{figure}

\subsubsection{Input Decomposition and Alignments Discovery}
\label{sssec:lex-phr:lex-input-decomposition}

According to the bi-lexical dependency structures of DM and PSD, and
implicit lexical token anchoring on AMR, the nodes/categorized graph
fragments of DM, PSD, and AMR are anchored to surface lexical units in
an explicit or implicit way. Especially, those lexical units do not
overlap with each other, and most of them are just single-tokens,
multiple word expression, or named entities. In other words, when
parsing a sentence into DM, PSD, AMR graphs, and tokens in the original
sentence can be merged by looking up a lexicon dict when preprocessing
and then may be considered as a single token for aligning or parsing.

The more difficult challenge exists in how to align the decomposed
input with the decomposed output. According to the previous anchoring
analysis, the training data of DM and PSD naturally contains the
anchoring information, while the AMR training data doesn't offer the
alignments. Hence, for AMR, we need to an extra model to resolve the
alignments discovery problem. According to the AMR annotation
guideline, a strong inductive bias about the alignment is that
\emph{each token or expressive is not overlapping and is
  exclusively aligned to the output decompositions.} We will
introduce the details of the latent alignment model in
\autoref{ssec:lex-phr:latent-alignment}.

%%% Local Variables:
%%% mode: latex
%%% TeX-master: "../../dissertation-main.ltx"
%%% End:


\subsection{Two-stage Graph-Based Parsing}
\label{ssec:lex-phr:two-stage}

Before formulating the graph-based model into a probabilistic model
as~\autoref{eq:graph_prob}, we denote some notations: $C$, $R$ are
sets of concepts (nodes) and relations (edges) in the graph, and $w$
is a sequence of tokens.  $a \in {\mathbb{Z}}^m$ as the alignment
matrix, each $a_{i}$ is the index of aligned token where $i$th node
aligned to. When modeling the negative log likelihood loss~(NLL), with
independence assumption between each node and edge, we decompose it
into node- and edge-identification pipelines.
\begin{equation}
  \label{eq:graph_prob}
\begin{aligned} \smaller[2]
 & NLL(P(C,R \mid w)) \\
 & = - \log(P(C,R \mid w)) \\
 & = - \log(\sum_{a}{P(a) P(C,R \mid w,a)}) \\
 & = - \log\left(\sum_{a}P(a) P(C \mid w, a) P(R\mid w,a,C)\right) \\
 & = - \log\left(\sum_{a}P(a) \prod_{i}^{m} P(c_{i} \mid h_{a_{i}}) \cdot \prod_{i,j=1}^{m}P(r_{ij} \mid h_{a_{i}}, c_{i}, h_{a_{j}}, c_{j})\right)
\end{aligned}
\end{equation}

Hence, we train a joint model to maximize the above probability for
both node identification
$P(c_{i} \mid h_{a_{i}})$~(\autoref{sssec:lex-phr:node-ident}) and edge
identification
$P(r_{ij} \mid h_{{a_{i}}, c_{i},h_{a_{j}},
  c_{j}})$~(\autoref{sssec:lex-phr:edge-ident}). The alignment
information is mainly used for training. We need to marginalize out
the discrete alignment variable $a$ to jointly learning the parameters
in node identification and relation identification networks. In DM,
PSD, and AMR, every token will only be aligned once. We know the exact
alignment information in DM and PSD, while we don't have the alignment
information for AMR for training. Hence, we introduce the latent
alignment model in~\autoref{ssec:lex-phr:latent-alignment} to resolve
the learning with discrete latent alignment
variable. \autoref{fig:graph-based-inference} summerize the unifed
two-stage graph based parsing framework. In the following subsections,
we will explain the framework in more details.

\begin{figure}[!tbp] \centering
  \includegraphics[width=1.00\textwidth]{graph-overview3.pdf}
  \caption{\label{fig:graph-based-inference} Two-stage graph-based
    parsing for the running exmaple [wsj\#0209013].}
\end{figure}

\subsubsection{Node Identification}
\label{sssec:lex-phr:node-ident}
The stage of \textbf{node identification} predicts a concept $c$ given
a word. A concept can be either \tquoted{$\Phi$} (when there is no
semantic node anchoring to that word, e.g., the word is dropped), or a
node label (e.g., lemma, sense, POS, name value in AMR, frame value in
PSD), or other node properties. One challenge in node identification
is the data sparsity issue. Many of the labels are from open sets
derived from the input token, e.g., its lemma.  Moreover, some labels
are constrained by a deterministic label set given the word. Hence, we
propose to use copy mechanism~\citep{luong2014addressing} in our
neural network architecture to decide whether to copy the
deterministic label given a word or directly estimate a classification
probability from a fixed label set. As the first stage shown
in~\autoref{fig:graph-based-inference}, each token will output its
corresponding node. Here, the two article \tquoted{The,} \tquoted{the}
and final period will produce \tquoted{$\Phi$} node. While other tokens
will copy themself and transform to the corresponding lemma, frames,
or other formations.

\subsubsection{Edge Identification}
\label{sssec:lex-phr:edge-ident}
By assuming the independence of each edge, we model the edges
probabilites independently.  Given two nodes and their underlying
tokens, we predict the edge label as the semantic relation between the
two concepts with a biaffine classifier~\cite{dozat2016deep}. As the
stage 2 shown in~\autoref{fig:graph-based-inference}, the model will
predict all the edges between the $n$ nodes produced in the node
identification stage, thus $n*n$ edges. For each edge, it will be
classified into $r$ labels and forms a cube as $n*n*r$. The $r$ edge
labels include \tquoted{:COREF} and \tquoted{:NULL}
labels. \tquoted{:COREF} indicates that the two nodes belongs to the
same entity, while \tquoted{:NULL} indicates no relation between the
two nodes.

\subsubsection{Inference}
\label{sssec:lex-phr:inference}
In our two-stage graph-based parsing, after nodes are identified, edge
identification only output a probility distribution over all the
relations between identified nodes. However, we need to an inference
algorithm to search for the \kw{maximum spanning connected
  graph}~(MSCG) from all the relations. We use
MSCG~\citep{Flanigan:2014vc} to greedily select the most valuable
edges from the identified nodes and their relations connecting
them. As shown in Figure \ref{fig:graph-based-inference}, an input
sentence goes through preprocessing, node identification, edge
identification, root identification, and MSCG to generate a final
connected graph as structured output. Finally, a postprocessing stage
will produce a valid graph according to the corresponding format of
DM, PSD, AMR.

\subsection{Latent Alignment Model}
\label{ssec:lex-phr:latent-alignment}

As the two-stage probablistic model shown
in~\autoref{eq:graph_prob}, we need to marginalize all the
alignment information $a$ to learn the above two-stage nerual networks
for node and edge identification. We do the following computing for
explicit and implicit alignments, respectively.
\begin{itemize}
\item \textbf{Explicit Alignments:} For DM, PSD, with explicit
  alignments $a*$, we can simply use $P(a^{*}) = 1.0$ and other
  alignments $P(a | a \neq a^{*}) = 0.0 $. Hence, in this case, with
  known alignment information, we don't need to worry the intractable
  marginalization problem. Further more, after reducing the latent
  alignment variable, we can easily learn the parameters for models of
  node identifiaction and edge identification.
\item \textbf{Implicit Alignments:} However, For AMR, without gold
  alignments, one requires to compute all the valid alignments and
  then condition the node- and edge-identification methods on the
  alignments.  However, it is computationally intractable to enumerate
  all combinatoral values for the discrete alignment variable. Hence,
  we estimate the latent alignments via variational inferece, which
  has been initialially used in~\citet{lyu2018amr}. In the following
  section, we introduce the details of latent alignment model via
  variational inference~(\autoref{sssec:lex-phr:vae}),
  Perturb-and-Max~(\autoref{sssec:lex-phr:perturb-and-map}), Gumbel-Sinkhorn
  networks~(\autoref{sssec:lex-phr:gumbel-sinkhorn}).
\end{itemize}

Hence, with the alignment models for both explicit and implicit
alignments, our unified two-stage parsing framework can support the
parsing on DM, PSD and AMR.

\subsubsection{Variational Inference}
\label{sssec:lex-phr:vae}
First, as shown in~\autoref{eq:elbo}, latent alignment prior $P(a_)$
is unknown for the two-stage parsing. We want estimate the latent
alignment model to ensure that it is consistent with the obversed $w$,
$C$, and $R$. Furthermore, we also need to marginal it out and jointly
learn the parameters $\theta$ and $\Phi$ to generate the concepts and
relations, respectively. Hence, we apply variational inferenc to reduce
the marginal likelihood into Evidence Lower
Bound~(ELBO)~\citep{kingma2013auto}.
\begin{equation}
 \label{eq:elbo}
\begin{aligned}
  & \log(P(C,R \mid w)) \\
  & = \log\left(\sum_{a}P(a) P(C\mid w, a) P(R\mid w,a,C)\right) \\
  & \ge E_{Q}[\log(P_{\theta}(C\mid w,a) P_{\Phi}(R\mid w,a,C))] - \kld{Q_{\Psi}(a\mid C, R,w)}{P(a)} \\
  & = \sum_{a}Q_{\Psi}(a\mid C, R,w)\log(P_{\theta}(C\mid w,a) P_{\Phi}(R\mid w,a,C))] \\
  & - \kld{Q_{\Psi}(a\mid C, R,w)}{P(a)}
\end{aligned}
\end{equation}
where $Q_{\Psi}(a|c,R,w)$ is the alignment model, parameterized by a
nerual network $\Psi$ to approximate the intractable posterior alignment
model $P_{\theta,\Phi}(a\mid C, R, w)$. The second term $D_{{KL}}$ is the
Kullback-Leibler divergence~\citep{kl-ims}, measuring the difference
between the approximated posterior and the true posterior. Please
refer to the original variational auto-encoder
paper~\citep{kingma2013auto} for more details.

\subsubsection{Perturb-and-Map}
\label{sssec:lex-phr:perturb-and-map}
To jointly learn the models according the~\autoref{eq:elbo}, the
posterior alignment model $Q$ can be computed as shown
in~\autoref{eq:posterior_prob}.
\begin{equation}
  \label{eq:posterior_prob}
\begin{aligned} \smaller[2]
Q_{\Psi}(a\mid c, R,w)= \frac{\exp(\sum_{i=1}^{n} \phi(g_{i}, h_{a_{i}}))} {Z_{\Psi}(c,w)}
\end{aligned}
\end{equation}
where $\phi(g_{i}, h_{a_{i}})$ score each alignment link between node i
and the corresponding words, $g_{i}$ is node encoding, and $h_{a_{i}}$
is encoding for the aligned token. However, the denominator $Z_{\Psi}$ is
an intratable partition fuction for the global normalization, which
has $n!$ combinations. Perturb-and-Max(MAP)
techinique~\cite{papandreouperturb} can help this intratable sampling
with the following two steps:
\begin{itemize}
\item Perturbing the alignment score~($\phi(g_{i}, h_{a_{i}})$) with an
  independent noise.
\item Computing the \kw{argmax} to sample the discrete alignment
  variable $a$.
\end{itemize}

\subsubsection{Gumbel Sinkhorn}
\label{sssec:lex-phr:gumbel-sinkhorn}
Finally, the remaining challenge lies in the discrete \kw{argmax} in
the second step of Perturb-and-Max, which is not differtiable.
\citet{mena2018learning} resolve this issue with a simple
differentiable operator called Gumbel-Sinkhorn operator. As the
exlcusive alignment assumption when we introduce the input
decomposition of lexical-anchoring
parsing~(\autoref{sssec:lex-phr:lex-input-decomposition}), the
\kw{argmax} over exclusive alignment can be relaxiated into a
differentiable gumbel softmax over $n*n$ matrix scores. For further
details of the estimated posterior alignment model, we refer the
reader to the original paper~\citep{mena2018learning,lyu2018amr}.

%%% Local Variables:
%%% mode: latex
%%% TeX-master: "../../dissertation-main.ltx"
%%% End:
