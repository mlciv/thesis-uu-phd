\section{Impact of Description Styles}
\label{sec:sgd:abl-desc}
%As shown in Figure \ref{fig:schema-dst} and the predefined notations
%in~\autoref{sec:chanllenges-in-sgd}, there are three components in the
%schema definitions: services $V$, intents $I$, slots $S$. Based on the
%given description, participators in DSTC8 schema-guided dialogue state
%tracking task use different compositions of service/intent/slots
%descriptions. For example, service description was used by the
%official baseline model for intent/slot matching, but not by the
%winning team~\cite{ma2019end}. The diverse design choices also exists
%in slot value matching.  The difference in architecture choice and
%state tracking strategy confounds different dialogue state tracking
%models, thus not comparable to discover which composition is the best
%for each tasks. On the other hand,
Previous work on schema-guided dialogue~\citep{rastogi2020schema} are
only based on the provided descriptions in \sgdst~dataset. Recent work
on modeling dialogue state tracking as reading
comprehension~\citep{gao2019dialog}~only formulate the descriptions as
simple question format with existing intent/slot names, it is unknown
how it performs when compared to other description styles. Moreover,
they only conduct homogeneous evaluation where training and test data
share the same description style. In this section, We also investigate
how a model trained on one description style will perform on other
different styles, especially in a scenario where chat-bot developers
may design their own descriptions. We first introduce different styles
of descriptions in our study, and then we train models on each
description style and evaluate on tests with corresponding homogeneous
and heterogeneous styles of descriptions. Given the best performance
of \CE~shown in the previous section and its popularity in DSTC8
challenges, we adopt it as our model architecture in this section.

\subsection{Benchmarking Styles and Experiment Setup}
\label{ssec:desc-styles}
For each intent/slot, we describe their functionalities by the
following different descriptions styles:

\begin{itemize}
\item \textbf{\ID:} This is the least informative case of name-based
description: we only use meaningless intent/slot identifiers,
\eg, \kw{Intent\_1}, \kw{Slot\_2}. It means we don't use description from any
schema component. We want to investigate how a simple identifier-based
description performs in schema-guided dialogue modeling, and the
performance lower-bound on transferring to unseen services.

\item \textbf{\NAMEONLY:} Using the original intent/slot names in~\sgdst
and~\multiwoz~datasets as descriptions, to show whether name is enough
for schema-guided dialogue modeling.

\item \textbf{\QANAMEONLY:} This is corresponding to previous
  work by~\citet{gao2019dialog}. For each intent/slot, it generate a
  question to inquiry about the intent and slot value of the
  dialogue. For each slot, it simply follows the template '{\it What
    is the value for {\bf slot i}?}'. Besides that, our work also
  extend the intent description by following the template ``{\it Is
    the user intending to {\bf intent j} }".

 \item \textbf{\ORIGIN:} The original descriptions in~\sgdst~and~\multiwoz~datasets.

 \item \textbf{\QARICH:} Different from the \QANAMEONLY, firstly it is
  based on the original descriptions; secondly, rather than always use
  the ``what is" template to inquiry the intent/slot value, We add
  ``what", ``which", ``how many" or ``when" depending on the entity
  type required for the slot.  Same as \QANAMEONLY, we just add
  prefixes as ``Is the user intending to\ldots'' in front of the original
  description. In a sum, this description is just adding question
  format to original description. The motivation of this description is
  to see whether the question format is helpful or not for
  schema-guided dialogue modeling.

\end{itemize}

To test the model robustness, we also create two paraphrased
  versions \textbf{\NAMEPARA}~and \textbf{\PARAPHRASE}~for
  \NAMEONLY~and~\ORIGIN, respectively. We first use
  nematus~\citep{sennrich-etal-2017-nematus} to automatically
  paraphrase the description with back translation, from English to
  Chinese and then translate back, then we manually check the
  paraphrase to retain the main meaning. Table
  \ref{tbl:sgd:schema-desc-ext} shows examples for different styles of
  schema descriptions.

  Unlike the composition used in Table \ref{tbl:sgd:schema-seq}, we
  don't use the service description to avoid its impact. For each
  style, we train separate models on 4 subtasks, then we evaluate them
  on different target styles in both homogeneous~(\autoref{ssec:sgd:homo-eval})
  and~heterogeneous settings~(\autoref{ssec:sgd:heter-eval})

\begin{table}[b]
\caption{\label{tbl:sgd:schema-desc-ext} Different extensions of schema descriptions.}
\begin{center}
\begin{tabular}{p{0.15\linewidth}|p{0.45\linewidth} | p{0.24\linewidth}}
\toprule
\hline
Style                         & Intent Description                                                                                               & Slot Description                                                                                                               \\ \hline
\ID                           & intent\_1                                                                                                        & slot\_4                                                                                                                        \\ \hline
\NAMEONLY                     & CheckBalance                                                                                                     & account\_type                                                                                                                  \\ \hline
\QANAMEONLY                   & Is the user intending to CheckBalance?                                                                           & What is the value of account\_type  ?                                                                                          \\ \hline
\ORIGIN                       & Check the amount of money in a user's bank account                                                               & The account type of the user                                                                                                   \\ \hline
\QARICH                       & Does the user want to check the amount of money in the bank account ?                                            & What is the account type of the user ?                                                                                         \\ \hline \hline
\NAMEPARA                     & CheckAccountBalance                                                                                              & user\_account\_type                                                                                                            \\ \hline
\PARAPHRASE                   & Check the balance of the user's bank account                                                                     & The type of the user account                                                                                                   \\ \hline
%\multirow{2}{*}{\CONSTRAINT} & \multirow{2}{*}{\parbox{5cm}{Check the amount of money in a user's bank account, requiring the type of account}} & \multirow{2}{*}{\parbox{5cm}{The account type of the user, required by mulitple intents, checking balance and transfer money}} \\
\bottomrule
\end{tabular}
\end{center}
\end{table}

%
%\Paragraph{\CONSTRAINT} To present the intent/slot dependecy
%  relation, add required slots or depended intent after the original
%  description. Multiwoz 2.1 will not support this style, because no
%  dependency information provided in the dataset.

%
%Take noncategorical slots subtask on \sgdst as an example, the results
%are summerized in {\bf last three columns} in Table
%\ref{tbl:style-results-sgd}. {\bf 75.63} under {\it self} column in
%{\bf NonCat} means the joint noncategorical accuracy we train the
%model with \NAMEONLY description and evaluate on \NAMEONLY
%description. However, $64.99\pm15.25$ means when we evaluate the
%\NAMEONLY model on two styles \NAMEONLY and \QANAMEONLY in {\it name}
%group, the average performance on {\it name} group is 64.99, while the
%large deviation 15.05 means the \NAMEONLY noncategorical slot model
%cannot perform consistent good on \QANAMEONLY style.  Finally, the
%last cell " in \ORIGIN row, $73.94\pm3.05$ means the model trained on
%\ORIGIN style actually perform good on \ORIGIN, \QARICH, \PARAPHRASE
%description style, the relative small value in standard deviation
%means relatively stable performance on {\it rich} description group.
\begin{table}[b]
\caption{\label{tbl:sgd:homo-style-results} Homogeneous evaluation results
  of different description style on \sgdst dataset and \multiwoz
  datasets. The middle horizontal line separate the two name-based
  descriptions and two rich descriptions in our settings. All
  numbers in the table are mixed performance including both seen and
  unseen services.}
\begin{center}{
\setlength{\tabcolsep}{2pt}
\begin{tabular}{c|cccc|cc}
  \toprule
  \hline
\multirow{2}{*}{Style\textbackslash{Task}} & \multicolumn{4}{c|}{ \sgdst } & \multicolumn{2}{c}{ \multiwoz }                                                                                                        \\
                                           & \multicolumn{1}{c}{ Intent}   & \multicolumn{1}{c}{Req} & \multicolumn{1}{c}{Cat} & \multicolumn{1}{c|}{NonCat} & \multicolumn{1}{c}{Cat} & \multicolumn{1}{c}{NonCat} \\ \hline
\ID                                        & 61.16                         & 91.48                   & 62.47                   & 30.19                       & 34.25                   & 52.28                      \\ \hline
\NAMEONLY                                  & {\bf 94.24}                   & 98.84                   & 74.01                   & 75.63                       & 53.72                   & 56.18                      \\
\QANAMEONLY                                & 93.31                         & {\bf 98.86}             & 74.36                   & 74.86                       & {\bf 54.19}             & 56.17                      \\ \hline
%\NAMEPARA                                 & 90.76                         &                         &                         &                             &                         &                            \\ \hline
\ORIGIN                                    & 93.01                         & 98.55                   & 74.51                   & 75.76                       & 52.19                   & 57.20                      \\
\QARICH                                    & 93.42                         & 98.51                   & {\bf 76.64}             & {\bf 76.60}                 & 53.61                   & {\bf 57.80}                \\
%\PARAPHRASE                                & 93.57                         & 98.43                   & 75.79                   & 71.94                       & 51.2                    & 56.52                      \\ \hline
  \hline
  \bottomrule
\end{tabular}
}
\end{center}
\end{table}

\subsection{Homogeneous Evaluation}
\label{ssec:sgd:homo-eval}
In this section, Table \ref{tbl:sgd:homo-style-results} summarizes the
performance for homogeneous evaluation, while Table
\ref{tbl:sgd:squad2-results-question} shows how the question style
description can benefit from SQuAD2 finetuning. Then we also conduct
heterogeneous evaluation on the other styles as
shown in Table \ref{tbl:sgd:hete-style-results-sgd}.~\footnote{We do not
  consider the meaningless \ID~style due to its bad performance.}
%
\subsubsection{Is Name-Based Description Enough?}
\label{sssec:sgd:question-name}
As shown in Table \ref{tbl:sgd:homo-style-results}, \ID~is the worst
case of using name description, its extremely bad performance
indicates name-based description can be very unstable. However, we
found that simple meaningful name-based description actually can
perform the best in \IC~and~\RSI~task, and they perform worse on
\CSL~and~\NSL~tasks comparing to the bottom two {\it rich}
descriptions.~\footnote{Only exception happens in \CSL~on
  \multiwoz. When creating \multiwoz~\citep{zang-etal-2020-multiwoz},
  the slots with less than 50 different slot values are classified as
  categorical slots, which leads to inconsistencies.} After careful
analysis on the intents in \sgdst~datasets, we found that most
services only contains two kinds of intents, an information retrieval
intent with a name prefix \tquoted{Find-,}~\tquoted{Get-,} and
\tquoted{Search-}; another transaction intent like \tquoted{Add-,}
\tquoted{Reserve-,} or \tquoted{Buy-.} Interestingly, we found that
all the intent names in the original schema-guided dataset strictly
follows an action-object template with a composition of words without
abbreviation, such as \tquoted{FindEvents,} \tquoted{BuyEventTickets.}
This simple name template is good enough to describe the core
functionality of an intent in \sgdst~dataset.~\footnote{This
  action-object template has also been found efficient for open domain
  intent induction task~\citep[\eg, OPINE][]{vedula2020open}.}
Additionally, \RSI is a relaitively simper task, requesting
information are related to specifial attributes, such as
\tquoted{has\_live\_music,} \tquoted{has\_wifi,} where keywords
co-occured in the slot name and in the user utterance, hence rich
explanation cannot help further. On the other side, {\it rich}
descriptions are more necessary for \CSL~and~\NSL~task. Because in
many cases, slot names are too simple to represent the functionalities
behind it, for example, slot name \tquoted{passengers} cannot fully represent
the meaning \tquoted{number of passengers in the ticket booking.}

\subsubsection{Does Question Format Help?}
\label{sssec:sgd:question-format}

As shown in~\autoref{tbl:sgd:homo-style-results}, when comparing row
\QARICH~v.s.~\ORIGIN, we found extra question format can improve the
performance on \CSL~and \NSL~task on both \sgdst~and
\multiwoz~datasets, but not for \IC~and \RSI~tasks. We believe that
question format helps the model to focus more on specific entities in
the dialogue history. However, when adding a simple question pattern to
\NAMEONLY, comparing row \QANAMEONLY~and \NAMEONLY, there is no
consistent improvement on both of the two datasets. Further more, we
are curious about whether BERT finetuned on SQuAD2~(SQuAD2-BERT) can
further help on the question format. Because \NSL~are similar with
span-based question answering, we focus on
\NSL~here. \autoref{tbl:sgd:squad2-results-question} shows that, after
applying the supplementary training on
SQuAD2~(\autoref{sec:sgd:sup-training}), almost all models get
improved on unseen splits however slightly dropped on seen
services. Moreover, comparing to \QANAMEONLY, \QARICH~is more similar
to the natural questions in the SQuAD2, we obverse that \QARICH~gains
more than \QANAMEONLY~from pretrained model on SQuAD2.

\begin{table}[!t]
\caption{\label{tbl:sgd:squad2-results-question} Performance changes when
  using BERT finetuned on SQuAD2 dataset to further finetuning on our
  \NSL~task.}
\begin{center}{
\setlength{\tabcolsep}{3pt}
\begin{tabular}{c|ccc|ccc}
  \toprule
  \hline
                         \multirow{2}{*}{Style/Dataset} & \multicolumn{3}{c}{\sgdst}  & \multicolumn{3}{c}{\multiwoz}  \\ \cline{2-7}
                                                        & all   & seen  & unseen      & all   & seen  & unseen      \\ \hline
 \multirow{1}{*}{\ORIGIN}                               & +1.99 & -1.79 & {\bf +3.25} & +1.93 & -2.21 & {\bf +4.27} \\ \
 \multirow{1}{*}{\QARICH}                               & +6.13 & -2.01 & {\bf +8.84} & +1.06 & -1.28 & {\bf +3.06} \\ \hline
 \multirow{1}{*}{\NAMEONLY}                             & -0.45 & -1.49 & -0.11       & +1.75 & +0.58 & {\bf +1.77}       \\
 \multirow{1}{*}{\QANAMEONLY}                           & +0.05 & -2.98 & {\bf +1.04} & -0.04 & -0.32 & {\bf +1.25} \\ \hline
  \bottomrule
\end{tabular}
}
\end{center}
\end{table}


\subsection{Heterogeneous Evaluation}
\label{ssec:sgd:heter-eval}
In this subsection, we first simulate a scenario when there is no
recommended description style for the future unseen services. Hence,
unseen services can follow any description style in our case. We
average the evaluation performance on three other descriptions and
summarized in Table \ref{tbl:sgd:hete-style-results-sgd}. The $\Delta$ column
shows the performance change compared to the homogeneous
performance. It is not surprising that almost all models perform worse
on heterogeneous styles than on homogeneous styles due to different
distribution between training and evaluation. The bold number shows
the best average performance on heterogeneous evaluation for each
subtask. The trends are similar with the analysis in homogeneous
evaluation~(\autoref{ssec:sgd:homo-eval}), the name-based descriptions perform
better than other rich descriptions on intent classification
tasks. While on other tasks, the \ORIGIN~description performs more
robust, especially on \NSL~task.

Furthermore, we consider another scenario where fixed description
convention such as \NAMEONLY~and \ORIGIN~are suggested to developers,
they must obey the basic style convention but still can freely use
their own words, such as abbreviation, synonyms, adding extra
modifiers. We train each model on \NAMEONLY~and~\ORIGIN, then evaluate
on the corresponding paraphrased version, respectively. In the last two
rows of Table \ref{tbl:sgd:hete-style-results-sgd}, the column `para'
shows performance on paraphrased schema, while $\Delta$ shows the
performance change compared to the homogeneous evaluation.
\ORIGIN~still performs more robust than \NAMEONLY~when schema
descriptions get paraphrased on unseen services.

\begin{table}[p]
  \caption{\label{tbl:sgd:hete-style-results-sgd} Results on unseen
    service with heterogeneous description styles on~\sgdst
    dataset. More results and qualitative analysis are
    in~\autoref{chap:appendix:schema-analysis}.}
\begin{center}{
\setlength{\tabcolsep}{3pt}
\begin{tabular}{c|c|c|c|c|c|c|c|c}
  \toprule
  \hline
\multirow{3}{*}{Style\textbackslash{Task}} & \multicolumn{8}{c}{ \sgdst }                                                                                                                                                 \\ \cline{2-9}
                                           & \multicolumn{2}{c|}{ Intent(Acc)} & \multicolumn{2}{c|}{Req(F1)} & \multicolumn{2}{c|}{Cat(Joint Acc)} & \multicolumn{2}{c}{NonCat(Joint Acc)}                               \\ \cline{2-9}
                                           & mean                              & $\Delta$                          & mean                                & $\Delta$         & mean        & $\Delta$         & mean        & $\Delta$         \\ \hline
\NAMEONLY                                  & 82.47                             & -11.47                       & 96.92                               & -1.64       & 61.37       & -5.54       & 56.53       & -14.68      \\
\QANAMEONLY                                & {\bf 93.27}                       & +0.58                        & {\bf 97.88}                         & -0.76       & 68.55       & +2.63       & 62.92       & -6.30       \\
\ORIGIN                                    & 79.47                             & -12.70                       & 97.42                               & -0.74       & {\bf 68.58} & -0.3        & {\bf 66.72} & -3.11       \\
\QARICH                                    & 84.57                             & -8.24                        & 96.70                               & -1.45       & 68.40       & -2.89       & 56.17       & -15.00      \\
  \hline
                                           & para                              & $\Delta$                          & para                                & $\Delta$         & para        & $\Delta$         & para        & $\Delta$         \\ \hline
\NAMEONLY                                  & {\bf 92.22}                       & -1.74                        & 97.69                               & -0.87       & 67.39       & -0.7        & 67.17       & -4.04       \\
\ORIGIN                                    & 91.54                             & {\bf -0.63}                  & {\bf 98.42}                         & {\bf +0.26} & {\bf 71.74} & {\bf +2.86} & {\bf 67.68} & {\bf -2.16} \\ \hline
  \bottomrule
\end{tabular}
}
\end{center}
\end{table}

%We first consider the diverse description
%within the same {\it name} or {\it rich} group. Within {\it name}
%group, models trained on \NAMEONLY~shows larger standard deviation,
%which indicating unsafe performance of just using simple name as
%description. Moreover, imaging an extremely bad name convention as
%{\bf ID} in section \ref{sssec:com-desc}, the model will lose
%transferbility on unseen service. Within {\it rich} group, we found
%our model trained on \PARAPHRASE~may perform worse than models trained
%on \ORIGIN~and~ \QARICH, because \PARAPHRASE~contains more
%perturbation on \ORIGIN~descriptions, although keeping the same
%meaning. This indicates BERT may fail to retain the similar
%representation for the perturbed sentence with the same meaning.
%Beside the intra-group evaluation, we also consider the inter-group
%hetergenuous evaluation. It is curious whether the model trained on
%{\it rich} description can still retain good performance on simple
%name-based description. We found this only happens on requested slot
%and categorical slot tasks, while on other subtasks, the performance
%dropped drastically. The performance drop is more significant on this
%task requires token-level representation predict the start and end
%position of the slot value, hence we believe this token-level tasks is
%more senstive than sentence-level tasks that depending on a summerized
%sentence representation.


%%% Local Variables:
%%% mode: latex
%%% TeX-master: "../../dissertation-main.ltx"
%%% End:
