\section{Related Work}
\label{sec:sgd:related-work}
Our work is related to three lines of research: multi-sentence
encoding, multi-domain and transferable dialog state
tracking. However, our focus is on the comparative study of different
encoder architectures, supplementary training, and schema
description style variation. Thus we adopt existing strategies from
multi-domain dialog state tracking.

\Paragraph{Multi-Sentence Encoder Strategies} Similar to the recent
study on encoders for response selection and article search
tasks~\citet{humeau2019poly}, we also conduct our comparative study on
the two typical architectures \CE~\cite{bordes2014open,
  lowe2015ubuntu} and
\DE~\cite{wu2017sequential,yang2018response}. However, they only focus
on sentence-level matching tasks. All subtasks in our case require
sentence-level matching between dialog context and each schema, while
the non-categorical slot filling task also needs to produce a sequence
of token-level representation for span detection. Hence, we study
multi-sentence encoding for both sentence-level and token-level
tasks. Moreover, to share the schema encoding across subtasks and
turns, we also introduce a simple \FE by caching schema token
embeddings in \S\ref{ssec:encoder-arch}, which improves efficiency
without sacrificing much accuracy.
%% YZ: not sure how the following point fits in here
%Beside that, rather
%than a provided set of candidate sentences to match in sentence
%selction tasks, we also study different compositions of schema
%components as candidates in \S\ref{sssec:com-desc}, e.g. compositions
%of service, intent, slot descriptions.

\Paragraph{Multi-domain Dialog State Tracking} Recent research on
multi-domain dialog system have been largely driven by the release of
large-scale multi-domain dialog datasets, such as
MultiWOZ~\cite{budzianowski2018multiwoz},
M2M~\cite{shah-etal-2018-bootstrapping}, accompanied by studies on key
issues such as in/cross-domain carry-over~\cite{
  kim2019efficient}. In this paper, our goal is to understanding the
design choice for schema descriptions in dialog state tracking.
Thus we simply follow the in-domain cross-over strategies used in {\bf
  TRADE}~\cite{wu2019transferable}. Additionally, explicit cross-domain
carryover~\cite{naik2018contextual} is difficult to generalize to new
services and unknown carryover links. We use longer dialog history to
inform the model on the dialog in the previous service. This
simplified strategy does impact our model performance negatively in
comparison to a well-designed dialog state tracking model on seen
domains. However, it helps reduce the complexity of matching extra slot
descriptions for cross-service carryover. We leave the further
discussion for future work.

\Paragraph{Transferable Dialog State Tracking} Another line of
research focuses on how to build a transferable dialog system that is
easily scalable to newly added intents and slots. This covers diverse
topics including e.g., resolving lexical/morphological variabilities
by symbolic de-lexicalization-based methods~\cite{henderson2014word,
  williams2016dialog}, neural belief
tracking~\cite{mrkvsic2017neural}, generative dialog state
tracking~\cite{peng2020soloist, hosseini2020simple}, modeling DST as a
question answering task~\cite{zhang2019find, lee2019sumbt,
  gao2020machine, gao2019dialog}. Our work is similar with the last
class. However, we further investigate whether the DST can benefit
from NLP tasks other than question answering. Furthermore, without
rich description for the service/intent/slot in the schema, previous
works mainly focus on simple format on question answering scenarios,
such as domain-slot-type compounded names~(e.g., ``{\it
  restaurant-food}"), or simple question template ``{\it What is the
  value for {\bf slot i}?}". We incorporate different description
styles into a comparative discussion on~\S\ref{ssec:desc-styles}.

%%% Local Variables:
%%% mode: latex
%%% TeX-master: "../../thesis-main.ltx"
%%% End:
