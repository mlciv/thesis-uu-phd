\section{Related Work}
\label{sec:sgd:related-work}
Our work is related to three lines of research: multisentence
encoding, multidomain and transferable dialogue state
tracking. However, our focus is on the comparative study of different
encoder architectures, supplementary training, and schema
description style variation. Thus we adopt existing strategies from
multidomain dialogue state tracking.

\subsection{Multisentence Encoder Strategies}
\label{ssec:sgd:multi-encoder}
Similar to the recent study on encoders for response selection and
article search tasks~\citet{humeau2019poly}, we also conduct our
comparative study on the two typical architectures
\CE~\citep{bordes2014open, lowe2015ubuntu} and
\DE~\citep{wu2017sequential,yang2018response}. However, they only focus
on sentence-level matching tasks. All subtasks in our case require
sentence-level matching between dialogue context and each schema, while
the noncategorical slot filling task also needs to produce a sequence
of token-level representation for span detection. Hence, we study
multisentence encoding for both sentence-level and token-level
tasks. Moreover, to share the schema encoding across subtasks and
turns, we also introduce a simple \FE by caching schema token
embeddings in~\autoref{ssec:encoder-arch}, which improves efficiency
without sacrificing much accuracy.
%% YZ: not sure how the following point fits in here
%Beside that, rather
%than a provided set of candidate sentences to match in sentence
%selction tasks, we also study different compositions of schema
%components as candidates in~\autoref{sssec:com-desc}, e.g. compositions
%of service, intent, slot descriptions.

\subsection{Multidomain Dialogue State Tracking}
\label{ssec:sgd:multidomain}
Recent research on multidomain dialogue system have been largely driven
by the release of large-scale multidomain dialogue datasets, such as
MultiWOZ~\citep{budzianowski2018multiwoz},
M2M~\citep{shah-etal-2018-bootstrapping}, accompanied by studies on key
issues such as in/cross-domain carry-over~\citep{ kim2019efficient}. In
this paper, our goal is to understanding the design choice for schema
descriptions in dialogue state tracking.  Thus we simply follow the
in-domain cross-over strategies used in {\bf
  TRADE}~\citep{wu2019transferable}. Additionally, explicit
cross-domain carryover~\citep{naik2018contextual} is difficult to
generalize to new services and unknown carryover links. We use longer
dialogue history to inform the model on the dialogue in the previous
service. This simplified strategy does impact our model performance
negatively in comparison to a well-designed dialogue state tracking
model on seen domains. However, it helps reduce the complexity of
matching extra slot descriptions for cross-service carryover. We leave
the further discussion for future work.

\subsection{Transferable Dialogue State Tracking}
\label{ssec:sgd:transfer-dialogue}
Another line of research focuses on how to build a transferable dialogue
system that is easily scalable to newly added intents and slots. This
covers diverse topics including, \eg, resolving lexical/morphological
variabilities by symbolic delexicalization-based
methods~\citep{henderson2014word, williams2016dialog}, neural belief
tracking~\citep{mrkvsic2017neural}, generative dialogue state
tracking~\citep{peng2020soloist, hosseini2020simple}, modeling DST as a
question answering task~\citep{zhang2019find, lee2019sumbt,
  gao2020machine, gao2019dialog}. Our work is similar with the last
class. However, we further investigate whether the DST can benefit
from NLP tasks other than question answering. Furthermore, without
rich description for the service/intent/slot in the schema, previous
works mainly focus on simple format on question answering scenarios,
such as domain-slot-type compounded names~(\eg, ``{\it
  restaurant-food}"), or simple question template ``{\it What is the
  value for {\bf slot i}?}". We incorporate different description
styles into a comparative discussion in~\autoref{ssec:desc-styles}.

%%% Local Variables:
%%% mode: latex
%%% TeX-master: "../../dissertation-main.ltx"
%%% End:
