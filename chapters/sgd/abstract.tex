\begin{abstract}
  Frame-based state representation is widely used in modern
  task-oriented dialog systems to model user intentions and slot
  values. However, a fixed design of domain ontology makes it
  difficult to extend to new services and APIs. Recent work proposed
  to use natural language descriptions to define the domain ontology
  instead of tag names for each intent or slot, thus offering a
  dynamic set of schema. In this paper, we conduct in-depth
  comparative studies to understand the use of natural language description for schema in
  dialog state tracking. Our discussion mainly covers three aspects:
  encoder architectures, impact of supplementary training, and effective
  schema description styles. We
  introduce a set of newly designed bench-marking descriptions and
  reveal the model robustness on both homogeneous and heterogeneous
  description styles in training and evaluation.
\end{abstract}
%%% Local Variables:
%%% mode: latex
%%% TeX-master: "main"
%%% End:
