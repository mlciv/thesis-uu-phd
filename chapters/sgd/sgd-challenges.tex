\section{Schema-Guided Dialog State Tracking}
\label{sec:chanllenges-in-sgd}
%Before introducing the details on schema-guided dialog, we recap the
%general concepts in dialog state tracking by defining notations
%used in our paper. We use superscripts to denote the indices of
%dialog utterances, services, intents and slots, and subscripts for indices of tokens
%in a sequence. Suppose $U^{T}=[u^{1}, u^{2}, ... u^{T}]$ are a
%set of $T$ history utterances in the dialog,
%$V^{K}=[v^{1}, v^{2}...v^{K}]$ is a set of $K$ service schema
%supported by a dialog agent. For each service $v^{k}$, it consists
%of a set intent labels $I^{M}=[i^{1}, i^{2}...i^{M}]$, and slots
%labels $S^{N}=[s^{1}, s^{2}...s^{N}]$. A classic dialog state tracking
%task is to predict a dialog state frame $f^{t,k}$ at turn $t$ for each
%service $s^{k}$ given the dialog history $U$.
A classic dialog state tracker predicts a dialog state frame at each
user turn given the dialog history and predefined domain ontology. As
shown in Figure \ref{fig:schema-dst}, the key difference between
schema-guided dialog state tracking and the classic paradigm is the
newly added natural language descriptions.  In this section, we first
introduce the four subtasks and schema components in schema-guided
dialog state tracking, then we outline the research questions in our
paper.

\Paragraph{Subtasks} As shown in Figure \ref{fig:schema-dst}, the
dialog state for each service consists of 3 parts: {\it active
  intent}, {\it requested slots}, {\it user goals~(slot
  values)}. Without loss of generality, for both \sgdst and \multiwoz
datasets, we divide their slots into categorical and non-categorical
slots by following previous study on
dual-strategies~\cite{zhang2019find}. Thus to fill the dialog state
frame for each user turn, we solve four subtasks: intent
classification~(\IC), requested slot identification~(\RSI), categorical
slot labeling~(\CSL), and non-categorical slot labeling~(\NSL). All
subtasks require matching the current dialog history with candidate
schema descriptions for multiple times.

\Paragraph{Schema Components} Figure \ref{fig:schema-dst} shows three
main schema components: service, intent, slot. For each intent, the
schema also describes {\it optional} or {\it required} slots for
it. For each slot, there are flags indicating whether it is
categorical or not. {\it Categorical} means there is a set of
predefined candidate values~(Boolean, numeric or text). For instance,
{\it has\_live\_music} in Figure~\ref{fig:schema-dst} is a categorical
slot with Boolean values. {\it Non-categorical}, on the other hand,
means the slot values are filled from the string spans in the dialog
history.

\Paragraph{New Questions} These added schema descriptions pose
the following three new questions. We discuss each of them in the
following sections.
\\
\setlength{\fboxsep}{1.5pt}
\fbox{\small\parbox{0.90\textwidth}{
\begin{enumerate}[label=Q\arabic*.]
\item How should dialogue and schema be encoded? \S\ref{sec:models}
\item How do different supplementary trainings impact each subtask? \S\ref{sec:sup-training}
\item How do different description styles impact the state tracking performance? \S\ref{sec:abl-desc}
\end{enumerate}
}}

%%% Local Variables:
%%% mode: latex
%%% TeX-master: "../../thesis-main.ltx"
%%% End:
