\section{Supplementary Training}
\label{sec:sgd:sup-training}
Besides the pretrain-fintune framework used in~\autoref{sec:sgd:models},
\citet{phang2018sentence} propose to add a supplementary training
phase on an intermediate task after the pretraining, but before
finetuning on target task. It shows significant improvement on the
target tasks. Moreover, large amount pretrained and finetuned
transformer-based models are publicly accessible, and well-organized
in model hubs for sharing, training and testing\footnote{e.g.,
  Huggingface(\url{https://huggingface.co/models}) and
  ParlAL(\url{https://parl.ai/docs/zoo.html}), etc.}.  Given the new
task of schema-guided dialog state tracking, in this section, we study
our four subtasks with different intermediate tasks for supplementary
training. However, how to choose those intermediate tasks is a
challenge problem. Our inductive biases on choosing intermediate tasks
are: \textit{Intermediate tasks that sharing similar structures with
  downsteam tasks may help the supplementary training.}  Hence, for
natural language description modelling, we will first show the
similarity between our proposed \textbf{Natural Language inference}
and \textbf{Question Ansering} with our four subtasks, then analyze
the results on how they performed.

\subsection{Intermediate Tasks}
\label{ssec:intermediate-tasks}
As described in \S~\ref{ssec:models-overview}, all our 4
subtasks take a pair of dialog and schema description as input, and
predict with the summerized sentence-pair~\CLS~representation. While
\NSL~also requires span-based detection such as question
answering. Hence, they share the similar problem structure with the
following sentence-pair encoding tasks.
\begin{itemize}
\item \textbf{Natural Language Inference:} Given a hypodissertation/premise
sentence pair, natural language inference is a task to determine
whether a hypodissertation is entailed, contradicted or neutral given
that premise.

\item \textbf{Question Answering:} Given a passage/question pairs, the
task is to extract the span-based answer in the passage.
\end{itemize}

Hence, when finetuning BERT on our subtaks, instead of directly using
the originally pretrained BERT, we use the BERT
finetuned on the above two tasks for further finetuning.  Due to better
peformance of \CE in~\autoref{ssec:models-overview}, we directly use the finetuned
\CE~version of BERT models on SNLI and SQuAD2.0 dataset from
Huggingface model hub. We add extra speaker tokens [user:] and
[system:] into the vocabulary for encoding the multi-turn dialog
histories.
%
%\Paragraph{Response Selection} Given a dialog history and a set of
%candidate response, the task is to select the best response follow the
%dialog hisotry. We didn't find any public BERT Cross-encoder model
%finetuned on this task. Hence, we finetuned a cross-encoder model on
%ConvAI2 dataset with code provided in ParlAI.

\subsection{Results on Supplementary Training}
\label{ssec:sgd:results-secondary}

\begin{table}[!t]
\begin{center}{
\setlength{\tabcolsep}{2pt}
\begin{tabular}{c|ccc|ccc|ccc|ccc}
  \toprule
  \hline
                       & \multicolumn{12}{c}{ \sgdst } \\ \cline{2-13}
                       & \multicolumn{3}{c|}{ intent }  & \multicolumn{3}{c|}{ req } & \multicolumn{3}{c|}{ cat } & \multicolumn{3}{c}{ noncat } \\ \cline{2-13}
                       & all                            & seen                       & unseen                     & all                           & seen                       & unseen & all   & seen  & unseen & all        & seen  & unseen     \\ \hline
  $\Delta_{\textbf{SNLI}}$  & {\bf +0.51}                    & +0.02                      & {\bf +0.68}                & -0.19                         & +0.38                      & -0.38  & -1.63 & -2.87 & -1.23  & -4.7       & -0.1  & -6.25      \\ \hline
  $\Delta_{\textbf{SQuAD}}$ & -1.81                          & -0.17                      & -1.32                      & -0.25                         & -0.01                      & -0.33  & -2.87 & -3.02 & -5.17  & {\bf +1.99} & -1.79 & {\bf +3.25} \\ \hline
  \bottomrule
\end{tabular}
}
\end{center}
\caption{\label{tbl:sup-training-sgd} Relative performance improvement of different supplementary training on \sgdst dataset}
\end{table}
\begin{table}[!t]
\begin{center}{
\setlength{\tabcolsep}{2pt}
\begin{tabular}{c|ccc|ccc}
  \toprule
  \hline
                       & \multicolumn{6}{c}{ \multiwoz }                                                                                                                                                                                                                      \\ \cline{2-7}
                       & \multicolumn{3}{c|}{ cat } & \multicolumn{3}{c}{ noncat }                                                                                                  \\ \cline{2-7}
                       & all   & seen  & unseen & all        & seen  & unseen      \\ \hline
  $\Delta_{\textbf{SNLI}}$  & +2.05 & +0.6  & --0.7   & {\bf +3.64} & +1.05 & {\bf +4.84} \\ \hline
  $\Delta_{\textbf{SQuAD}}$ &  +0.04 & -0.71 & +0.41  & {\bf +1.93} & -2.21 & {\bf +4.27} \\ \hline
  \bottomrule
\end{tabular}
}
\end{center}
\caption{\label{tbl:sup-training-multiwoz} Relative performance improvement of different supplementary training on \multiwoz~dataset}
\end{table}

\autoref{tbl:sup-training-sgd} and \autoref{tbl:sup-training-multiwoz}
shows the performances gain when finetuning 4 subtasks based on models
with the above SNLI and SQuAD2.0 supplementary training.

We mainly find that SNLI helps on \IC~task, SQuAD2 mainly helps on
\NSL~task, while neither of them helps much on \CSL~task. Recently,
\citet{namazifar2020language} also found that when modeling dialog
understanding as question answering task, it can benefit from a
supplementary training on SQuAD2 dataset, especially on few-shot
scenarios, which is a similar findings as our \NSL~task. Result
difference on \RSI~task is minor, because it is a relatively easy
task, adding any supplementary training did n't help much. Moreover,
for \CSL~task, the sequence 2 of the input pair is the slot
description with a categorical slot value, thus the meaning overlapping between the
full dialog history and the slot/value is much smaller than SNLI
tasks. On the other side, \CLS~token in SQuAD BERT is finetuned for
null predictions via start and end token classifers, which is
different from the the single CLS classifer in \CSL~task.

%%% Local Variables:
%%% mode: latex
%%% TeX-master: "../../dissertation-main.ltx"
%%% End:
