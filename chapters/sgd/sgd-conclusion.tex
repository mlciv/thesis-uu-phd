\section{Conclusion}
\label{sec:conclusion}
In this paper, we studied three questions on schema-guided dialog
state tracking: encoder architectures, impact of supplementary
training, and effective schema description styles.  The main findings
are as follows:

By caching the token embedding instead of the single \textsc{cls}
embedding, a simple partial-attention \FE can achieve much better
performance than \DE, while still infers two times faster than \CE.
We quantified the gain via supplementary training on two intermediate
tasks.  By carefully choosing representative description styles
according to recent works, we are the first of doing both
homogeneous/heterogeneous evaluations for different description style
in schema-guided dialog. The results show that simple name-based
description performs well on \IC~and \RSI~tasks, while \NSL~tasks
benefits from richer styles of descriptions.  All tasks suffer from
inconsistencies in description style between training and test, though
to varying degrees.

Our study are mainly conducted on two datasets: \sgdst~and~\multiwoz,
while the speed-accuracy balance of encoder architectures and the
findings in supplementary training are expected to be
dataset-agnostic, because they depend more on the nature of the
subtasks than the datasets. Based on our proposed benchmarking
descriptions suite, the homogeneous and heterogeneous evaluation has
shed the light on the robustness of cross-style schema-guided dialog
modeling, we believe our study will provide useful insights for future
research.



%%% Local Variables:
%%% mode: latex
%%% TeX-master: "main"
%%% End:
